%\documentclass[MScThesis,twoside]{sharifthesis}
%\documentclass[MScThesis,oneside]{sharifthesis}
\documentclass[MScThesis,twoside]{sharifthesis}
%\documentclass[PhDThesis,oneside]{sharifthesis}
%\documentclass[PhDProposal,twoside]{sharifthesis}
%\documentclass[PhDProposal,oneside]{sharifthesis}

%\def\enSubject{Master of Science Thesis (Information Technology Major), Computer Engineering Department, Sharif University of Technology, Tehran, I. R. Iran}
\def\enSubject{Doctor of Philosophy Thesis (Information Technology Major), Computer Engineering Department, Sharif University of Technology, Tehran, I. R. Iran}
%\def\enSubject{Doctor of Philosophy Research Proposal (Information Technology Major), Computer Engineering Department, Sharif University of Technology, Tehran, I. R. Iran}

%do not use newline command (\\) in following definition
\def\enTitle{The Bug Prediction Model Based On Mutation Metrics}
%if title is long and requires to be splitted in two lines, uncomment following two definitions and split title at appropriate location (you should uncomment two corresponding lines in the rest of this file too)
%\def\enTitleLineOne{First line of the long title}
%\def\enTitleLineTwo{and its continuation on the second line}
\def\enAuthor{‌Ali Mohebbi}
\def\enKeywords{Bug Prediction, Software Testing, Mutation Metrics, Process Metrics}


\input{general/preamble}
%\addbibresource{resources/resources.bib}
\bibliography{resources/resources}

\newcommand{\faKeywords}{
پیش‌بینی خطا، آزمون نرم‌افزار، معیارهای جهش، معیارهای فرآیند
}
\eqcommand{واژه‌های‌کلیدی}{faKeywords}
\آرم{\درج‌تصویر[scale=.7]{logo}}
\تاریخ{شهریور 1397}
%در دستور زیر، از \\ استفاده نکنید
\عنوان{مدل پیش بینی خطا مبتنی بر معیارهای جهش}
%اگر عنوان طولانی بوده (و در عنوان انگلیسی از دو خظ استفاده شده) باید دو خط زیر از کامنت خارج و دو خط عنوان توسط آن‌ها تعریف گردد.
%\عنوانخطیک{خط نخست عنوان طولانی}
%\عنوانخطدو{و ادامه‌ی آن در خط دوم}
\نویسنده{علی محبی   \newline}
\دانشگاه{{\نستعلیق\درشت‌تر دانشگاه صنعتی شریف %
\\[0.6cm]}
دانشکده‌ی مهندسی کامپیوتر}
\دانشگاه‌عادی{دانشگاه صنعتی شریف\\
دانشکده‌ی مهندسی کامپیوتر}
\موضوع{مهندسی نرم‌افزار}
\استادراهنما{دکتر حسن میریان}
%اگر استاد مشاور دارید، خط زیر را از comment خارج کنید
%همچنین، فراموش نکنید در آخر این پرونده، اطلاعات انگلیسی معادل این دستورها را هم پر کنید
%\استادمشاور{دکتر <نام استاد مشاور>}

\newcommand{\efootnote}[1]{\footnote{\lr{#1}}}
\newcommand{\ecfootnote}[1]{}


% ================ Correct hyphenations ================
\hyphenation{test}


\makeglossaries
%\includeonly{related_works/related_works}
%\includeonly{evaluation/evaluation}

% ===== DEPRACATED AREA =====
% Following commands are provided to make older documents compilable.
% These commands are depracated and should not be used in new documents.
\newcommand{\ترجمه‌ج}[2]
{\ترجمه[#1‌ها]{#1}{#2}}
\newcommand{\ترجمه‌جمع}[3]
{\ترجمه[#3]{#1}{#2}}
\newcommand{\برگردان}[3]
{\ترجمه{#1}{#3}\زیرنویس{#2}}
\eqcommand{اسم}
{نام}
% ===== END OF DEPRACATED AREA =====

\شروع{نوشتار}


\newcommand{\StartDocument}{\frontmatter \baselineskip1.2\baselineskip \pagestyle{empty} \null \vfill
\شروع{وسط‌چین}
{\نستعلیق‌درشت بسم اللّه الرّحمن الرّحیم}
\پایان{وسط‌چین}
\vfill}

%the initial title is supposed to be printed on the cover.
%for non final version, you can leave following commands as is to create only one title page (printed on paper)
%for final version you need to swap folowing commented/uncommented makethesistitle commands to achieve this order: title on the cover THEN in the name of god page THEN another title page but this time printed on paper
\makethesistitle
\StartDocument
%\makethesistitle

\pagestyle{plain.ThesisPagestyle}
% following parts are not required in PhD proposal and should be removed. BEGIN OF COMMENT FOR PhD Proposal........
%صفحه‌ی تصویب در پیشنهاد پژوهشی وجود ندارد.
\شروع{تصویب}
%خط‌های زیر در صورت نبود استاد مدعو comment شوند
\داور{استاد مدعو}{دکتر <نام استاد مدعو ۱>}
\داور{استاد مدعو}{دکتر <نام استاد مدعو ۲>}
\پایان{تصویب}
\newpage
% برگه‌ی اظهارنامه را که به صورت خالی با دستور زیر ایجاد می‌شود، پس از چاپ با خودکار پر کرده و امضاء کنید.
\اظهارنامه{رساله}{دکتری}
%\اظهارنامه{پایان‌نامه}{کارشناسی~ارشد}
\newpage

\تقدیم{\درشت تقدیم به پدر و مادر مهربانم که همواره پشتیان و مایه‌ی دلگرمی در تمام مراحل زندگی بوده‌اند.}

\setlength{\baselineskip}{0.9cm}
\begin{comment}
\فصل*{پیش‌گفتار}
\thispagestyle{plain.ThesisPagestyle}
پیشگفتار اختیاری است. در صورت تمایل به نگارش پیش‌گفتار، محیط کامنت که آن را دربرگرفته باید حذف شود.
\end{comment}

\شروع{قدردانی}
از زحمات استاد فرهیخته دکتر سیدحسن میریان حسین‌آبادی که راهنما و راهگشای اینجانب در انجام این پایان‌نامه بوده‌اند، بدین وسیله تقدیر و تشکر می‌نمایم.
همچنین از زحمات دوست گرانقدرم مهران ریواده که با راهنمایی خویش مرا یاری نمودند، تشکر می‌کنم.
لازم است در اینجا از زحمات دوست عزیزم خشایار اعتمادی که در همفکری و کمک به من نقش مهمی داشته‌اند، قدردانی به عمل آورم.
\پایان{قدردانی}
% END OF COMMENT FOR PhD Proposal.

\شروع{چکیده}{\واژه‌های‌کلیدی}
% abstract ...
% write it at the end...
%persian abstract

 توسعه‌دهندگان نرم‌افزار از طریق گزارش خطا در سیستم­‌های ردگیری خطا و یا شکست در آزمون نرم‌‌افزار متوجه حضور خطا می‌شوند و پس از آن به جستجوی محل خطا و درک مشکل  نرم‌‌افزار می‌‌پردازند. کشف زود هنگام خطاها موجب صرفه‌‌جویی در زمان و هزینه می­‌شود و فرآیند اشکال زدایی را تسهیل می‌­بخشد. ابزارهای آماری نوین امکان ساخت و بهره‌‌برداری از مدل‌های پیش‌بینی را فراهم می‌سازند. اصلی‌ترین جزء مدل‌های پیش‌بینی، معیارهای نرم‌افزار می‌باشد و با به کارگیری معیارهای نوین و موثر می‌توان به مدل‌های کاراتر دست پیدا کرد. در این پایان‌نامه از معیارهای فرآیند و معیارهایی که بر اساس تحلیل جهش ساخته شده‌اند استفاده شده  و عملکرد مدل‌های حاصل ارزیابی شده‌اند. علاوه بر بکارگیری معیارهای جهش در کنار معیارهای فرآیند دو دسته معیار جدید به نام‌های معیارهای \موکد{فرآیند مبتنی بر جهش } و معیارهای \موکد{ترکیبی جهش-فرآیند} نیز جهت به کارگیری در ساخت مدل‌های پیش‌بینی معرفی شده‌اند. نتایج ارزیابی نشان می‌دهد معیارهای جهش می‌تواند به قدرت پیش‌بینی معیارهای فرآیند بیافزاید. معیارهای فرآیند مبتنی بر جهش علی‌رغم داشتن قدرت پیش‌بینی بهتر از معیارهای جهش عمل نمی‌کنند. همچنین معیارهای ترکیبی جهش-فرآیند بهبود قابل توجهی را در عملکرد مدل‌های پیش بینی ایجاد می‌کنند. 

\پایان{چکیده}


\setlength{\baselineskip}{0.9cm}
\pagenumbering{tartibi}\tableofcontents\listoftables\listoffigures
%list of abbreviations may be added here...


\PrepareForMainContent
\pagestyle{ThesisPagestyle}
%	In the name of GOD

	
\chapter{سرآغاز}
\baselineskip=.95cm
\برچسب{chap:intro}
سامانه‌های نرم‌افزاری بسیار فراگیر شده‌اند و زندگی امروزی را ارتقا داده‌اند. در نتیجه کاربران کیفیت نرم‌افزار بالایی را تقاضا می‌کنند. کشف و برطرف کردن خطاها پرهزینه است و مدل‌های پیش‌بینی خطا  از طریق اولویت‌دهی به فعالیت‌های تضمین کیفیت موجب افزایش بازدهی می‌گردند. پیش‌بینی خطا از سال ۱۹۹۲ تا کنون یک زمینه‌ی فعال تحقیقاتی بوده است. محققان همواره به دنبال روش‌هایی بوده‌اند که پیش‌بینی خطا را با کیفیت بهتری انجام دهند و یا دامنه‌ی کاربرد آن را گسترش بخشند. 

به  منظور افزایش کارایی پیش‌بینی‌خطا محققان معیارهای نوینی را ارائه داده‌اند\cite{bacchelli2010popular}، سعی داشته‌اند محدودیت‌های یادگیری ماشین را تقلیل بخشند\cite{limsettho2018cross} و یا روش‌های بروزتری را به منظور \واژه{دسته‌بندی} به کار گیرند\cite{chen2016software}. 

\section{ تعاریف مقدماتی}
\label{sec:terms}
در این قسمت چند اصطلاح رایج در مبحث پیش‌بینی‌ خطا و مورد استفاده در پایانامه نوشته شده است.
\begin{itemize}
	\item 
	\واژه{مورد آزمون}: \\
	یک مورد آزمون متشکل است از مقادیر ورودی‌های آزمون، نتایج مورد انتظار که با اجرای برنامه تحت آزمون یک یا چند عملکرد آنرا ارزیابی می‌کند. 
	\item
	\واژه{سامانه‌ی کنترل نسخه}:\\
	این سامانه تغییرات اعمال شده بر روی یک یا چندین \واژه{پرونده} را ذخیره می‌کند تا در آینده بتوان یک نسخه‌ی خاص را بازخوانی کرد. 
	\item
	\واژه{ثبت}:\\
	ذخیره‌ی تغییرات ایجاد شده بر روی پرونده‌ها در سامانه‌ی کنترل نسخه‌ را ثبت می‌نامند. یک ثبت را می‌تواند معادل یک نسخه از برنامه در نظر گرفت که البته این نسخه می‌تواند ناکامل باشد.
	\item
	\واژه{انتشار}:\\
	انتشار به معنی توزیع نسخه‌ی نهایی یک نرم‌افزار است که قابل استفاده برای کاربر می‌باشد. یک انتشار ممکن است نسخه‌ای از یک برنامه‌ی جدید باشد و یا ارتقاء یافته‌ی نرم‌افزار موجود باشد. قبل از یک انتشار معمولا به ترتیب نسخه‌های \نام{آلفا}{Alpha} و \نام{بتا}{Beta} توزیع می‌شود. 
	\item
	\واژه{ماتریس درهم‌ریختگی}:\\
	در زمینه‌ی یادگیری ماشین، به خصوص مسئله‌ی دسته‌بندی،یک ماتریس درهم‌ریختگی یک جدول است که اجازه می‌دهد عملکرد یک الگوریتم تصویر‌سازی گردد. هر سطر از ماتریس نشان دهنده‌ی نمونه‌هایی است که پیش‌بینی شده‌اند در حالی که هر ستون نمونه‌ها در کلاسهای واقعی را نشان می‌دهند(یا بالعکس). این ماتریس با توجه به این واقعیت نامگذاری شده است که اجازه می‌دهد به سادگی مشخص شود که آیا یک سیستم دو کلاس را با هم اشتباه گرفته است یا خیر. ماتریس درهم ریختگی برای دسته‌بندی دو کلاس فرضی (آ) و (ب) در جدول \ref{tab:confusion-matrix} آمده است. \\
	در این جدول نمونه‌هایی که در دسته‌ی آ قرار می‌گیرند مثبت در نظر گرفته شده‌اند. این ماتریس از چهار عنصر اصلی تشکیل شده است که در زیر شرح داده شده اند. 
	\begin{itemize}
		\item 
		\واژه{مثبت واقعی}: تعداد نمونه‌هایی را نشان می‌دهد که به درستی در دسته‌ی آ پیش‌بینی شده‌اند.
		\item
		\واژه{مثبت اشتباه}: تعداد نمونه‌هایی را نشان می‌دهد که در دسته‌ی آ پیش‌بینی شده‌اند اما در واقع در دسته‌ی ب قرار دارند.
		\item
		\واژه{منفی اشتباه}: تعداد نمونه‌هایی را نشان می‌دهد که در دسته‌ی ب پیش‌بینی شده‌اند اما در واقع در درسته‌ی آ قرار دارند.
		\item
		\واژه{منفی واقعی}: تعداد نمونه‌هایی را نشان می‌دهد که به درستی در دسته‌ی ب پیش‌بینی شده‌اند.
	\end{itemize}
	
\end{itemize}	

\begin{table}[H] 
	\renewcommand*{\arraystretch}{1.5}	
	\centering \caption{ماتریس درهم‌ریختگی}
	\newcolumntype{C}{>{\centering\arraybackslash} m } 
	\label{tab:confusion-matrix}

	\begin{tabular}{|C{2cm}|c|C{3cm}|C{3cm}|}
		
		\cline{3-4}
		\multicolumn{2}{c|}{}
	&	\multicolumn{2}{c|}{دسته‌ی واقعی}
		\\
	
		\cline{3-4}
		\multicolumn{2}{c|}{}
  & دسته‌ی آ  &  	دسته‌ی ب
		\\ \hline
		
		\multirow{2}{*}{\shortstack{
		دسته‌ی\\
			 پیش‌بینی\\
			  شده}
	} \rule{0pt}{6ex}   & 
	
دسته‌ی آ &مثبت واقعی \cellcolor{green!15} &مثبت اشتباه \cellcolor{red!15} 
		\\ \cline{2-4}\rule{0pt}{6ex}  & 

دسته‌ی ب   &منفی اشتباه \cellcolor{red!15}  & منفی واقعی \cellcolor{green!15} 

		\\
		\hline
		
	\end{tabular}
\end{table}


\section{بیان مسئله}
آزمون نرم‌افزار اصلی‌ترین فعالیت تیم  تضمین کیفیت می‌باشد. آزمون نرم‌افزار می‌تواند تا ۵۰ درصد هزینه‌ی تولید نرم‌افزار را به خود اختصاص دهد. هدف از پیش‌بینی خطا افزایش بازدهی این فرآیند می‌باشد. حال با بهبود پیش‌بینی خطا می‌توان به دستیابی به این هدف کمک نمود. به منظور پیش‌بینی خطا معیارهایی  در سطح مورد نظر استخراج می‌گردد. منظور از سطح مورد نظر سطوح مختلف برنامه مانند زیر‌سیستم، \واژه{بسته}، \واژه{پرونده} و تابع می‌باشد. سپس با استفاده از دسته‌بندی، خطادار بودن یا نبودن قطعه‌ی مورد بررسی پیش‌بینی می‌شود. یک دسته از معیارهای مورد استفاده در این زمینه معیارهای فرآیند است و معیارهای جهش نیز به تازگی در این راستا استفاده شده‌اند. این پایانامه قصد دارد تا بررسی کند که معیارهای جهش در کنار فرآیند  چه میزان در پیش‌بینی خطا تاثیر گذار است و همچنین بر اساس مفاهیم تحلیل جهش معیارهای جدیدی ارائه دهد تا پیش‌بینی خطا بهبود یابد. \\

با توجه به اینکه معیارهای جهش به تازگی در پیش‌بینی خطا مورد استفاده قرار گرفته‌اند لازم است تا تحقیقات بیشتری در مورد آنها صورت گیرد و عملکرد آنها از ابعاد مختلف مورد بررسی قرار گیرد. همچنین با بررسی مطالعات پیشین نقاط ضعف و قوت معیارهایی که تا کنون ارائه شده‌اند مورد بررسی قرار می‌گیرد. در این پایانامه عملکرد معیارهای فرآیند و جهش مورد بررسی بیشتری قرار می‌گیرند و با توجه به نقاط ضعف و قدرت معیارهای قبلی، معیارهایی ارائه شده تا بخشی از نقاط ضعف را پوشش دهند و به پیش‌بینی بهتری بیانجامند. 
\section{ساختار پایانامه}

این پایانامه در ۶ فصل تهیه گردیده است. در فصل \ref{chap:survey} به مرور مطالعات پیشین پرداخته می‌شود که در  قسمت \ref{sec:bug-predict}  مباحث مربوط به پیش‌بینی خطا از جمله فرآیند پیش‌بینی، معیارهای ارزیابی، معیارهای پیش‌بینی و مدل‌های پیش‌بینی بررسی می‌شوند. در قسمت \ref{sec:mutation} مباحث مربوط به آزمون جهش بررسی شده‌اند و در قسمت \ref{sec:conclustion} مطالعات مروری جمع‌بندی شده‌اند. در فصل \ref{chap:method} معیارهای مورد استفاده و ارائه شده در این پایانامه معرفی می‌شوند. در فصل \ref{chap:case-study} پنج پروژه‌ی صنعتی مورد مطالعه قرار گرفته‌اند و در فصل \ref{chap:evaluation} معیارها مورد ارزیابی قرار گرفته‌اند. در فصل \ref{chap:future} مباحث مطرح شده در این پایانامه جمع‌بندی شده و کارهای آتی شرح داده شده است. 
	

\chapter{مرور مطالعات پیشین}

\label{sec:survey}

\section{پیش بینی خطا}
\section{پیش بینی خطا}
\label{sec:bug-predict}
در این قسمت ابتدا نحوه‌ی پیش‌بینی خطا به طور کلی شرح داده می‌شود. سپس معیارهای متداول جهت ارزیابی مدل‌های پیش‌بینی بررسی می‌شوند. همانطور که اشاره شد جهت پیش‌بینی لازم است که معیارهای از کد استخراج شود این معیارها به دو دسته‌ی کلی معیارهای کد و معیارهای فرآیند تقسیم می‌گردند. معیارهای مختلف معرفی شده در پژوهش‌های محتلف معرفی می‌شوند. در انتها از مدل‌های که جهت پیش‌بینی استفاده می‌گردد بازبینی می‌شوند. 

\subsection{فرآیند پیش‌بینی خطا}
\label{subsec:process}
اکثریت پژوهش‌های پیش‌بینی خطا از روش‌های یادگیری ماشین  استفاده کرده‌اند. اولین گام در ساخت مدل پیش‌بینی تولید داده‌هایی با استفاده از آرشیو‌های نرم‌افزاری همانند \واژه[سامانه‌های کنترل نسخه]{سامانه‌ی کنترل نسخه} مانند \نام{گیت}{Git}، سیستم‌های ردگیری مشکلات  مانند \نام{جیرا}{Jira} و آرشیو ایمیل‌ها است. هر یک از این داده‌ها بر اساس درشت دانگی پیش‌بینی می‌توانند نمایانگر یک سیستم، یک \واژه[قطعه‌ی]{قطعه} نرم‌افزاری، \واژه{بسته}، فایل کد منبع، کلاس و یا تابع باشد. مقصود از داده یک بردار ویژگی حاوی چندین معیار (یا ویژگی) می‌باشد که از آرشیو‌های نرم‌افزاری استخراج شده و دارای برچسب \موکد{سالم} و  \موکد{خطادار}  و یا تعداد خطاها است. پس از تولید داده‌ها با استفاده از معیارها و برچسب‌ها می‌توان به پیش پردازش داده‌ها پرداخت (مانند انتخاب معیار) که البته این امر اختیاری می‌باشد. پس از بدست آوردن مجموعه‌ی نهایی داده‌ها یک مدل پیش‌بینی را آموزش می‌دهیم که می‌تواند پیش‌بینی کند یک داده‌ی جدید حاوی خطا است یا خیر. تشخیص \واژه[خطاخیز]{خطا‌خیزی} بودن داده معادل \واژه{دسته‌بندی دوتایی}. مقصود از دسته‌بندی دوتایی، دسته‌بندی عناصر مجموعه‌ی داده شده به دو گروه مجزا می‌باشد. همچنین پیش‌بینی تعداد خطاها معادل \واژه{رگرسیون} می‌باشد.  منظور از رگرسیون فرآیند آماری است که در آن با استفاده از متغیرهای مستقل سعی می‌شود متغیر وابسته تخمین زده شود که در اینجا متغیرهای مستقل معیارهای پیش‌بینی خطا و معیار وابسته تعداد خطاها می‌باشد. 

در شکل \ref{fig:prediction-process} فرآیند پیش‌بینی خطا نشان داده شده است. داده‌ها نمونه‌هایی هستند که می‌توانند خطادار  و بدون  خطا  بودن(   \lr{B = buggy} یا   \lr{C = clean} ) و یا تعداد خطا را نشان دهند. لازم به ذکر است که در یک مدل پیش‌بینی تنها از یک نوع از این داده‌ها استفاده می‌شود.
\begin{figure}[H]
	\centering
	\includegraphics[trim={2cm 5cm 2cm 5cm},clip,width=1.0\textwidth]{img/prediction-process.pdf}
	 \caption{فرآیند پیش‌بینی خطا \cite{nam2014survey}}
	\label{fig:prediction-process}
\end{figure}
\subsection{اندازه‌های ارزیابی}
\label{subsec:eval}
معیارهای ارزیابی را می‌توان به دسته‌ی کلی  معیارهای دسته‌بندی و رگرسیون تقسیم کرد.  معیارهای دسته بندی را می‌توان با استفاده از \واژه{ماتریس درهم‌ریختگی} محاسبه نمود. در ماتریس درهم ریختگی پیش‌بینی خطا، عناصر  به صورت زیر تعریف می‌شوند.  همچنین نحوه‌ی محاسبه‌ی معیارها در جدول \ref{tab:eval-metircs} آمده است. 
\begin{itemize}
	\setlength\itemsep{.01em}
\item 
مثبت واقعی: 
تعداد داده‌های حاوی خطا که به درستی تشخیص داده شدند
\item
مثبت اشتباه:
تعداد داده‌های سالم که به عنوان خطادار پیش‌بینی شدند
\item 
منفی اشتباه:
تعداد داده‌های سالم که به درستی تشخیص داده شدند
\item 
منفی واقعی:
تعداد داده‌های حاوی خطا که به عنوان داده‌ی سالم پیش‌بینی شدند

\end{itemize}


\begin{table}[H] 
		\renewcommand*{\arraystretch}{1.5}	
	\centering \caption{فرمول‌های محاسبه‌ی معیارهای ارزیابی}
	\label{tab:eval-metircs}
	\newcolumntype{C}{>{\centering\arraybackslash} m } 

	\begin{tabular}{|C{1.5cm} |C{2cm}|C{4.5cm}|C{6cm} |}
 
	\hline
	\hline
	نام معیار & نام لاتین & نحوه‌ی محاسبه & توضیح
		\\
	\hline
	\hline
	نرخ مثبت اشتباه &
	\lr{False Positive Rate (PF)}  &
	$ \displaystyle \frac{FP}{TN+FP} $ &
	نسبت تعداد داده‌هایی که به اشتباه خطادار پیش‌بینی شده‌اند به تعداد کل داده‌های بدون خطا
	\\
	\hline
	صحت & 
		\lr{Accuracy} & $ \displaystyle \frac{TP+TN}{TP+FP+TN+FN}$ &
	نسبت	تعداد پیش‌بینی‌های درست به تعداد کل پیش‌بینی‌ها
		
	\\
	\hline
	دقت &
	\lr{Precision} & $\displaystyle \frac{TP}{TP+FP}$ &
نسبت تعداد داده‌هایی که به درستی خطادار پیش‌بینی شده‌اند به تعداد کل داده‌هایی که خطادار پیش‌بینی شده‌اند
	\\
	\hline
	بازخوانی & 
	\lr{Recall (PD)} & $\displaystyle \frac{TP}{TP+FN}$ &
	نسبت تعداد داده‌هایی که به درستی خطادار پیش‌بینی شده‌اند به تعداد کل داده‌های خطادار
	\\
	\hline
	معیار اف &
	\lr{F-Measure} & $ \displaystyle \frac{2 \times Precision \times Recall}{Precision + Recall}$ &
	از آنجا که در بین معیارهای دقت و بازخوانی مصالحه وجود دارد معیار اف ترکیبی از آن دو را در نظر می‌گیرد
	\\
	\hline
	\end{tabular}
\end{table}

در ادامه به بررسی و تحلیل هر یک از این معیارها پرداخته می‌شود. 
\begin{itemize}
	\item 
	نرخ مثبت اشتباه: نام دیگر این معیار \واژه{احتمال اخطار اشتباه} می‌باشد. هر چقد که یک مدل پیش‌بینی به اشتباه داده‌ها را خطادار پیش‌بینی کند مقدار این معیار بیشتر می‌شود تا جایی که اگر مدل پیش‌بینی هیچ داده‌ای را بدون خطا پیش‌بینی نکند مقدار آن یک می‌شود و اگر داده‌ای را به اشتباه حاوی خطا معرفی نکند مقدار معیار صفر می‌شود. 
\item 
صحت: این معیار نسبت تعداد پیش‌بینی‌های مثبت واقعی و منفی واقعی را به تعداد کل پیش‌بینی‌ها می‌سنجد. با این حال صحت نمی‌تواند در مواردی که مجموعه داده‌های نا‌متوازن وجود دارد معیار مناسبی داشته باشد. به عنوان مثال اگر در یک مجموعه‌داده ۱۰ درصد از داده‌ها حاوی خا باشد آنگاه مدلی که همواره داده‌ها را بدون خطا پیش‌بینی می‌کند این معیار مقدار ۹۰ درصد می‌گیرد در صورتی که این مدل مناسب نیست. 
\item
دقت: نام دیگر این معیار \واژه{ارزش پیش‌بینی مثبت} می‌باشد. این معیار نشان دهنده‌ی آن است که به چه میزان داده‌های پیش‌بینی شده به عنوان خطادار درست پیش‌بینی شده است.  در صورتی که همه‌ی داده‌هایی که خطادار معرفی شده‌اند در واقعیت نیز حاوی خطا باشد این معیار مقدار یک می‌یابد. 
\item 
بازخوانی: این معیار مشخص می‌کند که چه مقدار از داده‌هایی که باید به عنوان خطادار معرفی می‌شدند در واقع توسط مدل خطادار پیش‌بینی شده‌اند.  زمانی که این معیار برابر یک می‌باشد بدان معنی است که تمام داده‌‌های حاوی خطا شناسایی شده‌اند. البته ممکن است برخی داده‌های بدون خطا نیز خطا دار پیش‌بینی شوند و همچنان معیار بازخوانی مقدار یک را داشته باشد. همانطور که در جدول \ref{tab:eval-metircs} مشخص شده ‌است بین دقت و بازخوانی یک \واژه{مصالحه} وجود دارد. این بدان معنی است که اغلب می‌توان یکی را به هزینه‌ی کاهش دیگری افزایش داد. 
\item
معیار اف: از آنجا که در محاسبه‌ی این معیار از ترکیب دقت و بازخوانی استفاده می‌شود از معایب بررسی جداگانه این دو معیار کاسته می‌شود. در برخی موارد اهمیت دقت و بازخوانی یکسان نیست که باید از نوع دیگری از معیار اف استفاده که دارای وزن‌دهی می‌باشد. 
\end{itemize}





دو اندازه دیگر نیز که در پژوهش‌ها کاربرد دارند عبارتند از \واژه{مساحت زیر منحنی} و \واژه{مساحت زیر منحنی هزینه-اثربخشی}. در محاسبه‌ی مساحت زیر منحنی از  نمودار \واژه{مشخصه‌ی عملکرد دریافت‌کننده} استفاده می‌شود . در این نمودار محورهای عمودی و افقی را به ترتیب بازخوانی و  نرخ مثبت کاذب تشکیل می‌دهد.  با تغییر \واژه{آستانه‌ی تصمیم} برای یک مدل می‌توان میزان بازخوانی و  نرخ مثبت کاذب را تغییر داده و بدین ترتیب منحنی را رسم نمود.  منظور از آستانه‌ی تصمیم  مرزی است  که یک مدل یک داده را حاوی خطا پیش‌بینی می‌کند یا سالم. به عنوان مثال زمانی که آستانه برابر ۳۰ درصد است در صورتی که یک داده به احتمال ۳۱ درصد حاوی خطا باشد آن داده به عنوان خطادار پیش‌بینی می‌شود. \\

یک مدل بی‌نقص دارای مساحت زیر نمودار 1 است.  مدل بی‌نقص مدلی است که تمام پیش‌بینی‌ها را به درستی انجام می‌دهد. این مدل  در برخورد با داده‌ی حاوی خطا ۱۰۰ درصد احتمال می‌دهد که حاوی خطا است  و برای داده‌ی سالم صفر درصد احتمال می‌دهد حاوی خطا است. اگر بخواهیم منحنی را برای مدل بی‌نقص رسم کنیم در ابتدا آستانه  را برابر یک  در نظر گرفته می‌شود. در نتیجه همه‌ی داده‌ها بدون خطا دسته‌بندی می‌شوند. در این حالت نرخ مثبت اشتباه برابر صفر است زیرا هیچ داده‌ای به اشتباه خطادار معرفی نشده. بازخوانی نیز صفر است چون هیچ داده‌ای به درستی خطادار پیش‌بینی نشده. پس منحنی از نقطه‌ی صفر و صفر آغاز می‌شود. زمانی که آستانه اندکی از یک کمتر شود مدل همه‌ی پیش‌بینی‌ها را به درستی انجام می‌دهد و نرخ مثبت اشتباه برابر صفر و بازخوانی برابر یک خواهد بود. در نتیجه نقطه‌ی دیگر  بر روی منحنی در بالا سمت چپ منحنی است. با کمتر کردن آستانه تغییری در کحل نقطه ایجاد نمی‌شود تا زمانی که آستانه به صفر برسد. در این حالت همه‌ی داده‌ها خطادار پیش‌بینی می‌شوند. نرخ مثبت اشتباه برابر یک خواهد شد چون هیچ داده‌ای سالم پیش‌بینی نشده است و بازخوانی برابر یک خواهد بود چون همه‌ی داده‌هایی که باید خطادار پیش‌بینی می‌شدند خطا‌دار پیش‌بینی شده‌اند. در نتیجه نقطه‌ی دیگر در بالا راست نمودار خواهد بود و مساحت زیر نمودار برابر یک خواهد بود. \\

برای یک مدل تصادفی  منحنی از مبدا به نقطه‌ی (1\lr{,}1) رسم خواهد شد. یک نمونه از منحنی مشخصه‌ی عملکرد دریافت‌کننده در شکل \ref{fig:ROC} آمده است. \\

\begin{figure}
	\centering
	\includegraphics[width=.60\textwidth]{img/ROC.PNG}
	\caption{ نمونه‌ای از نمودار \lr{ROC} \cite{menzies2007data}}
	\label{fig:ROC}
\end{figure}

 مساحت زیر منحنی هزینه-اثربخشی معیاری است که تعداد خطوطی از برنامه که  توسط تیم تضمین کیفیت و یا توسعه دهندگان نیاز است بررسی و آزمون شود را در نظر می‌گیرد. منظور از بررسی بازبینی کد جهت یافتن خطا بدون استفاده از روش‌های مرسوم آزمون نرم‌افزار می‌باشد. ایده‌ی  موثر بودن از نظر هزینه
برای مدل‌های ‌‌ پیش‌بینی خطا برای اولین بار توسط آریشلم و همکاران \cite{arisholm2007data} ارائه گردید. موثر بودن از نظر هزینه به این معنا است که چه تعداد خطا با بررسی و یا آزمون  \lr{n}  درصد اول خطوط می‌توان یافت. به عبارت دیگر اگر یک مدل پیش‌بینی خطا بتواند تعداد خطای بیشتری را با بررسی و تلاش در آزمون کمتر، نسبت به باقی مدل‌ها بیابد می‌توان گفت که تاثیر آن از نظر هزینه بیشتر است. دو منحنی در  قسمت راست شکل \ref{fig:AUCEC} برای دو مدل پیش‌بینی مختلف آمده است. هر دو مدل دارای سطح زیر نمودار یکسانی هستند اما زمانی که ۲۰ درصد اول محور افقی در نظر گرفته می‌شود مدل 
\lr{P$_2$}
  کارایی بهتری دارد. نمودار سمت چپ مدل‌های تصادفی، عملی\LTRfootnote {Practical} و بهینه را نشان می‌دهد.

\begin{figure}[H]
	\centering
	\includegraphics[width=.7\textwidth]{img/AUCEC.PNG}
	\begin{tabular}{c c c}
		\lr{O = optimal} & \lr{P = practical} &  \lr{R = random}\\
	 
	\end{tabular}
	\caption{ نمودار موثر بودن از نظر هزینه \cite{rahman2011bugcache}}
	\label{fig:AUCEC}
\end{figure}

معیارهایی که برای ارزیابی نتایج حاصل از روش رگرسیون به کار گرفته می‌شوند بر اساس همبستگی\LTRfootnote{Correlation} میان تعداد خطاهای پیش‌بینی شده و خطاهای واقعی محاسبه می‌شوند. نماینده‌ی این معیارها را می‌توان همبستگی اسپیرمن، پیرسون و $ R^2$ دانست \cite{nam2014survey}. 
\subsection{معیارهای پیش‌بینی خطا}
\label{subsec:metrics}
معیارهای پیش‌بینی خطا نقش مهمی را در ساخت مدل پیش‌بینی ایفا می‌کنند. اکثریت معیارهای پیش‌بینی خطا را می‌توان به دو دسته‌ی  کلی تقسیم کرد: معیارهای کد و معیارهای فرآیند. معیارهای کد می‌توانند به طور مستقیم از کدهای منبع موجود جمع آوری شوند در حالی که معیارهای فرآیند  از اطلاعات تاریخی که در مخازن نرم‌افزاری مختلف آرشیو شده‌اند استخراج می‌گردند. نمونه‌ای از این مخازن نرم‌افزاری سیستم‌های کنترل نسخه و سیستم‌های ردگیری خطا است. معیار‌های فرآیند از نظر هزینه موثرتر از سایر معیارها هستند\cite{arisholm2010systematic}. در برخی از مقالات نیز معیارهای  پیش‌بینی خطا به سه دسته‌ی: معیارهای کد منبع سنتی، معیارهای شئ‌گرایی و معیارهای فرآیند تقسیم شده‌اند\cite{radjenovic2013software}.\\\\
\textbf{معیارهای کد} \\

معیارهای کد تحت عنوان \واژه{معیارهای محصول} نیز شناخته می‌شوند و میزان پیچیدگی کد را می‌سنجند. \واژه{فرض زمینه‌ای} آنها این است که هرچقدر کد پیچیده‌تر باشد خطا‌خیز‌تر است. برای اندازه‌گیری پیچیدگی کد پژوهش‌گران معیار‌های مختلفی را ارائه داده‌اند که در ادامه مهم‌ترین آنها معرفی خواهند شد. 

این معیارها با استفاده از اندازه‌های مطرح شده در جدول \ref{tab:measure} محاسبه می‌شوند. 

\begin{table}[H] 
	\renewcommand*{\arraystretch}{1.5}	
	\centering \caption{اندازه‌های به کارگرفته شده  در معیارهای کد }
	\label{tab:measure}
	\newcolumntype{C}{>{\centering\arraybackslash} m } 
	\begin{tabular}{|C{2cm}|C{2cm}|C{1.5cm}|C{8cm}|}
		
		\hline
		\hline
		نام & نام لاتین & علامت اختصاری & توضیح \\
		\hline
		\hline
		تعداد خطوط کد & 		\lr{Line of Code}  & LOC		& این اندازه را می‌توان به اندازه‌های جزئی‌تر مانند تعداد خطوط توضیح، قابل اجرا، خالی از نوشته تقسیم کرد \\
		\hline
		تعداد عملگرها & \lr{Number of Operators} & 
		\lr{$N_1$}
		& تعداد عملگرهای موجود مانند + ، - ، \& \\
		\hline
		 
		تعداد عملوندها&  \lr{Number of Operands} & 
		\lr{$N_2$}
		& تعداد عملوندهای استفاده شده در کنار عملگرها\\
		\hline
			تعداد عملگرهای متمایز & \lr{Number of Unique Operators} & 
		\lr{$\eta_1$}
		& ---\\
		\hline
		
		تعداد عملوندهای متمایز &  \lr{Number of Unique Operands} & 
		\lr{$\eta_2$}
		& ---\\
		\hline
		
تعداد یال‌ها &   \lr{Number of Edges} &  E &  تعداد یال‌های گراف جریان کنترلی\\
		\hline
		تعداد گره‌ها &    \lr{Number of Nodes} & N & تعداد گره‌ها در گراف جریان کنترلی \\
		\hline
		
		تعداد قطعات متصل &    \lr{Number of Connected Component} & P  & تعداد قطعات متصل به هم در گراف جریان کنترلی
		\\
		\hline
		
	\end{tabular}
\end{table}


\begin{itemize}
	\item \textbf{معیار بزرگی: }
معیارهای \واژه{بزرگی} اندازه‌ی کلی و حجم کد را می‌سنجند. یکی از اندازه‌های برجسته که در محاسبه‌ی این معیارها و گاها خود به تنهایی به کار می‌رود "تعداد خطوط" می‌باشد. اولین بار \نام{آکیاما}{Akiyama}
 \cite{akiyama1971example}  
رابطه‌ی میان خطا و تعداد خطوط را مطرح کرد. \نام{هالستد}{Halstead}
 \cite{halstead1977elements} 
 چندین معیار بزرگی بر اساس  تعداد عملگرها و عملوند‌ها ارائه داده است و در مقاله‌ی \cite{pawade2016exploring} مورد بازنگری قرار گرفته است. معیارهایی که توسط هالستد مطرح شده‌اند در زیر آمده آمده‌اند که با استفاده از اندازه‌های جدول \ref{tab:measure} محاسبه می‌شوند. 
 \begin{latin}
 \baselineskip=1.1cm
Lenght: $N = N_1 + N_2$ \\
Volume: $V = N \times log_2 (\eta_1 + \eta_2)$\\
Difficulty: $D = \eta_1/2 \times N_2/\eta_2$ \\
Effort: $E = D \times V$ \\
Program Time: $T = E/18$ \\
 \end{latin}
 
 
\item \textbf{معیار پیچیدگی حلقوی: }
\نام{مک‌کیب}{McCabe} معیارهای \واژه{پیچیدگی حلقوی}
را پیشنهاد داد که این معیار با استفاده از تعداد گره‌ها، یالها و قطعات متصل در گراف  \واژه{جریان کنترلی} کد منبع محاسبه می‌گردد\cite{mccabe1976complexity}. این معیارها نشان می‌دهند که راه‌های کنترلی به چه میزان پیچیده هستند. باوجود اینکه جز اولین معیارها بوده است همچنان در پیش‌بینی خطا کاربرد دارد \cite{malhotra2014comparative}. این معیار  با استفاده از فرمول زیر محاسبه می‌شود. \\
\begin{latin}
$V(G) = E - N + 2P $
\end{latin}


\item \textbf{معیار مربوط به شئ‌گرایی: }
با ظهور زبان‌های شئ‌گرایی و محبوبیت آنها معیارهای کد  برای این زبان‌ها ارائه شد تا فرآیند توسعه بهبود یابد. نماینده‌ی این معیارها توسط  \نام{چدامبر و کمرر (CK)}{Chidamber and Kemerer  \lr{(CK)} }
 ارائه شده است\cite{chidamber1994metrics}. این معیارها  که در جدول  \ref{tab:ck-metrics} لیست آنها قرار داده شده، با توجه به خصیصه‌های زبان‌های شئ‌گرا مانند وراثت، \واژه{زوجیت}، \واژه{همبستگی} طراحی شده‌اند. بجز معیارهای  \چر{CK}، معیارهای شئ‌گرایی دیگری نیز بر اساس حجم و کمیت کد منبع پیشنهاد داده شده‌اند. مشابه معیارهای \موکد{اندازه}، معیارهای شئ‌گرایی تعداد نمونه‌های یک کلاس و توابع را می‌شمارند. \\
 \begin{table}[H] 
 	\renewcommand*{\arraystretch}{1.5}	
 	\centering \caption{معیارهای CK }
 	\label{tab:ck-metrics}
 	\newcolumntype{C}{>{\centering\arraybackslash} m } 
 	\begin{tabular}{|C{2cm}|C{3cm}|C{8cm}|}
 		
 		\hline
 		\hline
 		نام & توضیح & نحوه‌ی محاسبه \\
 		\hline
 		\hline
 		WMC &
 		تعداد توابع وزن‌دهی شده & وزن دهی بر اساس پیچیدگی هر تابع انجام می‌شود \\
 		\hline
DIT & 
عمق درخت وراثت & حداکثر طول مسیر از نوادگان یک کلاس تا خود کلاس\\
\hline
NOC &
تعداد فرزندان & تعداد نوادگان مستقیم کلاس\\
\hline
CBO & 
زوجیت میان اشیاء کلاس‌ها & تعداد کلاس‌هایی که کلاس مورد نظر با آن زوج شده است. دو کلاس با هم زوجیت دارند اگر یکی از توابع و یا متغیر‌های دیگری استفاده کرده باشد. \\
\hline
RFC &
پاسخ برای یک کلاس & تعداد توابعی که با فراخوانی یک تابع از کلاس احتمال فراخوانی دارند. برابر است با تعداد کل توابع کلاس و توابعی از سایر کلاس‌ها که در آنها فراخوانی می‌شوند. \\
\hline
LCOM & 
کمبود همبستگی میان توابع & تعداد جفت توابعی که متغیر مشترک ندارند منهای جفت توابعی که متغیر مشترک دارند. \\
 		
 		\hline

\end{tabular}
\end{table}

\end{itemize}
\textbf{معیارهای فرآیند} \\

در ادامه  تعدادی از معیارهای فرآیند بررسی می‌شوند که در این دسته شاخص محسوب می‌شوند. در جدول \ref{tab:process-measure} اندازه‌هایی که در محاسبه‌ی معیارهای فرآیند مثال زده شده به کار می‌رود آمده است. 
\begin{table}[H] 
	\renewcommand*{\arraystretch}{1.5}	
	\centering \caption{اندازه‌های به کارگرفته شده  در معیارهای فرآیند }
	\label{tab:process-measure}
	\newcolumntype{C}{>{\centering\arraybackslash} m } 
	\begin{tabular}{|C{2cm}|C{2cm}|C{2.25cm}|C{7.25cm}|}
		
		\hline
		\hline
		نام & نام لاتین & علامت اختصاری & توضیح \\
		\hline
		\hline
		تعداد خطوط تبدیلی &  Churned LOC & --- & تعداد خطوط اضافه شده به علاوه‌ی خطوط تغییر داده شده در دو نسخه‌ی متفاوت از برنامه \\
		\hline
		تعداد فایل‌های تبدیلی &
		\lr{Files Churned} & --- &
		تعداد فایل‌های تغییر یافته در یک قطعه \\
		\hline
		تعداد فایل‌ها &
		\lr{Files Count} & --- &
		تعداد فایل‌های موجود در یک قطعه\\
		\hline
		تجدیدنظر‌ها &
		\lr{Revisions} & --- &
		تعداد تجدید نظرهایی (اصلاح‌ها) که در فایل انجام شده است\\
		\hline 
		بازآرایی & 
		Refactoring & --- & 
		تعداد دفعاتی که یک فایل بازآرایی شده است. در واقع تعداد ثبت‌هایی شمرده می‌شود که در توضیح آنها کلمه‌ی refactor وجود داشته باشد\\
		\hline
		تعداد ایمیل‌ها &
		\lr{Number of Mails} & POP\_NOM &
		 تعداد ایمیل‌هایی که در آنها نام کلاس مورد نظر آورده شده است\\
		 \hline
		 تعداد نخ‌ها &
		 \lr{Number of Threads} & POP\_NOT &
		 تعداد نخ‌هایی که درباره‌ی یک کلاس صحبت می‌کنند \\
		 \hline
		 تعداد نویسندگان &
		 \lr{ Number of Authors}& POP\_NOA &
		 تعداد نویسندگانی که درباره‌ی کلاس مورد نظر صحبت می‌کنند\\
		 \hline
	\end{tabular}
\end{table}

\begin{itemize}
\item \textbf{تغییر تبدیلی نسبی کد: }
\نام{ناگاپان و بال}{Nagappan and Ball} هشت معیار تغییر \واژه[تبدیلی]{تبدیل} نسبی کد را ارائه داده‌اند\cite{nagappan2005use}. دو مثال از این معیارها در زیر آمده است.  در معیار $M_1 $ تعداد تجمعی خطوط اضافه و حذف شده بین دو نسخه از برنامه را می‌شمارد و بر تعداد خطوط برنامه تقسیم می‌کند. معیار دیگر تعداد فایل‌های تغییر یافته از یک قطعه برنامه را بر تعداد  کل فایل‌ها تقسیم می‌کند. 
\begin{latin}
\baselineskip=1.1cm
$M_1 =\mathlarger{\frac{ Churned LOC }{ Total LOC}}$\\
$M_2 =\mathlarger{\frac{ Files Churned}{ Files Count}}$
\end{latin}

\item \textbf{معیارهای تغییر: }
این معیارها  گستره‌ی تغییرات در تاریخچه‌ی ذخیره شده در سامانه‌ی کنترل نسخه را اندازه می‌گیرند. به عنوان مثال تعداد رفع خطاها، تعداد \واژه{بازآرایی کد} و یا تعداد نویسندگان یک فایل را می‌شمارند. \نام{موزر}{Moser} و همکاران 18 معیار تغییر را از مخازن \نام{اکلیپس}{Eclipse} استخراج کردند و یک تحلیل مقایسه‌ای میان معیارهای کد و معیارهای تغییر انجام دادند. آنها به این نتیجه رسیدند که معیارهای تغییر پیش‌بینی کننده‌ی بهتری از معیارهای کد هستند.  به عنوان نمونه دو مورد از ۱۸ معیار مطرح شده برابر اندازه‌های \موکد{تجدیدنظر‌ها} و \موکد{بازآرایی} است.

\item \textbf{معیارهای شهرت: }
 \نام{بکچلی}{Bacchelli} و همکاران معیارهای \واژه{شهرت} را بر اساس تحلیل ایمیل‌های آرشیو شده‌ی نویسندگان ارائه داده‌اند. ایده‌ی اصلی این معیارها این است که یک قطعه‌ی  نرم‌افزاری که در ایمیل‌ها درباره‌ی آن بیشتر صحبت شده است خطاخیزتر می‌باشد\cite{bacchelli2010popular}.  بکچلی پنج معیار شهرت معرفی کرده است.  به عنوان نمونه سه مورد از آنها  برابر است با اندازه‌های \موکد{تعداد ایمیل‌ها، تعداد نخ‌ها و تعداد نویسندگان}. \\
\end{itemize}


\نام{راجنویک}{Radjenovic} و همکاران در پژوهش خود به \واژه{بررسی قاعده‌مند} معیارهای پیش‌بینی خطا در مطالعات پیشین پرداخته‌اند.  طبق این پژوهش در 49\lr{\%} مطالعات از معیارهای شئ‌گرایی، در 27\lr{\%} معیارهای سنتی کد و در 26 \lr{\%} از معیارهای فرآیند استفاده شده است. با توجه به مطالعات بررسی شده دقت پیش‌بینی خطا  با انتخاب معیارهای مختلف، تفاوت قابل توجهی  پیدا می‌کند. معیارهای شئ‌گرایی و فرآیند موفق‌تر از معیارهای سنتی هستند. معیارهای سنتی  پیچیدگی کد، قویا با معیارهای اندازه مانند تعداد خطوط کد همبستگی دارند و این دو توانایی پیش‌بینی خطا دارند اما جز بهترین معیارها نیستند. معیارهای شئ‌گرایی بهتر از اندازه و پیچیدگی عمل می‌کنند و با این که با معیارهای اندازه همبستگی دارند اما ویژگی‌های بیشتری علاوه بر اندازه را دارند. معیارهای ایستای کد همانند اندازه، پیچیدگی و شئ‌گرایی به منظور بررسی یک نسخه از برنامه مفید هستند اما با هر \واژه{تکرار} در فرآیند توسعه‌ی نرم‌افزار دقت پیش‌بینی آنها کاسته می‌شوند و معیارهای فرآیند در چنین شرایطی بهتر عمل می‌کنند.  با این وجود  که  معیارهای فرآیند‌  دارای توانمندی بالقوه‌ای  هستند، اما در تعداد کمتری از پژوهش‌ها مورد استفاده قرار گرفته‌اند\cite{radjenovic2013software}. \\
 
\نام{آسترند}{Ostrand} و همکاران به بررسی این موضوع پرداخته‌اند که آیا اطلاعاتی درباره‌ی اینکه کدام توسعه‌دهنده یک فایل را اصلاح می‌کند قادر است که پیش‌بینی خطا را بهبود بخشد. در پژوهش قبلی آنها\cite{weyuker2008too} مشخص شده بود که تعداد کلی   افراد توسعه‌دهنده در یک فایل می‌تواند در پیش‌بینی خطا تاثیر متوسطی داشته باشد. در  مقاله‌ی \cite{ostrand2010programmer}  تعدادی از متغیرهای کد منبع و فرآیند به همراه معیار مرتبط به توسعه‌دهنده در نظر گرفته شده است.  در این پژوهش مشخص شد  که تعداد خطاهایی که یک توسعه‌دهنده تولید می‌کند ثابت است و با سایر توسعه دهندگان فرق دارد. این تفاوت با  حجم کدی که یک توسعه‌دهنده اصلاح می‌کند مرتبط است و در نتیجه در نظر گرفتن یک نویسنده خاص نمی‌تواند به بهبود پیش‌بینی خطا کمک کند\cite{ostrand2010programmer}. \\

\نام{رحمان و دوانبو}{Rahman and Devanbu} از جنبه‌های مختلف معیارهای فرآیند  را با سایر معیارها مقایسه کرده‌اند\cite{rahman2013and}. نتایج نشان می‌دهد  زمانی که مدل پیش‌بینی بر روی یک نسخه آموزش می‌بیند و در نسخه‌ی بعدی آزمون می‌شود معیارهای کد، مساحت زیر منحنی  قابل قبولی دارند اما مساحت آنها کمتر از معیارهای فرآیند است  و از نظر معیار مساحت زیر نمودار هزینه-اثربخشی ۲۰ درصد 
بهتر از یک مدل تصادفی عمل نمی‌کنند و  به آن معنی است که این معیارها از نظر هزینه چندان  موثر نیستند. همچنین معیارهای کد ایستاتر هستند، ‌یعنی با تغییرات پروژه و تغییر در توزیع خطاها همچنان معیارها بدون تغییر باقی می‌مانند. معیار ایستا تمایل دارد یک فایل را در \واژه[انتشارهای]{انتشار} متوالی همچنان حاوی خطا معرفی کند. معیارهای ایستا به مدل‌های راکد منجر می‌شوند که این مدل‌ها به سمت فایل‌های بزرگ با تراکم خطای کمتر \واژه{جهت‌گیری} دارند. به عنوان مثال حالتی را در نظر بگیرید که در یک پروژه فایل‌های بزرگ و پیچیده‌ای وجود دارد که پس از چندین انتشار خطاهای آنها برطرف می‌شود اما مدل‌هایی که بر اساس معیارهای کد ساخته شده‌اند همچنان این فایل‌ها را به عنوان خطا‌خیز معرفی می‌کنند. از طرف دیگر حالتی را در نظر بگیرید که یک فایل با اندازه و پیچیدگی کم به تازگی به وجود آمده و یا تغییرات فراوان یافته است. مدل‌های مبتنی بر کد به این فایل‌ها توجه چندانی نخواهند کرد در حالیکه که این فایل‌ها مستعد وجود خطا هستند. بدین ترتیب معیارهای فرآیند بهتر از معیارهای کد عمل می‌کنند. \\
معیارهای و اندازه‌های استفاده شده در این مقاله در جدول \ref{tab:process-measure} و \ref{tab:process-metircs} آورده شده‌اند. در ادامه هر یک از معیارها به طور مشروح توضیح داده می‌شوند. 
 
 \begin{table}[H] 
 	\renewcommand*{\arraystretch}{1.5}	
 	\centering \caption{اندازه‌های فرآیند 
 		\cite{rahman2013and}}
 	\label{tab:process-measures}
 	\newcolumntype{C}{>{\centering\arraybackslash} m } 
 	\newcounter{magicrownumbers}
 	\def\rownumber{}
 	\setcounter{magicrownumbers}{0}
 	\begin{tabular}{|@{\makebox[3em][c]{\rownumber\space}} |c|c|}
 		
 		\hline
 		\hline
 		نام اندازه  & توضیح
 		\gdef\rownumber{\stepcounter{magicrownumbers}\arabic{magicrownumbers}} 
 		\\
 		
 		\hline
 		\hline
 		\lr{COMM } & تعداد ثبت در سیستم کنترل نسخه
 		\\
 		\hline
 		\lr{ADEV} & تعداد توسعه‌دهندگان فعال
 		\\ 
 		\hline
 		\lr{DDEV} & تعداد توسعه‌دهندگان متمایز
 		\\ 
		\hline
 		\lr{MINOR} & تعداد مشارکت‌کنندگان جزئی
 		\\ 
 		\hline
		\lr{OEXP} & تجربه‌ی مالک پرونده
 		\\ 
 		\hline
 	\end{tabular}
 \end{table}

 \begin{table}[H] 
	\renewcommand*{\arraystretch}{1.5}	
	\centering \caption{معیارهای فرآیند 
		\cite{rahman2013and}}
	\label{tab:process-metircs}
	\newcolumntype{C}{>{\centering\arraybackslash} m } 

	\def\rownumber{}
	\setcounter{magicrownumbers}{0}
	\begin{tabular}{|@{\makebox[3em][c]{\rownumber\space}} |c|c|}
		
		\hline
		\hline
		نام معیار  & توضیح
		\gdef\rownumber{\stepcounter{magicrownumbers}\arabic{magicrownumbers}} 
		\\
		
		\hline
		\hline
		\lr{ADD} &  مقدار نرمال‌سازی شده‌ی تعداد خطوط اضافه شده
		\\ 
		\hline
		\lr{DEL}  & مقدار نرمال‌سازی شده‌ی تعداد خطوط حذف شده
		\\ 
		\hline
		\lr{OWN} &  درصد خطوطی که مالک پرونده مشارکت کرده
		\\ 
		\hline
		\lr{NCOMM} & تعداد ثبت‌های همسایگان
		\\ 
		\hline
		\lr{NADEV} & تعداد توسعه‌دهندگان فعال همسایگان
		\\ 
		\hline
		\lr{NDDEV} & تعداد توسعه‌دهندگان متمایز همسایگان
		\\ 
		\hline
	
		\lr{AEXP} & تجربه‌ی تمام مشارکت‌کنندگان
		\\ 
		\hline
		
	\end{tabular}
\end{table}




\begin{enumerate}
	\item
	\textbf{تعداد ثبت در سیستم کنترل نسخه:}
	تعداد ثبت‌هایی که در  پرونده‌ی ‌مورد نظر در طول انتشار قبلی تاکنون تغییر کرده است. برای محاسبه ی آن لازم است که تمام ثبت‌های پروژه بین ثبت کنونی و \واژه{انتشار} قبلی بررسی شود و ثبت‌هایی که در آن این پرونده تغییر کرده‌اند شمرده شوند.
	\item
	\textbf{تعداد توسعه‌دهندگان 
		فعال:}
	تعداد توسعه‌دهندگانی که در طول انتشار قبلی تاکنون (زمان ثبت) پرونده را تغییر داده‌اند. لازم است ثبت‌های موجود در باز‌ه‌ی زمانی خواسته شده بررسی شود و آنها که پرونده مورد نظر را تغییر داده‌اند انتخاب شوند. نام کسانی که ثبت را انجام داده‌اند بازیابی شود و تعداد نام‌های متمایز شمرده شود. 
	\item
	\textbf{تعداد توسعه‌دهندگان	متمایز:}
	مشابه معیار قبلی با این تفاوت که در طول انتشار محاسبه نمی‌شود. بلکه از ابتدای پروژه تا زمان ثبت در نظر گرفته می‌شود. 
	\item
	\textbf{تعداد مشارکت‌کنندگان جزئی:}
	مشارکت‌کننده‌ی جزئی کسی است که کمتر از ۵٪ خطوط موجود در پرونده به او تعلق داشته باشد. بدین منظور نویسنده‌ی هر خط مشخص می‌شود. تعداد خطوط هر نویسنده شمرده می‌شود و بر تعداد خطوط پرونده تقسیم می‌شود. سپس تعداد نویسندگانی که کمتر از ۵٪ مشارکت داشته‌اند شمرده می‌شود. 
\item
\textbf{تجربه‌ی مالک پرونده:}
	ابتدا لازم است که نحوه ی محاسبه تجربه را تعریف کنیم. هر چقدر یک فرد تعداد تغییرات بیشتری را در یک پروژه انجام دهد تجربه بیشتری را در آن پروژه دارد و ثبت را می‌توان به ایجاد تغییر تعبیر کرد. برای محاسبه‌ی معیار ابتدا مالک پرونده مشخص می شود. سپس تعداد ثبت‌هایی که مالک پرونده از ابتدای پروژه تا زمان مورد نظر انجام داده، شمرده می شود.
\end{enumerate}

 \begin{table}[H] 
	\renewcommand*{\arraystretch}{1.5}	
	\centering \caption{اندازه‌های فرآیند به کار رفته در ساخت معیارهای فرآیند 
	}
	\label{tab:process-measures2}
	
	\begin{tabular}{|c|c|}
		
		\hline
		\hline
		نام اندازه  & توضیح
		\\
		
		\hline
		\hline
		AddedLines &
		تعداد خطوط اضافه شده به پرونده در طول انتشار
		\\
		\hline
		AddedLinesInProject & 
		تعداد خطوط اضافه شده به پروژه در طول انتشار
		\\ 
		\hline
			DeletedLines &
		تعداد خطوط حذف شده از پرونده در طول انتشار
		\\
		\hline
		DeletedLinesInProject & 
		تعداد خطوط حذف شده از پروژه در طول انتشار
		\\ 
		\hline
		OwnerParticipation & تعداد خطوطی که به مالک پرونده تعلق دارد
		\\
		\hline
		LOC & تعداد خطوط پرونده
		\\
		\hline
	Neighbors & تعداد همسایگان یک پرونده
	\\ \hline
	$NCommit_i$ &
	اندازه‌ی COMM  برای همسایه‌ی i ام
	\\ \hline
	$NActiveDev_i$ & 
اندازه‌ی ADEV برای همسایه‌ی i ام 
	\\ \hline
	$NDistinctDev_i$ &
	اندازه‌ی DDEV برای همسایه‌ی  i ام
	\\ \hline
	$FON_i$ & 
	تعداد دفعات همسایگی  همسایه‌ی i ام\\
	\hline
Part & 
تعداد مشارکت کنند‌گان در یک پرونده 
\\ \hline
	$PCommit_i$ &
	 تعداد ثبت‌‌‌های مشارکت کننده‌ی i ام
	 \\ \hline
	
	
	
		
	\end{tabular}
\end{table}

\begin{enumerate}

	\item
	\textbf{مقدار نرمال‌سازی شده‌ی تعداد خطوط اضافه شده:}
	این معیار تعداد خطوط اضافه شده در یک پرونده را در طول انتشار قبلی می‌شمارد. سپس جهت نرمال سازی آنرا بر تعداد کل خطوط اضافه شده در پروژه در طول انتشار قبلی تقسیم می‌کند. برای بدست آوردن تعداد خطوط اضافه شده در یک پرونده هر ثبت نسبت به ثبت قبلی مقایسه می‌شود و تعداد خطوط اضافه شده جمع زده می‌شود.
\begin{equation} \label{eq:added_line}
ADD = \frac{AddedLines}{AddedLinesInProject}
\end{equation}

	\item
	\textbf{مقدار نرمال‌سازی شده‌ی تعداد خطوط حذف شده:}
\begin{equation} \label{eq:deleted_line}
	DEL = \frac{DeletedLines}{DeletedLinesInProject}
\end{equation}

	\item
	\textbf{درصد خطوطی که مالک پرونده مشارکت کرده:}
	درصد خطوطی  از پرونده، در  ثبت مورد نظر  که به مالک پرونده تعلق دارد. مالک پرونده کسی است که در آن لحظه از زمان بیشترین تعداد خطوط موجود در پرونده به او تعلق دارد. ابتدا نویسنده‌ی هر خط مشخص می‌شود سپس برای هر نویسنده تعداد خطوطی که به وی تعلق دارد شمرده می‌شود. تعداد خطوط مالک پرونده بر تعداد خطوط پرونده تقسیم می‌گردد.
\begin{equation} \label{eq:own}
OWN = \frac{OwnerPatricipation}{LOC} \times 100
\end{equation}

	\item
	\textbf{تعداد ثبت‌های همسایگان:}
	میانگین وزن دهی شده تعداد ثبت‌های همسایگان پرونده از انتشار قبلی تا کنون را اندازه‌گیری می‌کند. همسایگان یک پرونده در یک ثبت، پرونده‌هایی هستند که در آن نسخه از برنامه تغییر کرده‌اند. در‌واقع در هر ثبت از برنامه تعدادی پرونده نسبت به ثبت قبلی تغییر کرده‌اند که این پرونده‌ها همسایه‌ی یکدیگر محسوب می شوند. نحوه‌ی وزن دهی نیز به این صورت است که هرچقدر یک پرونده تعداد دفعات بیشتری را در طول انتشار با پرونده مورد نظر همسایه شده باشد وزن بیشتری می‌یابد. برای محاسبه ابتدا همسایگان پرونده در ثبت  و تعداد دفعاتی که  در طول انتشار همسایه شده‌اند مشخص می‌شوند. سپس برای هر پرونده‌ی همسایه، معیار تعداد ثبت در سیستم کنترل نسخه محاسبه می‌شود. هر معیار در تعداد دفعاتی همسایگی ضرب می‌شود و با هم جمع زده می‌شوند. در انتها بر تعداد کل دفعات همسایگی همسایگان تقسیم می‌شود. 
\begin{equation} \label{eq:ncomm}
NCOMM = \frac{\mathlarger{\sum}\limits_{i=1}^{Neighbors} FON_i \times NCommit_i}{\mathlarger{\sum}\limits_{i=1}^{Neighbors}FON_i}
\end{equation}
	\item
	\textbf{تعداد توسعه‌دهندگان فعال همسایگان:}
	مشابه معیار قبلی عمل می‌شود با این تفاوت که معیار توسعه‌دهندگان فعال در نظر گرفته خواهد شد.
\begin{equation}\label{eq:nadev}
	NADEV = \frac{\mathlarger{\sum}\limits_{i=1}^{Neighbors} FON_i \times NActiveDev_i}{\mathlarger{\sum}\limits_{i=1}^{Neighbors}FON_i}
\end{equation}
	\item
	\textbf{تعداد توسعه‌دهندگان متمایز همسایگان:}
	مشابه معیار قبلی عمل می‌شود با این تفاوت که معیار توسعه‌دهندگان متمایز در نظر گرفته خواهد شد.

\begin{equation} \label{eq:nddev}
NDDEV = \frac{\mathlarger{\sum}\limits_{i=1}^{Neighbors} FON_i \times NDistinctDev_i}{\mathlarger{\sum}\limits_{i=1}^{Neighbors}FON_i}
\end{equation}

	\item
	\textbf{تجربه‌ی تمام مشارکت‌کنندگان:}
	تمام مشارکت‌کنندگان در پرونده تا زمان ثبت مورد نظر یافت می‌شوند. برای هر یک مشابه  اندازه‌ی شماره 5، تجربه محاسبه می‌شود و از مقدار تجربه‌ها میانگین هندسی گرفته می‌شود. 
\begin{equation}\label{eq:aexp}
AEXP = \sqrt[Part]{ \prod\limits_{i=1}^{Part}PCommit_i}
\end{equation}	
	
\end{enumerate}
 

\input{literature_review/survey-models}
\input{literature_review/survey-mutation}
\input{literature_review/survey-conclusion}




\chapter{معیارهای جهش و فرآیند}
 
\label{sec:method}
با  مطالعات مروری انجام شده نقاطی از این حوزه که نیازمند پژوهش بیشتر هستند تا بتوان به وسیله‌ی آن به ارائه‌ی روشی کاراتر در پیش‌بینی خطا پرداخت مشخص شد. مقاله‌ی \cite{bowes2016mutation} اولین مقاله‌ای است که  یک  روش پیش‌بینی خطا با استفاده از تحلیل جهش ارائه نموده  است و این موضوع نیازمند تحقیق بیشتر است. از طرف دیگر بر طبق مقاله‌ی \cite{radjenovic2013software} استفاده از معیارهای فرآیند علی‌رغم توانایی بالقوه‌ای که در پیش‌بینی خطا دارند، در پژوهش‌های کمتری مورد بررسی قرار گرفته‌اند. یکی از دلایل آن می‌تواند نو ظهور بودن این معیارها نسبت به سایرین باشد. معیارهای فرآیند از جنبه‌های مختلف نیز از سایر معیار‌ها برتری دارند \cite{rahman2013and}. \\
این پژوهش قصد دارد سه رویکرد  پیشنهادی را به منظور بهبود پیش‌بینی خطا بررسی کند.  این رویکردها عبارتند از:
\begin{enumerate}
\item
در این رویکرد معیارهای جهش و معیارهای فرآیند در کنار یکدیگر استفاده می‌شوند و به وسیله‌ی آنها پیش‌بینی انجام می گیرد. این دو دسته معیار در پژوهش‌های گذشته مطرح شده‌اند اما تاکنون در کنار یکدیگر قرار نگرفته‌اند.
\item
معیارهای جدیدی مطرح می‌شوند که مبتنی بر مفاهیم آزمون جهش و فرآیند توسعه‌ی نرم‌افزار است.
\item
معیارهای جدیدی مطرح می‌شوند که با کمک مفاهیم جهش سعی در بهبود معیارهای فرآیند دارند.
\end{enumerate}

\section{معیارهای جهش و فرآیند}
\label{sec:method-phase1}

این رویکرد با توجه به مقاله‌ی \cite{bowes2016mutation} مطرح شده که در آن بررسی به کارگیری معیارهای جهش و فرآیند را در پژوهش‌های آتی توصیه می‌کند.  همچنین  معیار جهش یک معیار  مرتبط با کد است. مقاله‌ی \cite{rahman2013and}  بیان می‌کند که معیارهای کد ایستا هستند و تمایل دارند که یک موجودیت را در انتشارهای متوالی حاوی خطا معرفی کنند. حال شرایطی را در نظر بگیرید که که امتیاز جهش در یک موجودیت کم باشد و دلیل آن کافی نبودن مجموعه آزمون باشد چراکه توسعه‌دهندگان از درست بودن کد اطمینان دارند یا اینکه پس از انتشارهای متوالی خطاها بر طرف شده است. چنین موجودیتی حاوی خطا نیست اما با توجه به معیار جهش خطا‌خیز است. با در نظر گرفتن معیارهای فرآیند در مورد این موجودیت که نشان می‌دهند پایدار و بدون تغییر است از میزان خطا‌خیز بودن آن کاسته می‌شود و انتظار می‌رود کارایی مدل پیش‌بینی بهبود یابد. 
برای پاسخ به این پرسش مجموعه معیارهای جهش  از پژوهش \cite{bowes2016mutation}  و معیارهای فرآیند از پژوهش \cite{rahman2013and} انتخاب می‌شوند. در جداول  \ref{tab:process-metircs} و \ref{tab:mutation-metircs} معیارهای مورد نظر آورده شده است و در ادامه معرفی شده و  \\
\begin{table}[H] 
	\renewcommand*{\arraystretch}{1}	
	\centering \caption{معیارهای فرآیند 
		\cite{rahman2013and}}
	\label{tab:process-metircs}
	\newcolumntype{C}{>{\centering\arraybackslash} m } 
	\newcounter{magicrownumbers}
	\def\rownumber{}
	\setcounter{magicrownumbers}{0}
	\begin{tabular}{|@{\makebox[3em][c]{\rownumber\space}} |c|c|}
		
		\hline
		\hline
		نام معیار  & توضیح
		\gdef\rownumber{\stepcounter{magicrownumbers}\arabic{magicrownumbers}} 
		\\
		
		\hline
		\hline
	\lr{COMM } & تعداد ثبت در سیستم کنترل نسخه
		\\
		\hline
		\lr{ADEV} & تعداد توسعه‌دهندگان فعال
		\\ 
		\hline
		\lr{DDEV} & تعداد توسعه‌دهندگان متمایز
		\\ 
		\hline
		\lr{ADD} &  مقدار نرمال‌سازی شده‌ی تعداد خطوط اضافه شده
		\\ 
		\hline
		\lr{DEL}  & مقدار نرمال‌سازی شده‌ی تعداد خطوط حذف شده
		\\ 
		\hline
		\lr{OWN} &  درصد خطوطی که مالک فایل مشارکت کرده
		\\ 
		\hline
		\lr{MINOR} & تعداد مشارکت‌کنندگان جزئی
		\\ 
		\hline
		\lr{NCOMM} & تعداد ثبت‌های همسایگان
		\\ 
		\hline
		\lr{NADEV} & تعداد توسعه‌دهندگان فعال همسایگان
		\\ 
		\hline
		\lr{NDDEV} & تعداد توسعه‌دهندگان متمایز همسایگان
		\\ 
		\hline
		\lr{OEXP} & تجربه‌ی مالک فایل
		\\ 
		\hline
		\lr{AEXP} & تجربه‌ی تمام مشارکت‌کنندگان
		\\ 
		\hline
		
	\end{tabular}
\end{table}

\begin{table}[H] 
	\renewcommand*{\arraystretch}{1}	
	\centering \caption{معیارهای جهش 
		\cite{bowes2016mutation}}
	\label{tab:mutation-metircs}
	\newcolumntype{C}{>{\centering\arraybackslash} m } 
	\def\rownumber{}
	\setcounter{magicrownumbers}{0}
	\begin{tabular}{|@{\makebox[3em][c]{\rownumber\space}} |c|c|}
		
		\hline
		\hline
		نام معیار  & توضیح
		\gdef\rownumber{\stepcounter{magicrownumbers}\arabic{magicrownumbers}} 
		\\
		\hline
		\hline
		\lr{MuNOM } &   تعداد جهش‌یافته‌های تولید شده
		\\
		\hline
		\lr{MuNOC} &   تعداد جهش‌یافته‌های پوشش‌داده شده توسط آزمون‌ها
		\\
		\hline
		\lr{MuNMS} &   امتیاز جهش‌یافته‌های تولید شده
		\\
		\hline
		\lr{MuNMSC} &   امتیاز جهش‌یافته‌های پوشش‌داده شده توسط آزمون‌ها
		\\
		\hline
		
	\end{tabular}
\end{table}
از آنجا که  در این پژوهش پیش‌بینی‌ها در سطح فایل انجام می‌شود، معیارها برای هر فایل جداگانه محاسبه می‌شوند. در ادامه هر یک از معیارهای فرآیند معرفی و نحوه‌ی محاسبه‌ی آن‌ها بیان می‌شود. معیارهای جهش به طور مستقیم توسط ابزارهای موجود محاسبه‌ می‌گردد.\\

\begin{enumerate}
\item
\textbf{تعداد ثبت در سیستم کنترل نسخه:}
 تعداد ثبت‌هایی که در آن فایل مورد نظر در طول انتشار قبلی تاکنون تغییر کرده است. برای محاسبه ی آن لازم است که تمام ثبت‌های پروژه بین ثبت کنونی و \واژه{انتشار} قبلی بررسی شود و ثبت‌هایی که در آن این فایل تغییر کرده‌اند شمرده شوند.
 \item
\textbf{تعداد توسعه‌دهندگان 
	فعال:}
 تعداد توسعه‌دهندگانی که در طول انتشار قبلی تا کنون (زمان ثبت) فایل را تغییر داده‌اند. لازم است ثبت‌های موجود در باز‌ه‌ی زمانی خواسته شده بررسی شود و آنها که فایل مورد نظر را تغییر داده‌اند انتخاب شوند. نام کسانی که ثبت را انجام داده‌اند بازیابی شود و تعداد نامهای متمایز شمرده شود. 
 \item
 \textbf{تعداد توسعه‌دهندگان	متمایز:}
 مشابه معیار قبلی با این تفاوت که در طول انتشار محاسبه نمی‌شود. بلکه از ابتدای پروژه تا زمان ثبت در نظر گرفته می‌شود. 
\item
\textbf{مقدار نرمال‌سازی شده‌ی تعداد خطوط اضافه شده:}
این معیار تعداد خطوط اضافه شده در یک فایل را در طول انتشار قبلی می‌شمارد. سپس جهت نرمال سازی آنرا بر تعداد کل خطوط اضافه شده در پروژه در طول انتشار قبلی تقسیم می‌کند. برای بدست آوردن تعداد خطوط اضافه شده در یک فایل هر ثبت تسبت به ثبت قبلی مقایسه می‌شود و تعداد خطوط اضافه شده جمع زده می‌شود.
\item
\textbf{مقدار نرمال‌سازی شده‌ی تعداد خطوط حذف شده:}
مشابه معیار قبلی می‌باشد. 
\item
\textbf{تعداد خطوطی که مالک فایل مشارکت کرده:}
 درصد خطوطی  از فایل، در  ثبت مورد نظر  که به مالک فایل تعلق دارد. مالک فایل کسی است که در آن لحظه از زمان بیشترین تعداد خطوط موجود در فایل به او تعلق دارد. ابتدا نویسنده‌ی هر خط مشخص می‌شود سپس برای هر تویسنده تعداد خطوطی که به وی تعلق دارد شمرده می‌شود. تعداد خطوط مالک فایل بر تعداد خطوط فایل تقسیم می‌گردد.
\item
\textbf{تعداد مشارکت‌کنندگان جزئی:}
توسعه‌دهنده‌ی جزئی کسی است که کمتر از ۵٪ خطوط موجود در فایل به او تعلق داشته باشد. بدین منظور نویسنده‌ی هر خط مشخص می‌شود. تعداد خطوط هر نویسنده شمرده می‌شود و بر تعداد خطوط فایل تقسیم می‌شود. سپس تعداد نویسندگانی که کمتر از ۵٪ مشارکت داشته‌اند شمرده می‌شود. 
\item
\textbf{تعداد ثبت‌های همسایگان}
 میانگین وزن دهی شده تعداد ثبت‌های همسایگان فایل از انتشار قبلی تا کنون را اندازه‌گیری می‌کند. همسایگان یک فایل در یک ثبت، فایل‌هایی هستند که در آن نسخه از برنامه تغییر کرده‌اند. در‌واقع در هر ثبت از برنامه تعدادی فایل نسبت به ثبت قبلی تغییر کرده‌اند که این فایل‌ها همسایه‌ی یکدیگر محسوب می شوند. نحوه‌ی وزن دهی نیز به این صورت است که هرچقدر یک فایل تعداد دفعات بیشتری را در طول انتشار با فایل مورد نظر همسایه شده باشد وزن بیشتری می‌یابد. برای محاسبه ابتدا همسایگان فایل در ثبت  و تعداد دفعاتی که  در طول انتشار همسایه شده‌اند مشخص می‌شوند. سپس برای هر فایل همسایه، معیار تعداد ثبت در سیستم کنترل نسخه محاسبه می‌شود. هر معیار در تعداد دفعاتی همسایگی ضرب می‌شود و با هم جمع زده می‌شوند. در انتها بر تعداد کل دفعات همسایگی همسایگان تقسیم می‌شود. 
\item
\textbf{تعداد توسعه‌دهندگان فعال همسایگان:}
مشابه معیار قبلی عمل می‌شود با این تفاوت که معیار توسعه‌دهندگان فعال در نظر گرفته خواهد شد.

\item
\textbf{تعداد توسعه‌دهندگان متمایز همسایگان:}
مشابه معیار قبلی عمل می‌شود با این تفاوت که معیار توسعه‌دهندگان متمایز در نظر گرفته خواهد شد.
\item
\textbf{تجربه‌ی مالک فایل:}
 ابتدا لازم است که نحوه ی محاسبه تجربه را تعریف کنیم. هر چقدر یک فرد تعداد تغییرات بیشتری را در یک پروژه انجام دهد تجربه بیشتری را در آن پروژه دارد و ثبت را می‌توان به ایجاد تغییر تعبیر کرد. برای محاسبه‌ی معیار ابتدا مالک فایل مشخص می شود. سپس تعداد ثبت‌هایی که مالک فایل از ابتدای پروژه تا زمان مورد نظر انجام داده، شمرده می شود.
\item
\textbf{تجربه‌ی تمام مشارکت‌کنندگان:}
تمام مشارکت‌کنندگان در فایل تا زمان ثبت مورد نظر یافت می‌شوند. برای هر یک مشابه معیار قبلی تجربه، محاسبه می‌شود و از مقدار تجربه‌ها میانگین هندسی گرفته می‌شود. 

\end{enumerate}


%%%%%%%%%%%
\section{معیارهای جهش مبتنی بر فرآیند}
در رویکرد دوم، چهار معیار جدید در این پژوهش معرفی می‌شوند که با استفاده از مفاهیم آزمون جهش و تاریخچه‌ی توسعه‌ی نرم‌افزار ساخته می‌شوند. از این رو این معیارها  \نام{معیارهای جهش مبتنی بر فرآیند}{Process Based Mutation Metrics (PBMM)} نامیده شده‌اند. 

\begin{enumerate}
	\item  
	\textbf{
		تعداد جهش‌یافته‌های تولید شده‌ی جدید نسبت به انتشار قبلی برنامه: }همانطور که در مقاله‌ی \cite{just2014mutants} مطرح شده جهش‌یافته‌ها جایگزین خوبی برای خطاهای واقعی می‌باشند. زمانی که تعداد جهش‌یافته‌های جدید زیاد باشد یعنی تغییراتی که خطا‌خیز‌تر هستند بیشتر است. به منظور محاسبه‌ی این معیار لازم است خطوط اضافه شده به فایل مورد نظر در ثبت کنونی، نسبت به انتشار قبلی مشخص شود و سپس تعداد جهش یافته‌هایی که این خطوط تولید می‌کنند شمرده شوند. 
	\item 
	\textbf{
		تعداد جهش‌یافته‌های متمایز در چند انتشار اخیر:} این معیار نشان می‌دهد موجودیت مورد بررسی به چه میزان سابقه‌ی تغییراتی را دارد که احتمال بروز خطا را افزایش می‌دهد. تعداد انتشارها باید به گونه‌ای باشد که کم یا زیاد نباشد. زیرا تعداد انتشارهای کم سبب می‌شود تفاوت جندانی با معیار قبلی نداشت باشد و سابقه‌ی تغییرات به اندازه‌ی کافی مد نظر قرار نگیرد. از طرف دیگر در نظر گرفتن تعداد زیادی انتشار هم هزینه‌بر است و هم به دلیل تغییرات زیاد  فایل در طول توسعه‌ی نرم‌افزار اطلاعات اولیه مفید نخواهد بود. عدد در نظر گرفته شده برای تعداد انتشارها چهار می‌باشد. نحوه‌ی محاسبه به این شکل است که برای هر انتشار تعداد جهش‌یافته‌ها در انتشار جدید، نسبت به قبلی  شمرده می‌شود و با یکدیگر جمع زده  می‌شوند. 
	
	\item 
	\textbf{
		میزان تغییرات مثبت امتیاز جهش  در چند انتشار اخیر:}
	تغییرات امتیاز جهش نشان از تغییرات در برنامه و آزمون‌های نرم‌افزار است. این معیار نشان می‌دهد این تغییرات به چه میزان در جهت بهبود کیفیت نرم‌افزار بوده. چراکه امتیاز بالاتر جهش نشان از کیفیت بهتر آزمون‌ها و در نتیجه نرم‌افزار است.  به منظور محاسبه‌ی این معیار در هر انتشار امتیاز جهش محاسبه می‌شود و در صورتی که نسبت به انتشار قبلی تغییر مثبت  بود به مجموع تغییرات افزوده می‌شود. 
	\item 
	\textbf{
		میزان تغییرات منفی امتیاز جهش در چند انتشار اخیر:}
	این معیار مشابه معیار سوم عمل می‌کند با این تفاوت که میزان تغییرات در خلاف جهت بهبود نرم‌افزار را می‌سنجد. 	
	
\end{enumerate}


\section{معیارهای ترکیبی جهش-فرآیند}
رویکرد سوم با توجه به مطالب گفته شده در مقاله‌ی \cite{rahman2013and} مطرح شده که بیان می‌کند معیارها هر چقدر هم که پویا باشند (دچار رکود نشوند، مانند معیارهای فرآیند) زمانی در پیش‌بینی خطا مفید هستند که همراه با ایجاد خطا باشند. نکته‌ی قابل توجه این است که همه‌ی تغییرات در یک فایل به یک اندازه موجب بر پیچیدگی فایل نمی‌افزایند و به عبارت دیگر موجب بروز خطا نمی‌شوند. به عنوان مثال در یک فایل به زبان جاوا ممکن است \واژه{توضیح} و یا \واژه{مستند‌جاوا} وجود داشته باشد که بروزرسانی یا اضافه و کم شدن آنها تاثیری بر روند اجرای برنامه و میران پیچیدگی ندارند با این حال در محاسبه‌ی معیارهای پیش‌بینی خطا در نظر گرفته می‌شوند. هدف از ارائه‌ی معیارهای \نام{ترکیبی جهش-فرآیند}{Process-Mutation Hybrid} بهبود کاستی‌های معیارهای فرآیند در چنین شرایطی است. در اینجا دو معیار \موکد{مقدار نرمال شده‌ی خطوط اضافه شده} و یا \موکد{کم شده} است.  این دو معیار جز شاخص‌ترین معیارهای فرآیند هستند.\\
در نگاه اول  این ایده به ذهن می‌رسد که با توجه به تعداد جهش‌یافته‌هایی که  اضافه  و یا حذف هر خط ایجاد می‌کند، اضافه یا کم شدن خطوط وزن دهی شود و به منظور اجرای آن از دو  فرمول زیر بهره گرفت.\\
\begin{latin}
	
	$M_1 =\ number\ of\ lines\ added\ \times \ number\ of\ muatants\ derived$\\
	
	$M_2 =\ number\ of\ lines\ deleted\ \times \ number\ of\ mutants\ derived$\\
\end{latin}


با وجود مناسب بودن ایده ی اولیه با بررسی‌های بیشتر دو مشکل در معیارهای فوق مشخص می‌شود.\\
مشکل اول : هدف از ارئه ی این معیارها وزن دهی به خطوط اضافه و کم شده است. نکته قابل توجه این است که هر خط باید به صورت جداگانه وزن دهی شود و وزن یک خط بر وزن خط دیگر تأثیری نداشته باشد. مثال زیر را در نظر بگیرید.
\begin{latin}
\flushleft
//this method is important  \emph{→ 0 mutant} \\
// this method get root of \emph{→ 0 mutant}\\
// sum of a plus b \emph{→ 0 mutant} \\ 
b = sqrt(a+b) \emph{→ 2 mutant} \\
\end{latin}

فرض کنید ۴ خط بالا به یک فایل اضافه شده است. معیار  \موکد{مقدار نرمال شده‌ي خطوط اضافه شده} قبل ار نرمال سازی عدد چهار را نمایش می‌دهد در حالی که از این چهار خط ۳ خط توضیح است. حال معیار اولیه پیشنهادی برابر ۸ خواهد بود که بدیهی است، از هدف ارايه ی متریک فاصله گرفته است. حال اگر تنها جهش یافته‌های تولید شده در خطوط اضافه شده را در نظر بگیریم این مقدار می‌تواند جایگزین مناسبی باشد. در‌ واقع نگاشتی را ارائه می‌شود که هر خط از برنامه را به یک عدد نگاشت می‌دهد. این عدد میزان پیچیدگی آن خط و یا احتمال بروز خطا را تعریف می‌کند.  لازم به یادآوری است که در مقاله‌ی  \cite{just2014mutants} اشاره شده که جهش یافته ها جایگزین خوبی برای خطاهای واقعی هستند. این نگاشت برابر است با تعداد جهش یافته های تولید شده در آن خط.

مشکل دوم: این معیار برای عمل‌کرد هرچه بهتر مشابه معیار  \موکد{مقدار نرمال شده‌ي خطوط اضافه شده}‌ نیاز به نرمال‌سازی دارد. به جهت نرمال‌سازی نمی‌توان از همان روش استفاده کنیم چراکه در آن وزن دهی به خطوط وجود ندارد و از آن مهم‌تر توضیحات را نیز در نظر می‌گیرد. از طرف دیگر این امکان وجود ندارد که برای تمام خطوط اضافه یا کم شده در کل پروژه در طول یک انتشار جهش یافته تولید شود (به دلیل زمانبر بودن و پیچیدگی‌های فراوان در پیاده سازی). در مقالات گذشته اشاره شده که تعداد ثبت‌ها می‌تواند نشانگر میزان تغییرات باشد. بنابرین از تعداد ثبت‌های کل پروژه در طول یک انتشار به منظور نرمال‌سازی استفاده خواهد شد.

در نهایت نحوه‌ی محاسبه به این صورت خواهید بود که ابتدا ثبت‌هایی از برنامه در طول آخرین انتشار که در آن فایل مورد نظر تغییر کرده است بازیابی می شود. سپس برای هر ثبت تعداد جهش یافته‌های جدید نسبت به ثبت قبلی محاسبه می‌شود و برای محاسبه‌ی جهش‌یافته‌های حذف شده تعداد جهش یافته‌ها در  ثبت قبلی را یافته و آن‌ها که جز خطوط حذف شده در ثبت بعدی است شمرده می شود. تعداد جهش‌یافته‌های اضافه و حذف شده در ثبت‌ها جمع شده و بر تعداد ثبت‌های کل پروژه در طول انتشار تقسیم می گردد.





\chapter{مورد مطالعاتی}
\label{chap:case-study}
در این فصل مطالعه‌ی موردی بر روی مجموعه‌داده‌ی \lr{defects4j} \cite{Just:2014:DDE:2610384.2628055} انجام می‌گیرد. ابتدا نحوه‌ی کلی برپایی آزمایش شرح داده می‌شود و سپس چگونگی استخراج معیارها و پیاده‌سازی ارمایش توضیح داده خواهد شد. 
\section{طراحی آزمایش}
به منظور ارزیابی رویکردهای گفته شده لازم است که برای مجموعه معیارهای هر رویکرد مدلهای پیش‌بینی ساخته شود و هر عملکرد هر مدل نسبت به پژوهش‌های گذشته مقایسه شود. به این ترتیب ابتدا لازم است از مجموعه‌داده‌ی فراهم شده معیارهای بیان شده در فصل \ref{sec:method} استخراج شوند. مجموعه‌داده‌ی \lr{defects4j} که در قسمت‌های آتی معرفی می‌شود شامل اطلاعات خطا در چندین فایل است و به همین تعداد، فایل بدون خطا در ثبت و پروژه‌ی متناظر به طور تصادفی انتخاب می‌گردد. برای فایلهای حاوی خطا و سالم، معیارها استخراج می‌شود. معیارهای استخراج شده  برای هر فایل به عنوان بردار ویژگی در مدلهای دسته‌بندی عمل می‌کند. مدلهای دسته‌بندی به منظور پیش‌بینی حاوی خطا بودن ساخته می‌شود و عملکرد آنها مقایسه می‌گردد. مدلهایی که با هم مقایسه می‌شود در الگوریتم و \واژه{پیکربندی} یکسان هستند و تنها تفاوت آنها در معیارهای استفاده شده به منظور یادگیری است. بدین ترتیب تاثیر معیارها بر پیش‌بینی خطا سنجیده می‌شود. 

\section{ آشنایی با ابزارها و مجموعه داده}
این قسمت به معرفی ابزارهای استفاده شده در این پژوهش می‌پردازد. آشنایی با این ابزارها به درک هرچه بهتر  نحوه‌ی استخراج معیارها  و روند آزمایش کمک می‌کند.

\subsection{مجموعه داده \lr{defect4j}}
 مجموعه‌‌داده‌ی انتخابی به منظور انجام مورد مطالعاتی لازم است که دارای ویژگی‌های زیر باشد:
 \begin{itemize}
 	\item
 	اطلاعات خطاهای پروژه وجود داشته باشد و این اطلاعات نشان دهد که خطا متعلق به کدام پرونده در کدام ثبت است. 
 	\item
 	پروژه‌ها متن-باز باشد تا بتوان با استفاده از \واژه{کد منبع} آنها معیارها را استخراج نمود.
 	\item
 	برای پروژه‌ها موارد آزمون مناسب وجود داشته باشد تا بتوان معیارهای جهش را استخراج کرد.
 \end{itemize}
 در میان مجموعه‌داده‌های موجود  مانند 
 Promise \cite{boetticher2007promise} و BDP
 \cite{d2012evaluating}
  مجموعه‌داده‌ی \lr{defects4j} تنها موردی است که تمام ویژگی‌ها را دارد.\\
 
این مجموعه شامل  شش پروژه می‌باشد که این پروژه‌ها  \واژه{متن-باز} هستند و با استفاده از نرم‌افزارهای کنترل نسخه‌ی گیت و svn می‌توان به کدهای آن‌ها در طول فرآیند توسعه‌ی آنها دسترسی پیدا کرد. بجز پروژه‌ی Chart سایرین از سیستم گیت استفاده می‌کنند. همچنین این پروژه به دلیل نداشتن ساختار مناسب کنار گذاشته شد و از پرونده‌های حاوی خطای  موجود در آن استفاده نشد. 
مجموعه‌داده‌ی \چر{defects4j} به صورت یک \واژه{چارچوب} ارائه شده است که کارهایی بیش از نگهداری اطلاعات درباره‌ی پروژه‌ها انجام می‌دهد. مهم‌ترین  کارهایی که می‌توان به وسیله‌ی این ابزار انجام داد در جدول \ref{tab:defects4j-ops} آمده است. 


\begin{table}[H] 
	\renewcommand*{\arraystretch}{1.3}	
	\centering \caption{عملیات موجود در \چر{defects4j}  }
	\label{tab:defects4j-ops}
	\newcolumntype{C}{>{\centering\arraybackslash} m } 
	\begin{tabular}{ |c|c|}
		
		\hline
		\hline
		نام عمل  & توضیح
		\\
		\hline
		\hline
		\lr{info } &   نمایش پیکربندی یک پروژه‌ی خاص یا خلاصه‌ی یک خطای خاص
		\\
		\hline
		\lr{checkout} &   وارسی یک نسخه‌ی حاوی خطا یا تعمیر شده از پروژه
		\\
		\hline
		\lr{compile} &   کامپایل کدها و آزمون‌های نوشته شده توسط توسعه‌دهندگان
		\\
		\hline
		\lr{test} &   اجرای یک آزمون یا مجموعه‌ی آزمون در یک نسخه‌ی حاوی خطا یا تعمیر شده از پروژه
		\\
		\hline
		\lr{mutation} &   اجرای تحلیل جهش در یک نسخه‌ی حاوی خطا یا تعمیر شده از پروژه
		\\
		\hline
		
	\end{tabular}
\end{table}

این ابزار در اجرای عملیات بالا دارای محدودیت است و تنها آن‌ها را بر روی ثبت‌های از پیش تعیین شده انجام می‌دهد. ثبت‌های از پیش تعیین شده شامل ثبت‌های حاوی خطا و تعمیر آن خطا می‌باشد. در جدول \ref{tab:defects4j-bugs} اطلاعات مربوط به تعداد خطاهای هر پروژه آمده است. 

\begin{table}[H] 
	\renewcommand*{\arraystretch}{1.3}	
	\centering \caption{پروژه‌های موجود در \چر{defects4j}  }
	\label{tab:defects4j-bugs}
	\newcolumntype{C}{>{\centering\arraybackslash} m } 
	\begin{tabular}{ |c|c|c|}
		
		\hline
		\hline
	نام مختصر &	نام کامل  & تعداد خطا
		\\
		\hline
		\hline
		\lr{Chart } & JFreeChart &	26
		\\
		\hline
		\lr{Closure} & \lr{Closure compiler}	& 133
		\\
		\hline
		\lr{Lang} &   \lr{Apache commons-lang} &	65
		\\
		\hline
		\lr{Math} &  \lr{Apache commons-math} &	106
		\\
		\hline
		\lr{Mockito} &   Mockito &	38
		\\
		\hline
		Time & Joda-Time &	27
		\\
		\hline
	-	& کل  پروژه‌ها &   395
		\\
		\hline
		
	\end{tabular}
\end{table}


به منظور نصب و راه اندازی ابزار  \lr{defects4j} ابتدا از صفحه‌ی \نام{گیت‌هاب}{Github} آن  کدهای مربوطه دریافت می‌شود. سپس باید  دستوراتی را اجرا کرد تا سایر متعلقات دریافت شود. این تعلقات شامل مخزن نرم‌افزاری مربوط به شش پروژه‌ی یاد شده است که کدهای پروژه‌ها در آن قرار دارد. نکته‌ی قابل توجه در این ابزار این است  که بجز  دستور info سایر دستورات عملیات مربوط را بر روی کامپیوتر کاربر انجام می‌دهد و خروجی را نمایش  داده می‌شود، نه اینکه از یک پایگاه داده اطلاعات را صرفاً بازخوانی کند. \\
در نیازمندی‌های این ابزار اشاره شده که باید از جاوا نسخه‌ی ۷ استفاده شود. اما مسأله‌ای که به آن اشاره نشده توزیع‌کننده‌ی جاوا است. جاوا دو توزیع‌کننده‌ی عمده دارد. یکی OpenJDK و دیگری Oracle می‌باشد. با استفاده از OpenJDK ابزار \lr{defects4j} و ابزارهایی که به آن وابسته است به خوبی کار نمی‌کنند. به عنوان مثال برخی مجموعه آزمون‌ها که باید بدون خطا اجرا شوند به دلیل نبود  \واژه[وابستگی‌های]{وابستگی} لازم با شکست مواجه می شوند. در این پایانامه از توزیع‌کننده‌ی Oracle استفاده شده است.\\ 
راه ارتباط با  ابزار defects4j \واژه{خط دستور}‌ می‌باشد و  یک نمونه‌  از دستورات قابل استفاده در این ابزار  در شکل  \ref{fig:d4j-info-command} است که این دستور اطلاعات مربوط به پروژه‌ی Lang و خطای شماره‌ی یک را خواهد داد. 
\begin{latin}
	\flushleft
defects4j info -p Lang -b 1
\end{latin}

\begin{figure}[H]
	\centering
	\includegraphics[width=.8\textwidth]{img/case_study/d4j-info-commadn.png}
	\caption{اجرای دستور info در \lr{defects4j}}
	\label{fig:d4j-info-command}
\end{figure}

\subsection{ابزار Major}
\label{sec:tools-major}

این ابزار جهت تولید جهش‌یافته و تحلیل جهش استفاده می‌شود. یک ابزار دیگر در این حوزه 
\نام{PIT}{ \url{http://pitest.org/}} می‌باشد اما به دلیل سازگاری ابزار \lr{defects4j} با Major و نیز قابلیت‌های ویژه‌ی آن از این ابزار استفاده شد.
چند مورد از ویژگی‌های مهم ابزار Major عبارتند از :
\begin{itemize}

\item
 راحتی استفاده  به دلیل نیاز به دستورات کمتر نسبت به PIT
\item
امکان اجرای تحلیل جهش در پروژه‌هایی که از ابزار \واژه{ساخت} گریدل استفاده می‌کنند
\item
 مجموعه عملگرهای کاملتر
\item
انعطاف در پیکربندی : امکان انجام تحلیل تنها برای یک کلاس یا تابع، تنظیمات ساده و کامل جهت مشخص کردن مجموعه عملگرها
\end{itemize}
لازم به ذکر است که این ابزار از کامپایلر‌ مخصوص به خود جهت کامپایل برنامه و ساخت جهش‌یافته استفاده می‌کند که گسترش یافته‌ی یک کامپایلر جاوا است. 
استفاده از این ابزار را می‌توان در سه مرحله خلاصه کرد :
\begin{enumerate}
\item
\textbf{پیکربندی تولید جهش‌یافته به وسیله‌ی دستورات \lr{MML}: }
این ابزار برای مشخص نمودن اینکه از چه عملگرهایی استفاده شود و آن‌ها در چه محل‌هایی از برنامه به کار گرفته شوند یک زبان ساده ابداع کرده است به نام MML که یک کامپایلر نیز دارد. ابتدا کد MML نوشته می‌شود و سپس با MMLC کامپایل می‌شود و نتیجه به عنوان یکی از پارامترها  در هنگام فراخوانی به ابزار  Major ارسال می شود.

نمونه‌ای از این کد در شکل \ref{fig:major-mml} آمده است. در بلاک targetOp مجموعه‌ی عملگرها مشخص می‌شود و در خط آخر محلی که عملگرهای جهش مشخص شده عمل می‌کند. در واقع تنها برای محل مشخص شده جهش‌یافته تولید می‌شود. در بلاک targetOP ابتدا مشخص می‌شود که عملگرهای باینری موجود در برنامه به چه عملگرهای دیگری تبدیل شود. همچنین امکان مشخص کردن نحوه‌ی تبدیل عملگرهای یگانی با دستو UNI به جای BIN وجود دارد. سپس عباراتی که امکان حذف آنها وجود دارد مشخص می‌شود که در شکل امکان حدف خروجی توابع مشخص شده است. در قسمت آخر مجموعه عملگرهای جهش قابل استفاده مشخص شده که در شکل عملگرهای حذف عبارت، جایگزینی عملگر شرطی و جایگزینی عملگر رابطه‌ای به کار گرفته شده است. مجموعه‌ی عملگرهای موجود در این ابزار در \cite{just2014major} آمده است. در خط آخر بیان شده که این عملگرها به نحو مشخص شده در targetOp بر روی تابعی به نام classify که سه آرگومان از نوع int دارد و در کلاس Triangle و پکیج trinangle قرار دارد اعمال شوند.

\begin{figure}[H]
	\centering
	\includegraphics[width=.8\textwidth]{img/case_study/major-mml.png}
	\caption{نمونه کد MML در Major}
	\label{fig:major-mml}
\end{figure}
\item 	
\textbf{تولید جهش‌یافته‌ها:}
 همانطور که اشاره شد ابزار Major جهت تولید نسخ جهش‌یافته نیاز به کامپایل پروژه دارد. امروزه پروژه‌های نرم‌افزاری از جمله پروژه‌های موجود در \lr{defects4j} از ابزارهایی استفاده می‌کنند که فرآیند ساخت را خودکارسازی می‌کنند. فرآیند ساخت به طور کلی شامل مراحل زیر است:
\begin{itemize}
	\item پاک سازی  پوشه‌های کاری، از پرونده‌های ساخت‌های قبلی
	\item 
	معرفی‌ وابستگی‌ها و کامپال پروژه
	\item
	معرفی وابستگی‌ها و کامپایل موارد آزمون
	\item
	اجرای موارد آزمون و ارائه‌ی گزارش
\end{itemize}
سه نوع از مهم ترین ابزارهای خودکارسازی مورد استفاده در صنعت عبارتند از Ant ،  Maven و \lr{Gradle}. در پروژه‌های مورد مطالعه نیز این سه نوع به کار گرفته شده است.  
هر یک از روش‌ها خودکارسازی دارای دستورات مربوط به خود می‌باشد  و برای تولید جهش‌یافته باید متناسب با آنها عمل نمود که در زیر خلاصه شده است. 
\begin{itemize}
	\item
	Ant : 
	این دسته از پروژه‌ها دارای یک پرونده به نام build.xml است که دستورات لازم جهت پیکربندی و انجام عملیات ساخت در آن قرار دارد. به منظور تولید جهش‌یافته کافیست کامپایلر مورد استفاده در قسمت کامپایل پروژه را کامپایلر توسعه‌یافته‌ی Major قرار داد و پارامترهای لازم به آن ارسال شود. 
	\item Maven : 
	در این دسته از پروژه‌ها دستورات لازم در پرونده‌ی pom.xml قرار دارد.  به وسیله‌ی  یک \واژه{افزونه} در ابزار Maven این پرونده تبدیل به یک پرونده build.xml می‌شود که قابل استفاده توسط ابزار Ant است. پس از این تبدیل مشابه حالت قبل عمل می‌شود. 	
	\item Gradle : 
	در این دسته از پروژه‌ها دستورات لازم در پرونده‌ی  build.gradle قرار دارد.  به منظور تولید جهش‌یافته کافیست کامپایلر مورد استفاده در قسمت کامپایل پروژه را کامپایلر توسعه‌یافته‌ی Major قرار داد و پارامترهای لازم به آن ارسال شود. 
\end{itemize}

نمونه‌ای از حاصل اجرای  عملیات جهش برای یک پرونده در شکل \ref{fig:major-mutant}  آمده است   که نشان می‌دهد ۸۶ جهش یافته تولید شده است.  همچنین ابزار یک پرونده به نام mutants.log تولید می‌کند که نشان می‌دهد چه جهش‌یافته‌هایی در کجا تولید شده‌اند. نمونه‌ای از محتویات این پرونده در شکل \ref{fig:major-log}  آمده است. 

\begin{figure}[H]
	\centering
	\includegraphics[width=.8\textwidth]{img/case_study/major-mutant.png}
	\caption{اجرای عملیات جهش برای یک پرونده}
	\label{fig:major-mutant}
\end{figure}

\begin{figure}[H]
	\centering
	\includegraphics[width=\textwidth]{img/case_study/major-log.png}
	\caption{نمونه‌ای از پرونده‌ی mutants.log}
	\label{fig:major-log}
\end{figure}

\item
\textbf{ اجرای تحلیل جهش:}
 ابتدا پرونده‌های آزمون کامپایل می‌شود و سپس هر مجموعه آزمون بر روی جهش‌یافته‌هایی که تاکنون کشته نشده‌اند اجرا می‌شود. در پایان نتایج  در خروجی چاپ می‌شوند. همچنین نتایج  در پرونده‌های با پسوند csv قرار می‌گیرد. نمونه‌ای از اجرای تحلیل جهش در شکل \ref{fig:major-analysis} و پرونده‌ی نتایج خروجی در شکل \ref{fig:major-results} نشان داده شده.
 
 \begin{figure}[H]
 	\centering
 	\includegraphics[width=.8\textwidth]{img/case_study/major-analysis.png}
 	\caption{اجرای تحلیل جهش}
 	\label{fig:major-analysis}
 \end{figure}
 
 \begin{figure}[H]
 	\centering
 	\includegraphics[width=.8\textwidth]{img/case_study/major-results.png}
 	\caption{نتایج خروجی تحلیل جهش}
 	\label{fig:major-results}
 \end{figure}

\end{enumerate}


\subsection{کتابخانه‌ی Jgit}
این کتابخانه جهت کار با مخازن نرم‌افزاری  که از نوع گیت هستند به کار گرفته می‌شود و به زبان جاوا است. تمام عملیات مهم و اساسی که در نرم‌افزار اصلی گیت وجود دارد در این کتابخانه نیز قابل انجام است. مشکلی که کار با این کتابخانه دارد نبود منابع آموزشی به اندازه‌ی کافی است. چراکه کاربران زیادی ندارد و آموزش‌های ابتدایی معمولاً نیازهای عموم کاربران را بر طرف می‌کند. 
\subsection{چارچوب Hibernate }
به وسیله‌ی این چارچوب می‌توان  اشیاء موجود در برنامه‌ی جاوا را به داده‌های موجود در پایگاه داده تبدیل کرد. اصطلاحاً به این ابزار ها \نام{ORM}{Object Relational Mapping} می‌گویند. مزیت استفاده از این نوع از ابزارها این است که ارتباط با پایگاه داده ساده‌تر خواهد شد و حجم کدهای لازم کاهش چشم‌گیری خوهد داشت. همچنین این ابزارها اشیاء را در حافظه‌ی موقت مدیریت می‌کنند و حجم کاری پایگاه داده کاسته می‌شود. 
\section{نکات پیاده‌سازی پروژه}

پیاده‌سازی پروژه در زبان جاوا انجام گرفت. یکی از نکات مهم و قابل توجه در پیاده‌سازی این پروژه این است که تمام مراحل انجام کار  به طور کاملاً خودکار انجام شود و در هیچ مرحله‌ای نیاز به دخالت عامل خارجی  ندارد بجز پیکربندی اولیه مانند آدرس پایگاه داده. همچنین در تمام مراحل سعی شده است که تمام اصول لازم در طراحی معماری نرم‌افزار به کار گرفته شود و نیازمندی‌های کیفی پروژه نیز مد نظر قرار گیرد. این نیازمندی‌ها شامل موارد زیر است:
\begin{enumerate}
\item 
\واژه{کارایی} : جهت پاسخ به این نیازمندی از پایگاه داده استفاده شده است.
\item
قابلیت نگهداری: این قابلیت از سایرین بیشتر حائز اهمیت است. زیرا پروژه های پروژهشی خود به صورت مستقیم کاربران عمومی ندارند و از این جهت نیازمند کارایی بالا یا رابط گرافیکی کاربر پسند نیستند. استفاده آن‌ها معمولاً در گسترش آن‌ها توسط سایر محققین است که راه را ادامه خواهند داد. 
\begin{itemize}
\item

 برای پاسخ به این نیازمندی اصول مربوط به کدنویسی  در فصل سوم و چهارم کتاب  \cite{martin2009clean} به کار گرفته شده است.
 \item
 از الگوهای نرم افزاری پر کاربرد مانند \نام{اداپتور}{Adaptor}،  \نام{فکتوری}{Factory} و \نام{سینگلتون}{Singelton} استفاده شده است.
 \item
 به منظور جلوگیری از قطعه کد تکراری از وراثت و توابع \واژه{عمومی} استفاده شده است. همینطور عمق وراثت از عدد ۳ بیشتر نشده است زیرا وراثت عمیق از خوانایی کد می‌کاهد و محل اشتباه خواهد بود. 
\end{itemize}
\item
امنیت: از آنجا که پروژه قرار نیست به استفاده ی عموم برسد و کاربران عمل متخاصمانه‌ای انجام نخواهند داد   به نوع خاصی از امنیت  نسبت به انواع متداول دارد.  باید روند توسعه‌ی پروژه دارای امنیت باشد. از این نظر که کدها مفقود نشوند یا در صورت اشتباه در توسعه بتوان پروژه را به حالت قبل بازگرداند. در این راستا کدهای پروژه در مخزن نرم افزاری از نوع گیت نگهداری شده که یک مخزن در کامپیوتر شخصی و دیگری در سایت \نام{بیت‌باکت}{Bitbucket - \url{https://bitbucket.org/alimohebbi/bug_predict }} 
قرار دارد. مزیت این سایت نسبت به گیت‌هاب این است مخازن خصوصی  را به صورت رایگان ارائه می دهد. در مخازن خصوصی  اجازه‌ی دسترسی تنها به  افراد تعیین شده از طرف مالک  داده می‌شود و عموم کاربران به آن دسترسی ندارند. از ابتدای شروع پیاده‌سازی کدها در مخازن بروزرسانی شده است. نمایی از ثبت‌های مختلف پروژه در مخزن در شکل \ref{fig:bitbucket}
آورده شده است. 
\end{enumerate}

 \begin{figure}[H]
	\centering
	\includegraphics[width=1\textwidth]{img/case_study/bitbucket.png}
	\caption{نمایی از مخزن نرم‌افزاری}
	\label{fig:bitbucket}
\end{figure}


\section{رویکرد اول : معیارهای فرآیند در کنار جهش}
در این قسمت  چگونگی استخراج معیارهای رویکرد اول شرح داده می‌شود. ابتدا بازم است اطلاعات مربوط به ثبت‌های  حاوی خطا از ابزار \lr{defects4j} بازیابی شود و سپس این اطلاعات با استفاده از مخرن نرم‌افزاری تکمیل شود. در مراحل بعد ابتدا معیار‌های فرآیند و سپس معیارهای جهش استخراج خواهند شد. 

\subsection{ استخراج اطلاعات مربوط به ثبت‌های   حاوی خطا}
اطلاعاتی که درباره‌ی ثبت‌های حاوی خطا قابل بازیابی است در زیر آمده است:
\begin{enumerate}
\item شناسه‌ی ثبت در مخزن 
\item نام فایل حاوی خطا
\item شماره ی خطا در ابزار \lr{defects4j}
\item شماره‌ی ثبت تعمیر خطا
\item نام پروژه
\item نام انتشار قبلی پروژه
\item شماره‌ی ثبت انتشار قبلی پروژه
\end{enumerate}

از میان اطلاعات بالا همگی به سادگی با استفاده از ابزار defect4j قابل استخراج است بجز دو مورد آخر. همچنین شماره‌ی ثبت تعمیر مورد استفاده قرار نگرفت ولی نگهداری شد چراکه ممکن بود لازم شود. 
برای بدست آوردن اطلاعات مربوط به هر انتشار لازم است که مخرن نرم افزاری هر پروژه مورد بررسی قرار گیرد. در  مخازن پروژه‌های نرم‌افزاری  از نوع گیت برای مشخص کردن یک رویداد مهم از \نام{تگ}{Tag} استفاده می‌شود. هر تگ می‌تواند به یک ثبت از برنامه اشاره کند. تگ می‌تواند نمایانگر رویدادهایی چون انتشار برنامه، انتشار بتا، و یا کاندید انتشار باشد. بنابرین با استفاده از تگ می‌تواند انتشار را پیدا کرد.\\

تگ‌های مخازن گیت دو نوع \واژه{سبک‌وزن} و \واژه{حاشیه‌نویسی شده} که در میان پروژه‌های مورد مطالعه از هر دو نوع جهت مشخص کردن انتشار استفاده شده است.  کار کردن با این دو نوع تگ دارای تفاوتهایی در پیاده‌سازی است که در ایتجا از پرداختن به جزییات صرف نظر می‌شود. \\ ابتدا همه‌ی تگ‌های موجود در مخازن نرم‌افزاری استخراج می‌شود و در پایگاه داده قرار می‌گیرد. از میان تگ‌های استخراج شده تگ‌های نا مرتبط با انتشار از پایگاه داده حذف می‌شود. تگ‌های نامرتبط با توجه به نام آنها مشخص می‌شود به عنوان مثال تگ‌هایی که حاوی لغات Beta یا Dev هستند نامرتبط  محسوب می‌شوند. در نهایت جدولی به نام ReleaseProject ساخته می‌شود که در آن اطلاعات انتشارهای مختلف وجود دارد. نمایی از این جدول در شکل \ref{fig:project-release} آمده است. \\
\begin{figure}[H]
	\centering
	\includegraphics[width=1\textwidth]{img/case_study/project-release.png}
	\caption{نمایی از جدول محتوای انتشارها}
	\label{fig:project-release}
\end{figure}


در قدم بعدی باید مشخص شود اولین انتشار ما قبل هر ثبت حاوی خطا کدام است. برای این منظور لیست ثبت‌ها  در یک پروژه به ترتیب زمانی بررسی می‌شود. اولین ثبت  ماقبل ثبت مورد نظر که مربوط به یک انتشار است یافت می‌شود و به عنوان انتشار ماقبل آن ثبت در نظر گرفته می‌شود. \\

 در نهایت جدولی به نام BugInfo تولید شده که نمایی از آن در شکل \ref{fig:bug-info} آمده است. این جدول ۴۰۵ سطر دارد که بیشتر از تعداد کل خطاهای ذکر شده در مجموعه داده‌ی \lr{defects4j} است. علت این است که یک خطا می‌تواند خطا در چندین پرونده به طور همزمان باشد و از آنجا که پیش‌بینی در سطح پرونده انجام می‌شود لازم است اطلاعات برای پرونده‌ها ذخیره شود.

\begin{figure}[H]
\centering
\includegraphics[width=1\textwidth]{img/case_study/bug-info.png}
\caption{نمایی از جدول محتوای اطلاعات پرونده‌های حاوی خطا}
\label{fig:bug-info}
\end{figure}


\subsection{ استخراج معیارهای فرآیند}
در  این قسمت نحوه‌ی استخراج هر یک از معیارهای ذکر شده در قسمت \ref{sec:method-phase1} بیان می‌شود. \\
\textbf{‫تعداد ثبت در سیستم کنترل نسخه‬:}
اولین راه حلی که به ذهن می‌رسد استفاده ی مستقیم از Jgit برای این کار است.  به این صورت که تعداد ثبت‌های بین ثبت کنونی و انتشار قبلی  بررسی کرده و تعداد  ثبت‌هایی که در آن‌ها فایل حاوی خطا تغییر کرده است شمرده شوند. مشکل این راه این است که بسیار پر هزینه  خواهد بود زیرا مرتبا باید عملیات \واژه{ورودی/خروجی} بر روی دیسک انجام پذیرد و همچنین بررسی‌های تکراری بسیاری انجام می‌گیرد. به عنوان مثال دو ثبت حاوی خطا را در نظر بگیرید که دارای انتشار ما قبل یکسانی هستند. تعدادی از بررسی‌های ثبت‌های ما بین آن‌ها تا ثبت مربوط به انتشار دارای همپوشانی خواهد بود. از طرف دیگر می‌توان اطلاعاتی که در بررسی ثبت‌ها بدست می‌آید در محاسبه‌ی معیارهای دیگر نیز مورد استفاده قرار گیرد.\\
همچنین برای یافتن ثبت‌های بین انتشار و ثبت مورد نظر نمی‌توان از تاریخ ثبت آنها استفاده کرد. زیرا تعداد زیادی از ثبت‌های ابتدای  برخی پروژه ‌های مورد مطالعه دارای تاریخ یکسانی هستند استفاده از تاریخ غیر ممکن می‌شود. علت داشتن تاریخ یکسان احتمالاً مهاجرت از یک نوع مخرن نرم‌افزاری به نوع گیت بوده است. \\

 بنابرین کل ثبت‌های پروژه‌ها مورد بررسی قرار گرفت و دو جدول تولید شد.
جدول اول به نام CommitInfo‌ که اطلاعات کلی ثبت‌ها را در بر می‌گیرد و جدول دوم  CommitChangedFile که اطلاعات مربوط به پرونده‌هایی که در یک ثبت از برنامه نسبت به ثبت قبلی تغییر کرده است نگهداری می‌شود.  در این جدول برای هر پرونده تعداد خطوط اضافه و کم شده نسبت به ثبت قبلی ذخیره شده است. در جدول اول \lr{Sequence\textunderscore Number} نشان می‌دهد که چندمین نسخه از ابتدای پروژه می‌باشد و این عدد در هنگام بررسی‌ها به آن ثبت داده‌ شده زیرا برای یافتن ثبت‌های بین ثبت کنونی و ثبت مربوط به انتشار قبلی لازم است از آنها استفاده شود. \\
  هر سطر از جدول دوم یک کلید خارجی دارد به سطری از جدول اول. قسمتی از جدول CommitInfo  در شکل \ref{fig:commit-info} و جدول CommitChangeFile در شکل \ref{fig:change-file-info}  زیر آمده است:

\begin{figure}[H]
	\centering
	\includegraphics[width=1\textwidth]{img/case_study/commit-info.png}
	\caption{نمایی از جدول اطلاعات ثبت‌ها}
	\label{fig:commit-info}
\end{figure}


\begin{figure}[H]
	\centering
	\includegraphics[width=1\textwidth]{img/case_study/change-file-info.png}
	\caption{ نمایی از جدول تغییرات پرونده‌ها در ثبت‌ها }
	\label{fig:change-file-info}
\end{figure}

در نهایت با استفاده از قطعه کد \ref{code:commit-info} اطلاعات مربوط به ثبت مورد نظر و ثبت انتشار بازیابی می‌شوند و سپس  از شماره‌ی دنباله‌ی آنها در پرسمان موجود در قطعه کد \ref{code:ncomm} استفاده می‌شود و معیار محاسبه می‌گردد.



\begin{latin}
	\begin{lstlisting}[language=SQL]
SELECT  * from CommitInfo CI where CI.COMMIT_GIT_ID = :gitId AND CI.PROJECT = :project
\end{lstlisting}
\end{latin}
\captionof{lstlisting}{بازیابی اطلاعات ثبت}
\label{code:commit-info}

\begin{latin}
\begin{lstlisting}[language=SQL]
SELECT  count(*) from CommitChangedFile CC where CC.COMMIT_INFO_ID IN
	(SELECT CI.ID from CommitInfo CI WHERE CI.SEQUENCE_NUMBER BETWEEN 
	 :startSeq AND :endSeq AND CI.PROJECT = :project)
AND CC.FILE_NAME = :fileName
\end{lstlisting}
\end{latin}
\captionof{lstlisting}{محاسبه‌ی معیار تعداد ثبت‌ در سیستم کنترل نسخه}
\label{code:comm}

\متن‌سیاه{‫تعداد توسعه‌دهندگان فعال:‬}
به منظور محاسبه‌ی این معیار تعداد آدرس ایمیل‌های ثبت‌کننده‌های ثبت‌هایی شمرده می‌شود که آن ثبت‌ها شماره‌ی دنباله‌ی آنها بین شماره‌ی دنباله‌ی ثبت پرونده‌ی مورد نظر و ثبت انتشار قبلی است و همچنین در آن ثبت پرونده‌ی مورد نظر در آن ثبت‌ها تغییر کرده است. به عبارت دیگر ثبت‌هایی که نام پرونده در جدول  CommitChangeFile  برای آن‌ها وجود دارد. 

\begin{latin}
\begin{lstlisting}[language=SQL]
SELECT  count(DISTINCT CI.COMMITTER_MAIL) from CommitInfo CI WHERE
CI.SEQUENCE_NUMBER BETWEEN :startSeq AND :endSeq AND CI.PROJECT = 
project AND CI.ID IN 
	(SELECT CC.COMMIT_INFO_ID from CommitChangedFile CC where CC.FILE_NAME = :fileName)
\end{lstlisting}
\end{latin}
\captionof{lstlisting}{محاسبه‌ی تعداد توسعه‌دهندگان فعال}

\textbf{تعداد توسعه‌دهندگان متمایز:}
برای محاسبه‌ی معیار از پرسمان قبلی استفاده می‌شود اما اینبار به جای استفاده \lr{Sequence\_Number} انتشار قبلی، عدد یک  قرار داده می‌شود که از ابتدای پروژه توسعه دهندگان شمرده شوند. 
\\

\متن‌سیاه{‫مقدار نرمال‌سازی شده‌ی تعداد خطوط اضافه شده:‬}
از پرسمان \ref{code:added-line-file} جهت محاسبه‌ی مجموع تعداد خطوط اضافه شده به پرونده در طول انتشار استفاده می‌شود و از پرسمان \ref{code:added-line-project} جهت محاسبه‌ی مجموع خطوط اضافه شده به پروژه استفاده می‌شود. سرانجام حاصل پرسمان اول بر دوم تقسیم می‌شود. 

\begin{latin}
	\begin{lstlisting}[language=SQL]
SELECT sum(CC.ADDED_LINES) from CommitChangedFile CC where
CC.COMMIT_INFO_ID IN
	(SELECT CI.ID from CommitInfo CI WHERE CI.SEQUENCE_NUMBER BETWEEN 
	:startSeq AND :endSeq AND CI.PROJECT = :project)
AND CC.FILE_NAME = :fileName
	\end{lstlisting}
\end{latin}
\captionof{lstlisting}{محاسبه‌ی تعداد خطوط اضافه شده به پرونده}
\label{code:added-line-file}

\begin{latin}
\begin{lstlisting}[language=SQL]
SELECT  sum(CC.ADDED_LINES) from CommitChangedFile CC where 
CC.COMMIT_INFO_ID IN
	(SELECT CI.ID from CommitInfo CI WHERE CI.SEQUENCE_NUMBER
	 BETWEEN :startSeq AND :endSeq AND CI.PROJECT = :project)
\end{lstlisting}
\end{latin}
\captionof{lstlisting}{محاسبه‌ی تعداد خطوط اضافه شده به پروژه}
\label{code:added-line-project}

\متن‌سیاه{‫مقدار نرمال‌سازی شده‌ی تعداد خطوط حذف شده:‬} به طور مشابه معیار قبلی محاسبه می‌گردد.

\textbf{درصد خطوطی که مالک فایل مشارکت کرده:}
دستور Blame در Jgit نشان می‌دهد که هر خط از پرونده در یک ثبت  در کدام یک از ثبت‌های گذشته اضافه شده است.  با یافتن ثبت مسئول اضافه کردن آن خط نویسنده‌ی آن خط مشخص می‌شود که همان ثبت‌کننده است. با کمک این دستور به دلایل مشابه ساخت جداول مربوط به ثبت‌ها، جدولی با عنوان Participation ساخته شده که در آن هر سطر نشان می‌دهد که یک نویسنده در یک نسخه از برنامه چند درصد از خطوط به وی اختصاص دارد. در شکل \ref{fig:participation} نمایی از این جدول آورده شده است.  از این جدول علاوه بر محاسبه‌ی این معیار برای یافت سایر معیارها نیز استفاده خواهد شد. در نهایت معیاری که در ابتدا بسیار پیچیده به نظر می رسید به کمک پرسمان ساده‌ی \ref{code:own} محاسبه خواهد شد. 

\begin{figure}[H]
	\centering
	\includegraphics[width=1\textwidth]{img/case_study/participation.png}
	\caption{نمایی از مخزن نرم‌افزاری}
	\label{fig:participation}
\end{figure}


\begin{latin}
	\begin{lstlisting}[language=SQL]
SELECT max(PARTICIPATION_PERCENT) from Participation P 
where COMMIT_ID = :commitId AND FILE_NAME = :fileName")
\end{lstlisting}
\end{latin}
\captionof{lstlisting}{محاسبه‌ی درصد خطوط مالک پرونده}
\label{code:own}

\textbf{تعداد مشارکت‌کنندگان جزئی:‬}
 با استفاده از جدول Participation و پرسمان  \ref{code:minor}  معیار محاسبه می‌شود. مقدار minorThereshold برابر ۵ درصد قرار می‌گیرد.

\begin{latin}
	\begin{lstlisting}[language=SQL]
SELECT count(AUTHOR_EMAIL) from Participation P 
where COMMIT_ID = :commitId AND FILE_NAME = :fileName
and PARTICIPATION_PERCENT < :minorThreshold
\end{lstlisting}
\end{latin}
\captionof{lstlisting}{محاسبه‌ي تعداد مشارکت‌کنندگان جزئی}
\label{code:minor}

\textbf{تعداد ثبت‌های همسایگان:‬}
ابتدا لازم است که همسایگان پرونده در یک ثبت و نیز تعداد دفعات همسایگی در طول انتشار مشخص شود. این عمل به وسیله‌ی پرسمان \ref{code:neighbor} انجام می‌شود. سپس معیار تعداد ثبت‌ها در سیستم کنترل نسخه مشابه قبل با استفاده از کد \ref{code:comm} محاسبه می‌گردد و از آنها میانگین وزن‌دهی شده گرفته می‌شود.

\begin{latin}
\begin{lstlisting}[language=SQL]
SELECT FILE_NAME as `name`, count(ID) as `frequency` FROM
CommitChangedFile WHERE COMMIT_INFO_ID IN
	(SELECT COMMIT_INFO_ID FROM CommitChangedFile WHERE FILE_NAME = :fileName) 
AND COMMIT_INFO_ID IN
	(SELECT CI.ID from CommitInfo CI WHERE CI.SEQUENCE_NUMBER BETWEEN :startSeq AND :endSeq AND PROJECT = :project)
AND FILE_NAME != :fileName GROUP BY FILE_NAME
\end{lstlisting}
\end{latin}
\captionof{lstlisting}{یافتن همسایگان و تعدد همسایگی}
\label{code:neighbor}

\textbf{تعداد توسعه‌دهندگان فعال همسایگان:‬}
 به طور مشابه با معیار قبلی محاسبه می‌شود.\\
\textbf{‫تعداد توسعه‌دهندگان متمایز همسایگان:‬}
 به طور مشابه با معیار قبلی محاسبه می‌شود. \\

\textbf{تجربه‌ی مالک فایل:‬‬}
 برای محاسبه‌ی معیار ابتدا   با استفاده از پرسمان \ref{code:find-owner} مالک پرونده مشخص می‌شود. سپس تعداد ثبت‌هایی که مالک پرونده از ابتدای پروژه تا آن زمان ثبت کرده است  با استفاده از پرسمان \ref{code:commit-of-commiter} شمرده می‌شود. به ترتیب از دو جدول Participation 
 و 
 CommitInfo  استفاده می‌شود.

\begin{latin}
\begin{lstlisting}[language=SQL]
SELECT AUTHOR_EMAIL FROM Participation P WHERE COMMIT_ID = :commitId 
AND FILE_NAME = :fileName AND PARTICIPATION_PERCENT  = 
	(SELECT max(PARTICIPATION_PERCENT) FROM Participation P2 
	 WHERE P2.COMMIT_ID = :commitId  AND P2.FILE_NAME = :fileName)
\end{lstlisting}
\end{latin}
\captionof{lstlisting}{یافتن مالک پرونده}
\label{code:find-owner}



\begin{latin}
\begin{lstlisting}[language=SQL]
SELECT  count(*) from CommitInfo CI where CI.SEQUENCE_NUMBER BETWEEN
:startSeq AND :endSeq AND CI.PROJECT = :project AND CI.COMMITTER_MAIL =
:authorEmail
\end{lstlisting}
\end{latin}
\captionof{lstlisting}{شمارش تعداد ثبت‌های یک ثبت کننده در بازه‌ی زمانی داده شده}
\label{code:commit-of-commiter}

\textbf{‫تجربه‌ی تمام مشارکت‌کنندگان:‬}
ابتدا همه‌ی توسعه‌دهندگان پرونده با استفاده از پرسمان \ref{code:contributers} مشخص می‌شوند.  سپس میزان تجربه‌ی هر یک با استفاده از پرسمان \ref{code:commit-of-commiter} جداگانه محاسبه می‌شود و از آن‌ها میانگین هندسی گرفته می شود. 
\begin{latin}
\begin{lstlisting}[language=SQL]
SELECT AUTHOR_EMAIL FROM Participation P WHERE COMMIT_ID = :commitId 
AND FILE_NAME = :fileName
\end{lstlisting}
\end{latin}
\captionof{lstlisting}{یافتن مشارکت‌کنندگان در پرونده}
\label{code:contributers}


در نهایت جدولی برای معیارهای فرآیند تولید می‌شود که نمایی از آن در شکل \ref{fig:process-metics} آورده شده است. 
\begin{figure}[H]
	\centering
	\includegraphics[width=1\textwidth]{img/case_study/process-metrics.png}
	\caption{نمایی از جدول معیارهای فرآیند }
	\label{fig:process-metics}
\end{figure}


 
\section{رویکرد دوم: معیارهای فرآیند مبتنی بر جهش}
همانطور که در قسمت \ref{sec:method-phase-two} اشاره شده چهار معیار معرفی شدند و مبتنی بر جهش نامیده شدند. این قسمت به نحوه‌ی پیاده‌سازی دسته‌ی دوم از معیارها را شرح خواهد داد. 
\begin{itemize}
\item
\متن‌سیاه{	تعداد جهش‌یافته‌های تولید شده‌ی جدید نسبت به انتشار قبلی برنامه:}
به منظور محاسبه‌ی این معیار ابتدا لازم است که مشخص شود که پرونده‌ی مورد نظر نسبت به انتشار قبلی چه تغییراتی داشته است. این کار با استفاده از ابزار JGit انجام  می‌شود. JGit این امکان را فراهم می‌کند که دو پرونده در دو ثبت متفاوت مقایسه شوند و مشخص می‌کند که کدام خطوط حذف شده‌اند و کدام خطوط اضافه شده‌اند. در اینجا لازم است خطوط اضافه شده  مشخص شود. سپس با استفاده از ابزار Major جهش‌یافته‌ها تولید می‌شود. در  قسمت  \ref{sec:tools-major} توضیح داده شد که پس تولید جهش‌یافته‌ها یک فایل خروجی نیز به نام mutant.log تولید می‌شود که در آن مشخص شده در هر خط از برنامه چه جهش‌یافته‌هایی تولید شده است. حال کافیست تعداد جهش‌یافته‌های تولید شده در خطوطی شمرده شوند که ابزار Jgit آن‌ها را به عنوان خطوط جدید نسبت به انتشار قبلی معرفی کرده است. بدین ترتیب این معیار محاسبه خواهد شد.\\
لازم به ذکر است روش یاد شده پایه ی محاسبه‌ی معیار بعدی و معیارهای رویکرد سوم است.
\end{itemize}

\section{رویکرد سوم: معیارهای ترکیبی جهش-فرآیند}

نحوه‌ی محاسبه به این صورت خواهید بود که ابتدا ثبت‌هایی از برنامه در طول آخرین انتشار که در آن فایل مورد نظر تغییر کرده است  توسط  پرسمان \ref{code:commit-during-release}بازیابی می شود. سپس برای هر ثبت تعداد جهش یافته‌های جدید نسبت به ثبت قبلی محاسبه می‌شود و برای محاسبه‌ی جهش‌یافته‌های حذف شده تعداد جهش یافته‌ها در  ثبت قبلی را یافته و آن‌ها که جز خطوط حذف شده در ثبت بعدی است شمرده می شود. تعداد جهش‌یافته‌های اضافه و حذف شده در ثبت‌ها جمع شده و بر تعداد ثبت‌های کل پروژه در طول انتشار تقسیم می گردد.



\begin{latin}
\begin{lstlisting}[language=SQL]
SELECT CC.* from CommitChangedFile CC, CommitInfo CI where CC.COMMIT_INFO_ID = CI.ID
AND CI.SEQUENCE_NUMBER BETWEEN :startSeq AND :endSeq
AND CI.PROJECT = :project
AND CC.FILE_NAME = :fileName ORDER BY CI.SEQUENCE_NUMBER asc
\end{lstlisting}
\end{latin}
\captionof{lstlisting}{بازیابی اطلاعات ثبت‌هایی که یک فایل در بازه‌ی مشخص در آنها تغییر کرده است}
\label{code:commit-during-release}

\chapter{ارزیابی}
\label{chap:evaluation}
در این بخش به تشریح نحوه‌ی ساخت مدلهای پیش‌بینی و ارزیابی معیارهای شرح داده شده در فصل \ref{chap:method} پرداخته می‌شود. با استفاده از معیارهای استخراج شده در فصل \ref{chap:case-study} مدلهای مورد نظر شاخته می‌شوند. ساخت مدل‌ها در زبان R انجام می‌گردد به وسیله‌ی بسته‌ی \نام{کرت}{Caret} \cite{kuhn2008caret} انجام می‌شود.\\
در ساخت و ارزیابی مدل‌ها از روش \واژه{ارزیابی میان دسته‌ای} استفاده می‌شود که تعداد دسته‌ها ۱۰ و تعداد تکرار نیز ۱۰ مورد می‌باشد. لازم به ذکر است که دسته‌بندی‌ها به طور تصادفی انجام می‌شود.  همچنین در بسته‌ی کرت در هر روش دسته‌بندی پارامترهای مختلفی به طور پیش فرض به کار گرفته می‌شود تا بهترین مدل ممکن ساخته شود. در ابتدا ۱۰ درصد از داده‌ها به عنوان داده‌ی آزمون جدا می‌شود. با استفاده از ۹۰ درصد باقی‌مانده به ساخت مدل پرداخته می‌شود.  با استفاده از ارزیابی میان‌دسته‌ای و تنظیم خودکار پارامترهای مختلف مدل نهایی ساخته شده و از این مدل برای پیش‌بینی داده‌های آزمون مورد استفاده قرار گرفته است.  \\
در ادامه هر یک از رویکردها به طور جداگانه ارزیابی شده و نتایج در زیر آمده است. 


\section{ارزیابی معیارهای فرآیند و جهش}
همانطور که اشاره شد هدف از این آزمایش این است که مشخص شود قرارگیری معیارهای جهش در کنار معیارهای فرآیند باعث بهبود پیش‌بینی خطا می‌گردد یا خیر و این تاثیر تا چه میزان است. به همین منظور یک با استفاده از ۱۲ معیار فرآیند یک مدل پیش‌بینی ساخته شده و مدل دیگری  با استفاده از ۱۲ معیار فرآیند و ۴ معیار جهش ساخته شده است. در نهایت این دو مدل با استفاده از معیارهای ارزیابی مختلف با هم مقایسه شدند. بدیهی است که دو مدلی که با هم مقایسه می‌شوند بجر در معیارهای استفاده شده (بردار ویژگی) به منظور ساخت مدل از هیچ منظری تفاوت ندارند و داده‌های یکسانی در ساخت و ارزیابی آنها استفاده شده. \\
در این ارزیابی از چهار روش دسته‌بندی استفاده شده است. این روش‌های دسته‌بندی بیش از سایرین در مقالات مورد استفاده قرار گرفته‌اند. \\
در جدول \ref{tab:eval-phase1} بخشی از نتایج آمده است. این نتیاج نشان می‌دهد که قرار گیری معیارهای جهش در کنار معیارهای فرآیند موجب بهبود پیش‌بینی خطا به مقدار قابل ملاحظه‌ای می‌شود و در تمام  روشهای  یادگیری موجب بهبود  پیش‌بینی می‌گردد. از میان روشهای دسته‌بندی بهترین عملکرد   پس از افرودن معیارهای جهش  از نظر صحت و دقت را روش \lr{Neural Network} داشته است. روش   \lr{Decition Tree} نیز بهترین عملکرد از نظر معیار بازخوانی را  داشته است. همچنین بیشترین تغییر مثبت در صحت پیش‌بینی پس از افزودن معیارهای جهش را روش \lr{ Neural Network}   و  \lr{Decition Tree} با مقدار $20$ درصد داشته است.  کمترین تاثیر با مقدار $۷.۵$ درصد در روش SVM بوده است. بیشترین افزایش دقت در روش \lr{Decition Tree} بوده است که مقدار آن $15.1$ درصد می‌باشد. از نظر معیار بازخوانی بیشترین تغییر مثبت را درخت تصمیم دارد که رشد $25$ درصدی داشته و روش \lr{Logestic Regression} کاهش $2.5$ درصدی داشته است. به طور کلی می‌توان این نتیجه حاصل شود که بیشترین بهبود در روش \lr{Decision Tree} و کمترین در SVM روی داده است.
\begin{table}[H] 
	\renewcommand*{\arraystretch}{1.3}	
	\centering \caption{مقایسه‌ی معیارهای فرآیند به تنهایی  و به همراه جهش}
	\label{tab:eval-phase1}
 \rowcolors{2}{blue!15}{white}   
	\begin{tabular}{|c|c|c|c|c|}
		
		\hline
		\hline
معیار & نام روش  & صحت & دقت & بازخوانی	
		\\
		\hline
		\hline
فرآیند & 
\lr{Decition Tree} & $0.587$&$0.574$&$0.675$
 \\
		\hline
		فرآیند و جهش & 
\lr{Decition Tree} & $0.787$&$0.725$&$0.925$
		\\
		\hline
فرآیند & 
\lr{SVM} & $0.662$&$0.685$&$0.600$
\\
\hline
فرآیند و جهش & 
\lr{SVM} & $0.737$&$0.806$&$0.625$
\\

\hline
فرآیند &
\lr{Logestic Regression} &   $0.612$&$0.591$&$0.725$
\\
\hline
فرآیند و جهش & 
\lr{Logestic Regression} & $0.725 $&$0.736$&$0.700$
\\
\hline
فرآیند &
\lr{Nueral Network} & $0.612$&$0.725$&$0.591$
\\
\hline
فرآیند و جهش & 
\lr{Nueral Network} & $0.812$&$0.777$&$0.875$
\\
\hline		
	\end{tabular}
\end{table}

در شکل \ref{fig:ROC-phase1} نمودارهای ROC به تفکیک روش دسته‌بندی آمده است. در هر یک از زیر شکل‌ها منحنی ROC مربوط به  دو مدل با هم مقایسه شده است. درمدل اول که در ساخت آن از معیارهای فرآیند استفاده شده  با خط ممتد نمایش داده شده است و مدل دوم  از معیارهای فرآیند به همراه معیارهای جهش ساخته شده‌ است  و با خط چین نمایش داده شده‌. همانطور که قابل مشاهده است در تمامی روش‌ها دسته‌بندی مدل‌ حاوی معیار جهش مساحت زیر نمودار بیشتری نسبت به مدل دیگر دارند و نشان از عملکرد بهتر این مدل‌ها می‌باشد. 
\begin{figure}[H]
	\begin{subfigure}{.5\textwidth}
		\centering
		\includegraphics[width=\linewidth]{img/evaluation/phase1-roc-dt.pdf}
		\caption{\lr{Decition Tree}}
	\end{subfigure}
	\begin{subfigure}{.5\textwidth}
	\centering
	\includegraphics[width=\linewidth]{img/evaluation/phase1-roc-svm.pdf}
	\caption{SVM}
\end{subfigure}
	\begin{subfigure}{.5\textwidth}
	\centering
	\includegraphics[width= \linewidth]{img/evaluation/phase1-roc-lr.pdf}
	\caption{\lr{Logestic Regression}}
\end{subfigure}
	\begin{subfigure}{.5\textwidth}
	\centering
	\includegraphics[width= \linewidth]{img/evaluation/phase1-roc-nn.pdf}
	\caption{\lr{Nueral Network}}
\end{subfigure}
\caption{نمودارهای ROC معیارهای فرآیند و به همراه جهش}
\label{fig:ROC-phase1}
\end{figure}

در جدول \ref{tab:auc-phase1} مساحت زیر نمودار ROC در هر یک از روش‌های دسته‌بندی آورده شده است.در میان  روش‌های یادگیری به کار گرفته شده بیشترین افزایش مساحت زیر نمودار را \lr{Neural Network} به مقدار $0.226$  واحد داشته است و کمترین  تغییر را نیز  \lr{Logestic Regression} با مقدار $0.038$ واحد داشته است.  به طور متوسط  $0.151$ واحد در مدلها بهبود مشاهده می‌شود. این موضوع نشان از تاثیر قابل توجه معیارهای جهش می‌باشد. 

\begin{table}[H] 
	\renewcommand*{\arraystretch}{1.2}	
	\centering \caption{مقادیر زیر نمودار ROC معیارهای فرآیند و به همراه جهش}
	\label{tab:auc-phase1}
	 \rowcolors{2}{blue!15}{white}   
	\begin{tabular}{|c|c|c|c|c|}
		\hline
		\hline
		معیار & 
		 \lr{ Decition Tree} & SVM &\lr{ Logestic Regression} &\lr{ Neural Network} \\
		 \hline
		 \hline
		 فرآیند & $.596$ & $.697$ & $.643$ & $.593$
		 \\
		 \hline
		 فرآیند و جهش  & $.822$ & $.802$ & $.761$ & $.829$
		 \\
		 \hline
		 
	\end{tabular}
\end{table}

\section{ارزیابی معیارهای فرآیند مبتنی بر جهش }
ارزیابی این معیارها در دو مرحله انجام می‌شود. در مرحله‌ی اول  سه  مدل ساخته می‌شود. این مدل‌ها به ترتیب با استفاده از معیارهای فرآیند، فرآیند و جهش و  مدل آخر با استفاده از معیارهای فرآیند و فرآیند مبتنی بر جهش ساخته می‌شود. در مرحله‌ی دوم دو  مدل ساخته می‌شود. در مدل اول معیارهای فرآیند و جهش مدل پیش‌بینی را خواهد ساخت و در مدل دوم معیارهای فرآیند مبتنی بر جهش نیز به مجموعه‌ی معیارها افزوده می‌شود. 

\subsection{مرحله‌ی اول}
 مقایسه‌ی این مدل‌ها امکان را فراهم می‌کند مشخص شود آیا معیارهای فرآیند مبتنی  بر جهش دارای قابلیت پیش‌بینی هستند یا خیر. همچنین در صورت داشتن این قابلیت مشخص شود که این قابلیت از معیارهای جهش کمتر است یا بیشتر. \\
 
مقایسه‌ی نتایج بدست آمده در جدول \ref{tab:eval-phase2-part1}  با جدول \ref{tab:eval-phase1} نشان می‌دهد که در تمامی روشهای دسته‌بندی  بجز SVM معیار صحت در مدل سوم از مدل اول مقدار بیشتری دارد.  در مدل ساخته شده توسط SVM نیز اختلاف معیار صحت کم می‌باشد(۳ درصد).  این مدل در مقایسه با مدل دوم عملکرد بهتری از نظر معیار صحت و بازخوانی در هیچکدام از روشهای دسته‌بندی نداشته است. از نظر معیار دقت  در  تمامی روشها مدل سوم از مدل اول عملکرد بهتری داشته و حتی در روش \lr{Decition Tree} مدل سوم از  مدل دوم نیز بهتر عملکرده است. از نظر معیار بازخوانی مدل سوم نسبت به مدل اول تنها در روش \lr{Neural Network} عملکرد بهتری داشته، در \lr{Decision Tree} بدون تغییر مانده و در دو روش دیگر کاهش یافته است. \\
می‌توان این نتیجه را برداشت کرد که معیارهای ارائه شده دارای توانایی پیش‌بینی بیشتری نسبت به معیارهای فرآیند به تنهایی هستند.
 \\
 \begin{table}[H] 
 	\renewcommand*{\arraystretch}{1.3}	
 	\centering \caption{نتایج پیش‌بینی‌خطای معیارهای فرآیند مبتنی بر جهش - مرحله‌ی اول} 
 	\label{tab:eval-phase2-part1}

 	\begin{tabular}{|c|c|c|c|}
 		
 		\hline
 		\hline
 		 نام روش  & صحت & دقت & بازخوانی	
 		\\
 		\hline
 		\hline
 		 
 		\lr{Decition Tree} & $0.725 $&$0.750$&$0.675$
 		\\
 		\hline
 	
 		\lr{SVM} & $0.637$&$0.689$&$0.500$
 		
 		\\
 		\hline
 	 
 		\lr{Logestic Regression} & $0.662$&$0.685$&$0.600$
 		\\
 		\hline
 	 
 		\lr{Neural Network} & $0.762$&$0.756$&$0.775$
 		\\
 		\hline
	\end{tabular}
 \end{table}

در شکل \ref{fig:ROC-phase2-part1} نمودارهای ROC سه مدل ساخته شده نشان داده شده است. در زیرشکل‌های (آ)(ج)(د) به وضوح عملکرد بهتر مدل سوم از مدل اول قابل مشاهده است. در زیرشکل (ب)نیز که متعلق به SVM است با رجوع به جدول \ref{tab:auc-phase2-part1} مشخص می‌شود که در این شکل نیز مساحت زیر نمودار ROC در مدل سوم بیشتر از اول است. همچنین مساحت زیر نمودار در مدل سوم در زیرشکل (ج) به مقدار $0.015$ واحد از مدل دوم نیز بیشتر است. 

این نتایج در راستای نتایج بدست آمده از جدول \ref{tab:eval-phase2-part1} می‌باشد. در نهایت می‌توان این نتیجه را گرفت که معیارهای مبتنی بر جهش معرفی شده دارای توانایی پیش‌بینی خطای بیشتری نسبت به معیارهای فرآیند هستند اما این توانایی بیشتر از معیارهای جهش نیست. همچنین از آنجا که هزینه‌ی محاسباتی بیشتری نسبت به معیارهای جهش دارند جایگزینی آنها به جای یکدیگر مزیتی ندارد. 

\begin{figure}[H]
	\begin{subfigure}{.5\textwidth}
		\centering
		\includegraphics[width=\linewidth]{img/evaluation/phase2-part1-roc-dt.pdf}
		\caption{\lr{Decition Tree}}
	\end{subfigure}
	\begin{subfigure}{.5\textwidth}
		\centering
		\includegraphics[width=\linewidth]{img/evaluation/phase2-part1-roc-svm.pdf}
		\caption{SVM}
	\end{subfigure}
	\begin{subfigure}{.5\textwidth}
		\centering
		\includegraphics[width= \linewidth]{img/evaluation/phase2-part1-roc-lr.pdf}
		\caption{\lr{Logestic Regression}}
	\end{subfigure}
	\begin{subfigure}{.5\textwidth}
		\centering
		\includegraphics[width= \linewidth]{img/evaluation/phase2-part1-roc-nn.pdf}
		\caption{\lr{Nueral Network}}
	\end{subfigure}
	\caption{نمودارهای ROC معیارهای فرآیند ، فرآیند و جهش ، فرآیند مبتنی بر جهش}
	\label{fig:ROC-phase2-part1}
\end{figure}

\begin{table}[H] 
	\renewcommand*{\arraystretch}{1.2}	
	\centering \caption{مقادیر زیر نمودار ROC معیارهای فرآیند مبتنی جهش}
	\label{tab:auc-phase2-part1}
	\begin{tabular}{|c|c|c|c|}
		\hline
		\hline
		 
		\lr{ Decition Tree} & SVM &\lr{ Logestic Regression} &\lr{ Neural Network} \\
		\hline
		\hline
		 $.772$ & $.707$ & $.693$ & $.798$
		\\
		\hline
	
		
	\end{tabular}
\end{table}

\subsection{مرحله‌ی دوم}
همانطور که اشاره شد دو مدل ساخته می‌شود که مدل اول از معیارهای فرآیند و جهش استفاده می‌کند و مدل دوم همگی معیارها (با افزودن معیارهای فرآیند مبتنی بر جهش) در ساخت مدل استفاده می‌شود. هدف از این آزمایش این است که مشخص شود در صورتی که معیارهای ارائه شده‌ی جدید در کنار معیارهای قبلی قرار گیرد، در پیش‌بینی بهبودی حاصل می‌گردد یا خیر. 

نتایج بدست آمد در جدول \ref{tab:eval-phase2-part2} نشان می‌دهد که مدل دوم  در هیچ یک از روش‌ها بجز \lr{Logestic Regression} از نظر معیارهای صحت، دقت و بازخوانی نسبت به مدل اول بهبودی پیدا نکرده است. همچنین در روش \lr{Decition Tree} نتایج دو مدل یکسان است.  در روش \lr{Logestic Regression} مدل دوم در معیار صحت $1.2$ درصد افزایش، در معیار دقت $5$ درصد کاهش و $17.5$ درصد، بازخوانی افزایش داشته است. 

 \begin{table}[H] 
	\renewcommand*{\arraystretch}{1.3}	
	\centering \caption{نتایج پیش‌بینی خطای مدل حاصل از بکارگیری تمامی معیارها}
	\label{tab:eval-phase2-part2}
	
	\begin{tabular}{|c|c|c|c|}
		
		\hline
		\hline
		نام روش  & صحت & دقت & بازخوانی	
		\\
		\hline
		\hline
		
		\lr{Decition Tree} & $0.787$&$0.725$&$0.925$
		\\
		\hline
		
		\lr{SVM} & $0.712$&$0.774$&$0.600$
		\\
		\hline
		
		\lr{Logestic Regression} & $0.737 $&$0.686$&$0.875$
		\\
		\hline
		
		\lr{Nueral Network} & $0.750$&$0.717$&$0.825$
		\\
		\hline
	\end{tabular}
\end{table}
 
 نمودارهای ROC هر یک از این دو مدل در روش‌های دسته‌بندی مختلف در شکل \ref{fig:ROC-phase2-part2} آمده است. در روش‌های مختلف مدل اول با دوم تفاوت چندانی ندارند و طبق جدول \ref{tab:auc-phase2-part2} تنها در مدل‌های حاصل از روش \lr{Logestic Regression} به مقدار $0.009$ واحد مساحت زیر نمودار افزایش پیدا  کرده است. بنابرین قرار گیری معیارهای فرآیند مبتنی بر جهش نمی‌تواند به بهبود پیش‌بینی بیانجامد.
\begin{figure}[H]
	\begin{subfigure}{.5\textwidth}
		\centering
		\includegraphics[width=\linewidth]{img/evaluation/phase2-part2-roc-dt.pdf}
		\caption{\lr{Decition Tree}}
	\end{subfigure}
	\begin{subfigure}{.5\textwidth}
		\centering
		\includegraphics[width=\linewidth]{img/evaluation/phase2-part2-roc-svm.pdf}
		\caption{SVM}
	\end{subfigure}
	\begin{subfigure}{.5\textwidth}
		\centering
		\includegraphics[width= \linewidth]{img/evaluation/phase2-part2-roc-lr.pdf}
		\caption{\lr{Logestic Regression}}
	\end{subfigure}
	\begin{subfigure}{.5\textwidth}
		\centering
		\includegraphics[width= \linewidth]{img/evaluation/phase2-part2-roc-nn.pdf}
		\caption{\lr{Nueral Network}}
	\end{subfigure}
	\caption{نمودارهای ROC معیارهای جهش و فرآیند و تمامی معیارها }
	\label{fig:ROC-phase2-part2}
\end{figure}

\begin{table}[H] 
	\renewcommand*{\arraystretch}{1.2}	
	\centering \caption{مقادیر زیر نمودار ROC تمامی معیارها}
	\label{tab:auc-phase2-part2}
	\begin{tabular}{|c|c|c|c|}
		\hline
		\hline
		
		\lr{ Decition Tree} & SVM &\lr{ Logestic Regression} &\lr{ Neural Network} \\
		\hline
		\hline
		$.822$ & $.786$ & $.770$ & $.830$
		\\
		\hline
		
		
	\end{tabular}
\end{table}

\section{ارزیابی معیارهای ترکیبی فرآیند-جهش}
 در این قسمت به ارزیابی دو معیار مطرح شده پرداخته می‌شود. به منظور ارزیابی آنها دو مدل با استفاده از هر یک از روشهای انتخابی استفاده می‌شود. در مدل اول معیارهای فرآیند استفاده می‌شود و در مدل دوم معیار \موکد{مقدار نرمال شده‌ی خطوط اضافه شده} با معیار \موکد{تعداد خطوط اضافی وزن‌دهی شده} جایگزین می‌شود و معیار \موکد{مقدار نرمال شده‌ی خطوط حذف شده} به طور مشابه جایگزین می‌شود. سایر معیارهای مدل دوم با مدل اول یکسان خواهد بود.

 نتایج به دست آمده در جدول \ref{tab:eval-phase3} نشان می‌دهد که معیارهای صحت، دقت و بازخوانی برای تمامی مدل‌ها بجز مدل ساخته شده توسط روش SVM افزایش قابل ملاحظه‌ای داشته است. بیشترین افزایش صحت در روش \lr{Neural Network} به میزان $13.8$ درصد روی داده است. از نظر افزایش دقت بیشترین تغییر مثبت در روش \lr{Decition Tree} بوده است که $13.1$ درصد رشد داشته است. معیار بازخوانی در دو  روش \lr{Logestic Regression} و \lr{Neural Network} به ترتیب $8.4$ و $2.5$ رشد داشته و در دو روش دیگر کاهش داشته است. \\
 به طور میانگین معیار صحت $6.6$ درصد افزایش، معیار دقت $6$ درصد افزایش  و معیار بازخوانی $0.4$ درصد کاهش داشته است. در نهایت می‌توان این نتیجه را گرفت که معیارهای ترکیبی جهش-فرآیند موجب بهبود در صحت و دقت پیش‌بینی می‌شوند و تاثیر چندانی در بازخوانی ندارند. لازم به ذکر است که تنها دو معیار از ۱۲ معیار مورد استفاده در دو مدل ساخته شده با هم متفاوت هستند که این دو معیار توانسته‌اند حدود $6$ درصد صحت و دقت را بهبود بخشند. این امر نشان از تاثیر قابل ملاحظه‌ی این معیارها می‌باشد. \\
 



\begin{table}[H] 
	\renewcommand*{\arraystretch}{1.3}	
	\centering \caption{مقایسه‌ی معیارهای فرآیند و معیارهای ترکیبی جهش-فرآیند}
	\label{tab:eval-phase3}
	\rowcolors{2}{blue!15}{white}   
	\begin{tabular}{|c|c|c|c|c|}
		
		\hline
		\hline
		معیار & نام روش  & صحت & دقت & بازخوانی	
		\\
		\hline
		\hline
		فرآیند & 
		\lr{Decition Tree} & $0.587$&$0.574$&$0.675$
		\\
		\hline
		ترکیبی جهش-فرآیند& 
		\lr{Decition Tree} & $0.675$&$0.705$&$0.600$
		\\
		\hline
		فرآیند & 
		\lr{SVM} & $0.662$&$0.685$&$0.600$
		\\
		\hline
		ترکیبی جهش-فرآیند & 
		\lr{SVM} & $0.637$&$0.666$&$0.550$
		\\
		
		\hline
		فرآیند &
		\lr{Logestic Regression} & $0.612$&$0.591$&$0.725$
		\\
		\hline
		ترکیبی جهش-فرآیند & 
		\lr{Logestic Regression} & $0.675 $&$0.652$&$0.750$
		\\
		\hline
		فرآیند &
		\lr{Nueral Network} & $0.612$&$0.725$&$0.591$
		\\
		\hline
		ترکیبی جهش-فرآیند & 
		\lr{Nueral Network} & $0.750$&$0.794$&$0.675$
		\\
		\hline		
	\end{tabular}
\end{table}

در شکل \ref{fig:ROC-phase3} نمودارهای ROC به تفکیک روش دسته‌بندی آمده است. در هر یک از زیر شکل‌ها منحنی ROC مربوط به  دو مدل با هم مقایسه شده است. درمدل اول که در ساخت آن از معیارهای فرآیند استفاده شده  با خط ممتد نمایش داده شده است و مدل دوم  از جایگزینی دو معیار فرآیند با معیارهای ترکیبی جهش-فرآیند ساخته شده و با خط چین نمایش داده شده‌. همانطور که قابل مشاهده است در تمامی روش‌ها بجز  SVM مدل‌ دوم مساحت زیر نمودار بیشتری نسبت به مدل اول داشته است.
\begin{figure}[H]
	\begin{subfigure}{.5\textwidth}
		\centering
		\includegraphics[width=\linewidth]{img/evaluation/phase3-roc-dt.pdf}
		\caption{\lr{Decition Tree}}
	\end{subfigure}
	\begin{subfigure}{.5\textwidth}
		\centering
		\includegraphics[width=\linewidth]{img/evaluation/phase3-roc-svm.pdf}
		\caption{SVM}
	\end{subfigure}
	\begin{subfigure}{.5\textwidth}
		\centering
		\includegraphics[width= \linewidth]{img/evaluation/phase3-roc-lr.pdf}
		\caption{\lr{Logestic Regression}}
	\end{subfigure}
	\begin{subfigure}{.5\textwidth}
		\centering
		\includegraphics[width= \linewidth]{img/evaluation/phase3-roc-nn.pdf}
		\caption{\lr{Nueral Network}}
	\end{subfigure}
	\caption{نمودارهای ROC معیارهای فرآیند و به همراه جهش}
	\label{fig:ROC-phase3}
\end{figure}

در جدول \ref{tab:auc-phase3} مساحت زیر نمودار ROC  دو مدل به تفکیک روش دسته‌بندی آورده شده است. در میان  روش‌های یادگیری به کار گرفته شده بیشترین افزایش مساحت زیر نمودار را روش  \lr{Neural Network} به مقدار $0.128$  واحد داشته است. به طور متوسط  $0.051$ واحد در مدل‌ها بهبود مشاهده می‌شود. این موضوع نشان می‌دهد که معیارهای ترکیبی جهش-فرآیند از نظر مساحت زیر نمودار ROC نیز موجب تغییر مثبت ایجاد می‌کند. \\

با توجه به اینکه تنها روش SVM  نتایج ضعیفی نسبت به سایرین داشته است این موضوع را می‌توان با توجه نحوه‌ی عملکرد این روش توجیه کرد. به طور خلاصه این روش سعی می‌کند که  \واژه{فضای ویژگی}  را با ایجاد یک \واژه{ابرصفحه} به دسته‌های مختلف تقسیم کند اما توزیع نقاط داده در فضای ویژگی به نحوی نیست که این روش بتواند به خوبی عمل کند. 

\begin{table}[H] 
	\renewcommand*{\arraystretch}{1.2}	
	\centering \caption{مقادیر زیر نمودار ROC معیارهای فرآیند و معیارهای ترکیبی جهش-}
	\label{tab:auc-phase3}
	\rowcolors{2}{blue!15}{white}   
	\begin{tabular}{|c|c|c|c|c|}
		\hline
		\hline
		معیار & 
		\lr{ Decition Tree} & SVM &\lr{ Logestic Regression} &\lr{ Neural Network} \\
		\hline
		\hline
		فرآیند & $.596$ & $.697$ & $.643$ & $.593$
		\\
		\hline
		جهش-فرآیند  & $.654$ & $.656$ & $.705$ & $.721$
		\\
		\hline
		
	\end{tabular}
\end{table}
\فصل{نتیجه‌گیری و کارهای آتی}
\برچسب{chap:future}
در این پایانامه سعی شد که تاثیر معیار‌های جهش بر پیش‌بینی خطا در هنگام قرار گیری در کنار معیار‌های فرآیند ارزیابی  شود و معیارهای جدیدی با استفاده از مفاهیم تحلیل جهش و تاریخچه‌ی توسعه‌ی نرم‌افزار ارائه گردد. در فصل \ref{chap:intro} به بیان مسئله و مفاهیم مقدماتی پرداخته شد. در فصل \ref{chap:survey}  پژوهش‌های پیشین در حوزه‌ی پیش‌بینی خطا مورد بررسی قرار گرفت. پژوهشگران به طرق مختلف سعی در دستیابی به نتایج بهتری در پیش‌بینی خطا هستند. در این بررسی مشخص شد که در پژوهش‌های پیشین دو دسته‌ی کلی از معیارها مورد استفاده قرار گرفته است. این دسته‌ها عبارتند از معیارهای کد و معیارهای فرآیند. معیارهای فرآیند دارای مزیت‌ها بیشتری نسبت به معیارهای کد هستند و پژوهش‌های کمتری نیز به بررسی آنها پرداخته است. در یکی از پژوهش‌های اخیر از معیارهای جهش  در کنار معیارهای کد به منظور پیش‌بینی‌خطا استفاده گردیده و موجب بهبود پیش‌بینی شده است. \\

پس از مشخص شدن بخش‌هایی از این حوزه که نیازمند تحقیق بیشتر هستند و شناسایی پتانسیل‌های موجود در معیارهای فرآیند و جهش در فصل \ref{chap:method} راهکارهایی ارائه شدند تا با استفاده از معیارهای فرآیند و مفاهیم تحلیل جهش پیش‌بینی خطا بهبود یابد. در رویکرد اول معیارهای فرآیند در کنار معیارهای جهش قرار می‌گیرند و پیش‌بینی خطا با استفاده از آنها انجام می‌پذیرد. در رویکرد دوم، چهار معیار فرآیند مبتنی بر مفاهیم تحلیل جهش ارائه شده‌اند و در رویکرد سوم دو معیار فرآیند با استفاده از مفاهیم جهش اصلاح شدند و معیارهای ترکیبی جهش-فرآیند به وجود آمدند. \\

در فصل \ref{chap:case-study}  نحوه‌ی پیاده‌سازی هر یک از سه رویکرد ارائه شده و ابزارهای مورد استفاده شرح داده شد. به منظور انجام مطالعه‌ی موردی، پنج پروژه‌ی صنعتی جاوا مورد استفاده قرار گرفتند و معیارهای مورد بررسی در آنها استخراج شد. این معیارها برای دو گروه از پرونده‌ها که یکی حاوی خطا و دیگری سالم هستند محاسبه شده است. در این دو گروه تعداد یکسانی پرونده وجود دارد. پرونده‌های حاوی خطا در مجموعه‌داده‌ی \lr{defects4j} مشخص شده‌اند و پرونده‌های سالم به طور تصادفی انتخاب شدند. \\

معیارهای استخراج شده در فصل \ref{chap:evaluation} ارزیابی شدند.  مدل‌های پیش‌بینی با استفاده از  چهار روش دسته‌بندی ساخته شدند و عملکرد مدل‌ها با یکدیگر مقایسه گردید. نتایج ارزیابی نشان داد که معیارهای جهش زمانی که در کنار معیارهای فرآیند قرار گیرند می‌توانند تاثیر قابل توجهی در بهبود پیش‌بینی داشته باشند.\\
 معیارهایی که تحت عنوان فرآیند مبتنی بر جهش ارائه شدند، زمانی که در کنار معیارهای فرآیند قرار می‌گیرند موجب بهبود پیش‌بینی خطا می‌شوند اما توانایی آنها بیشتر از معیارهای جهش نیست. از آنجا که این دسته از معیارها هزینه‌ی محاسباتی بیشتری دارند جایگزینی آنها با معیارهای جهش نمی‌تواند مزیتی داشته باشد. همچنین قرارگیر همه‌ی این معیارها در کنار هم نیز تاثیر مثبت چندانی نخواهد داشت.\\
 معیارهای ترکیبی جهش-فرآیند به طور میانگین ۶ درصد در صحت، $6.6$ درصد در دقت و $5.1$ در مساحت زیر نمودار ROC تغییر مثبت ایجاد کرده است و از نظر معیار بازخوانی تغییر قابل توجهی ایجاد نمی‌کند. این تغییرات نشان می‌دهد که اصلاح  معیارهای فرآیند موفق آمیز بوده است و عرصه‌ی جدیدی را می‌توان به منظور ساخت معیارهای پیش‌بینی در نظر گرفت و این عرصه ارائه‌ی معیارهای ترکیبی است. همچنین با توجه به این نکته  که تولید جهش‌یافته نیازمند وجود موارد آزمون نیست می‌توان برای این معیارها دامنه‌ی کاربرد وسیع‌تری در نظر گرفت. 

در ادامه به گام‌هایی اشاره می‌شود که می‌توانند  به نتایج این پایانامه   جامعیت بخشند شود و ابعاد دیگری از بکارگیری این معیارها  مورد بررسی قرار گیرد.

\begin{itemize}
	\item
	\textbf{بررسی تاثیر استفاده از عملگرهای متفاوت:}\\
	در این پایانامه مجموعه‌ی محدودی از عملگرها جهت ساخت جهش استفاده شده است. در پژوهش‌های آتی می‌توان به این موضوع پرداخت که افزایش و یا  کاهش مجموعه‌ی عملگرهای جهش‌یافته چه تاثیری بر پیش‌بینی خطا داشته باشد. همچنین اینکه کدام نوع از عملگرهای مورد استفاده در استخراج معیارهای ارائه شده تاثیر بیشتری بر پیش‌بینی خطا دارد. 
	\item
	\textbf{ارزیابی معیارهای کد در کنار معیارهای ارائه شده:}\\
همانطور که بیان شد معیارهای جهش می‌توانند به معیارهای فرآیند کمک کنند تا پیش‌بینی دقیقتری انجام شود. از طرف دیگر استفاده از معیارهای کد نیز می‌تواند به معیارهای جهش کمک کند و این معیارها هزینه‌ی محاسباتی کمتری دارند. با توجه به پر هزینه بودن معیارهای جهش لازم است میزان بهبود پیش‌بینی  خطا توسط آنها با معیارهای کد مقایسه شود و مشخص شود  در هنگام قرار گیری در کنار معیارهای فرآیند مزیتی در مقابل معیارهای کد دارند یا خیر. 
\item
\textbf{ساخت چهارچوب پیش‌بینی خطا با استفاده از پژوهش موجود:}\\
استخراج معیارها و ساخت مدل‌های پیش‌بینی در این پایانامه به صورت خودکار انجام می‌گیرد. با ایجاد تغییرات لازم می‌توان چهارچوبی ارائه داده که برای سایر پروژه‌های نرم‌افزاری نیز این معیارها را استخراج کند. با ایجاد یک چهارچوب هم انجام پژوهش‌های آتی توسط سایرین سهولت می‌یابد و هم زمینه‌ی به کارگیری پیش‌بینی خطا در صنعت توسعه می‌یابد. 
	
	
\end{itemize}





\appendix 
\addcontentsline{toc}{chapter}{پیوست‌ها}


\lstdefinestyle{appstyle}{
	backgroundcolor=\color{white},   
	commentstyle=\color{codegreen},
	keywordstyle=\color{magenta},
	numberstyle=\tiny\color{codegray},
	stringstyle=\color{codeblue},
	basicstyle=\footnotesize,
	breakatwhitespace=false,         
	breaklines=true,                 
	captionpos=b,                    
	keepspaces=true,                 
	numbers=left,                    
	numbersep=5pt,                  
	showspaces=false,                
	showstringspaces=false,
	showtabs=false,                  
	tabsize=2,
	basicstyle=\normalsize
}

\lstset{style=appstyle}

\chapter{ساخت مدل‌های پیش‌بینی و ارزیابی}

در این قسمت قطعه کدهای ساخت مدل‌های پیش‌بینی و ارزیابی آنها آورده شده است. قطعه کد \ref{code:data-set}  مجموعه‌داده‌ها را آماده می کند و تنظیمات مربوط به آموزش مدل‌ها را انجام می‌دهد. 

\begin{latin}
	\begin{lstlisting}[language=R]
library(RMySQL);
library(caret);
library(pROC);
library(e1071)

mydb = dbConnect(MySQL(), user='root', password='1', dbname='bug_predict', host='127.0.0.1');

rs_mutation_metric = dbSendQuery(mydb, "select * from MutationMetric");
mutation_metircs = fetch(rs_mutation_metric, n=-1);
rs_process_metric = dbSendQuery(mydb, "select * from ProcessMetric ");
process_metircs = fetch(rs_process_metric, n=-1);

##### clean up data #####
source("/home/ali/project/R-scripts/kill-live-to-score.R");

merged_metrics <- merge(x=clean_mutation_metircs,y=process_metircs, by.x="MetricId", by.y="ID")
lables<- as.factor(merged_metrics[,names(merged_metrics) %in% c("FILE_TYPE")]);


############test and train#############
b_number<-nrow(merged_metrics[merged_metrics$FILE_TYPE == "B",])
c_number<- nrow(merged_metrics[merged_metrics$FILE_TYPE == "C",])
smp_size_b <- floor(0.9 * b_number);
smp_size_c <- floor(0.9 * c_number);

## set the seed to make your partition reproducible
set.seed(1423)
train_ind_b <- sample(seq_len(b_number), size = smp_size_b)
train_ind_c <- sample(seq_len(c_number), size = smp_size_c)
train_ind_c <- train_ind_c + b_number
train_ind <- c(train_ind_b, train_ind_c)

#########train control##########
MyFolds <- createMultiFolds(merged_metrics[train_ind,4], k = 10, times=10)
train_control <- trainControl(method = "cv", index = MyFolds,
savePredictions = TRUE,
classProbs = TRUE
#    ,summaryFunction = twoClassSummary
)

	\end{lstlisting}
\end{latin}
\captionof{lstlisting}{آماده‌سازی مجموعه داده}
\label{code:data-set}
در قطعه کد \ref{code:clean} پاک‌سازی داده‌ها و تبدیل داده‌های جهش به امتیاز جهش انجام می‌شود.
\begin{latin}
	\begin{lstlisting}[language=R]
	clean_mutation_metircs<- mutation_metircs[!is.na(mutation_metircs$Covered),];
	
	for(i in 1:dim(clean_mutation_metircs)[1])
	{
	if(clean_mutation_metircs[i,'Lived']==-1)
	{
	clean_mutation_metircs[i,'Lived']<- 0;
	clean_mutation_metircs[i,'Killed']<-clean_mutation_metircs[i,'Killed']-1;
	}
	}
	
	temp<-clean_mutation_metircs;
	temp[,6]<-clean_mutation_metircs[,6]/clean_mutation_metircs[,5]
	temp[,7]<-clean_mutation_metircs[,6]/clean_mutation_metircs[,2]
	clean_mutation_metircs <- temp
	
	for(i in 1:dim(clean_mutation_metircs)[1])
	{
	if(is.nan(clean_mutation_metircs[i,'Lived']))
	clean_mutation_metircs[i,'Lived']<- 0;
	if(is.nan(clean_mutation_metircs[i,'Killed']))
	clean_mutation_metircs[i,'Killed']<- 0;
	}
	\end{lstlisting}
\end{latin}
\captionof{lstlisting}{تبدیل داده‌های جهش به امتیاز جهش }
\label{code:clean}
در قطعه کد \ref{code:modeling} مدل‌های پیش‌بینی ساخته می‌شوند و با استفاده از داده‌های آزمون پیش‌بینی انجام می‌گیرد. تمامی معیارها در متغیر merged\_metrics وجود دارند و با انتخاب ستون‌های مورد نظر زیر مجموعه‌ی مناسب انتخاب می‌شود. همچنین در تابع train روش دسته‌بندی انتخاب می‌گردد.

\begin{latin}
	\begin{lstlisting}[language=R]

############Process Metrics#############
p_features<-merged_metrics[,-c(seq(1,13),17,20)];
model1 <- train(p_features[train_ind,], lables[train_ind], trControl = train_control, method="nnet");
predict1_raw<-predict.train(model1, p_features[-train_ind,], type="raw")
predict1_prob<-predict.train(model1, p_features[-train_ind,], type="prob")


############Process Metrics with mutation#############

m_features1<-merged_metrics[,!names(merged_metrics) %in% c("FILE_TYPE","MetricId","FileType","FileInfoId","FILE_INFO_ID")];
m_features1<-m_features1[,-c(seq(5,10))];
model2 <- train(m_features1[train_ind,], lables[train_ind], trControl=train_control, method="nnet");
predict1_raw<-predict.train(model2, m_features1[-train_ind,], type="raw")
predict1_prob<-predict.train(model2, m_features1[-train_ind,], type="prob")


m_features2<-merged_metrics[,!names(merged_metrics) %in% c("FILE_TYPE","MetricId","FileType","FileInfoId","FILE_INFO_ID")];
m_features2<-m_features2[,-c(7,8)];
model3 <- train(m_features2[train_ind,], lables[train_ind], trControl=train_control, method="nnet");
predict2_raw<-predict.train(model3, m_features2[-train_ind,], type="raw")
predict2_prob<-predict.train(model3, m_features2[-train_ind,], type="prob")

	\end{lstlisting}
\end{latin}
\captionof{lstlisting}{ساخت مدل‌های پیش‌بینی}
\label{code:modeling}
در قطعه کد \ref{code:evluation} پیش‌بینی‌های انجام شده ارزیابی می‌شوند.

\begin{latin}
	\begin{lstlisting}[language=R]

auc(lables[-train_ind], predict1_prob$B)
auc(lables[-train_ind], predict2_prob$B)
auc(lables[-train_ind], predict3_prob$B)

plot(roc(lables[-train_ind], predict1_prob$B), col="blue", cex.lab=2, cex.axis=1.5);
plot(roc(lables[-train_ind], predict2_prob$B), add=TRUE, col="red", lty=2);
plot(roc(lables[-train_ind], predict3_prob$B), add=TRUE, col="black", lty=4);
 legend(x="bottomright",col=c("blue", "red", "black"),lwd=3,legend=c("Process","Process & Mutation", "Process & Mutation Base"),bty="6", lty=c(1,2,4), cex = 1.5)


confusionMatrix(predict1_raw, lables[-train_ind])
confusionMatrix(predict2_raw, lables[-train_ind])
confusionMatrix(predict3_raw, lables[-train_ind])

	\end{lstlisting}
\end{latin}
\captionof{lstlisting}{ساخت مدل‌های پیش‌بینی}
\label{code:evluation}

 
\chapter{آماده‌سازی رایانه به عنوان سرور}
\label{app:server}
انجام تحلیل جهش امری زمان‌بر است. به همین علت لازم است که رایانه‌ای به این فرآیند اختصاص یابد تا  این کار بدون وقفه انجام شود و رفع خطا در زمان توسعه‌ی کد در هر مکان و زمانی امکان پذیر باشد. در ادامه گام‌های لازم برای تبدیل رایانه به سرور آمده است. 

\section{تنظیمات پایگاه داده}
پایگاه‌داده‌ی  مورد استفاده در این پژوهش \lr{MySQL 5.7.22} می‌باشد. در ابتدا لازم است امکان برقراری ارتباط از راه دور توسط \نام{آی‌پی}{IP}‌های خارج از رایانه فرآهم شود. در فایل   mysqld.cnf پیکربندهای پایگاه داده وجود دارد و این فایل در آدرس 
\lr{/etc/mysql/mysql.conf.d}
قرار دارد. در این فایل لازم است که پارامتر bind-address با استفاده از \# به \نام{کامنت} {Comment}تبدیل شود. \\ 
سپس لازم است که یک کاربر مشخص شود که با هر آی‌پی بتواند به پایگاه‌داده وارد شود. این عمل می‌تواند با استفاده از نرم‌افزار  Workbench به سادگی انجام شود. این نرم‌افزار ابزار طراحی، توسعه و مدیریت پایگاه‌داده است. در قسمت server و سپس
\lr{ users and previlages}
  می‌توان به سادگی کاربر مورد نیاز را تعریف کرد. پس از انجام این تنظیمات لازم است که پایگاه داده راه‌اندازی مجدد شود. 
  
\section{ارتباط با اینترنت}
پیشنیاز اولیه هر سرور ارتباط با اینترنت می‌باشد. در برخی از شبکه‌ها برای برقراری این ارتباط لازم است از VPN مخصوص به آن شبکه استفاده شود. مشکلی که اغلب  یک vpn  دارد قطع شدن ارتباط آن است و لازم است این ارتباط پس از قطع دوباره ایجاد شود. قطعه کد  \ref{code:vpn} هر ۳۰ ثانیه ارتباط را چک می‌کند و در صورت قطع vpn را مجددا راه اندازی می‌کند. 




\begin{latin}
	\begin{lstlisting}[language=bash]
#!/bin/bash +x
while [ "true" ]
	do
	CON="Sharif-ID2"
	STATUS=`nmcli con show --active | grep $CON | cut -f1 -d " "`
	if [ -z "$STATUS" ]; then
		echo "Disconnected, trying to reconnect..."
		(sleep 1s && nmcli con up $CON)
	else
		echo "Already connected !"
	fi
	sleep 30
done
\end{lstlisting}
\end{latin}
\captionof{lstlisting}{راه اندازی مجدد vpn}
\label{code:vpn}

\section{
	رفع مشکل آی‌پی \نام{پویا}{Dynamic}
}

برای ارتباط با هر رایانه از راه دور لازم است که آدرس آن رایانه را داشته باشیم که این آدرس همان آی‌پی می‌باشد. در بسیاری از شبکه‌ها این آدرس  به دلایل مختلف ثابت نیست. به منظور حل این مشکل  می‌توان از سرویس‌هایی استفاده کرد که امکان \واژه{انقیاد} آی‌پی به آدرس URL را فراهم می‌کنند و همواره با تغییر آی‌پی، آدرس URL را به آی‌پی جدید متصل می‌کنند. سرویسی که در این پژوهش استفاده شد متعلق به سایت
\نام{noip }{\url{www.noip.com}} بود. این سایت یک برنامه هم فراهم می‌کند که این برنامه بر روی رایانه‌ی مورد نظر نصب می‌شود و در بازه‌های زمانی مشخص آدرس آی‌پی را برای سرویس ارسال می‌کند و انقیاد آدرس انجام می‌شود. 
\section{ارتباط با ترمینال}
ترمینال این امکان را فراهم می‌سازد تا تمام عملیات‌های ممکن در یک سیستم عامل از طریق آن انجام شود. یکی از راههای متدوال و مطمئن  ارتباط از راه دور استفاده از پروتکل ssh می‌باشد. برای استفاده از این پروتکل لازم است یک سرویس SSH بر روی سرور راه اندازی شود یکی از نرم‌افزارهایی که این کار را انجام می‌دهد OpenSSH می‌باشد. نسخه‌ی Client از این ابزار نیز بر روی رایانه‌ی مشتری نصب می‌گردد. کاربران مجاز می‌توانند از طریق گذرواژه  مشخص شده با سرویس   SSH بر روی سرور ارتباط برقرار کنند. اما راه ایمن‌تر شناسایی از طریق کلید است. کاربر یک کلید عمومی و خصوصی تولید می‌کند. کلید عمومی در نزد  سرور نگهداری می‌شود و کلید خصوصی در نزد کاربر و برای برقراری ارتباط از این کلید‌ها استفاده می‌شود.\\

در ارتباط SSH یک ترمینال برای کاربر ایجاد می‌شود که برای اجرای پروژه در پس‌زمینه کافی نیست. زیرا با قطع ارتباط اجرا نیز متوقف می‌شود. برای دسترسی به چند ترمینال از طریق یک ترمینال می‌توان از ابزار Screen استفاده نمود. پس از برقراری ارتباط SSH در ترمینال باز شده ترمینال‌های مختلفی از طریق ابزار Screen می‌توان ایجاد و مدیریت کرد. هر یک از این ترمینال‌ها در پس زمینه می‌توانند کار خود را بدون توجه به وجود یا عدم وجود ارتباط SSH ادامه دهند. 
\section{ساخت و اجرای پروژه‌ی جاوا}
برای \واژه{ساخت} و اجرای یک پروژه‌ی جاوا به صورت خودکار ابزارهای مختلفی وجود دارد. یکی از ابزارهای مناسب که به آن اشاره شد Maven است. لازم است که پیکربندی‌های مناسب جهت استفاده از وابستگی‌ها، کامپایل و ساخت فایل اجرایی jar  در فایل pom.xml انجام شود. این تنظیمات در قطعه کد \ref{code:pom} آمده است. سپس با استفاده از قطعه کد \ref{code:run} پروژه ساخته و اجرا می‌شود. از آنجا که پروژه به منظور کامپایل به نسخه‌ی ۸ جاوا کامپایلر نیاز دارد و در هنگام اجرا به نسخه‌ی ۷، این تنظمات به صورت خودکار انجام می‌شود. همچنین لازم نیست که با تغییر کد بر روی رایانه‌ی \ترجمه{مشتری}{Client}  کدها مستقیما بر به سرور منتقل شوند. کافیست از سیستم کنترل نسخه استفاده شود. کدها  با استفاده از ابزار گیت در سیستم کنترل نسخه‌ی بر روی \ترجمه{ابر}{Cloud} آپلود شود و سپس توسط همین ابزار در سرور دانلود شود. اصطلاحا به این عمل Pull و Push می‌گویند.
\definecolor{dkgreen}{rgb}{0,0.6,0}
\definecolor{gray}{rgb}{0.5,0.5,0.5}
\definecolor{mauve}{rgb}{0.58,0,0.82}
\definecolor{gray}{rgb}{0.4,0.4,0.4}
\definecolor{darkblue}{rgb}{0.0,0.0,0.6}
\definecolor{lightblue}{rgb}{0.0,0.0,0.9}
\definecolor{cyan}{rgb}{0.0,0.6,0.6}
\definecolor{darkred}{rgb}{0.6,0.0,0.0}

\lstdefinelanguage{x}
{
	morestring=[s][\color{mauve}]{"}{"},
	morestring=[s][\color{black}]{>}{<},
	morecomment=[s]{<?}{?>},
	morecomment=[s][\color{dkgreen}]{<!--}{-->},
	stringstyle=\color{black},
	identifierstyle=\color{lightblue},
	keywordstyle=\color{red},
	morekeywords={xmlns,xsi,noNamespaceSchemaLocation,type,id,x,y,source,target,version,tool,transRef,roleRef,objective,eventually}% list your attributes here
}
\lstset{
	basicstyle=\ttfamily\footnotesize,
	columns=fullflexible,
	showstringspaces=false,
	numbers=left,                   % where to put the line-numbers
	numberstyle=\tiny\color{gray},  % the style that is used for the line-numbers
	stepnumber=1,
	numbersep=5pt,                  % how far the line-numbers are from the code
	backgroundcolor=\color{white},      % choose the background color. You must add \usepackage{color}
	showspaces=false,               % show spaces adding particular underscores
	showstringspaces=false,         % underline spaces within strings
	showtabs=false,                 % show tabs within strings adding particular underscores
	frame=none,                   % adds a frame around the code
	rulecolor=\color{black},        % if not set, the frame-color may be changed on line-breaks within not-black text (e.g. commens (green here))
	tabsize=2,                      % sets default tabsize to 2 spaces
	captionpos=b,                   % sets the caption-position to bottom
	breaklines=true,                % sets automatic line breaking
	breakatwhitespace=false,        % sets if automatic breaks should only happen at whitespace
	title=\lstname,                   % show the filename of files included with \lstinputlisting;
	% also try caption instead of title  
	commentstyle=\color{gray}\upshape
}

\begin{latin}
	\begin{lstlisting}[language=x]
   <?xml version="1.0" encoding="UTF-8"?>
   <build>
   <plugins>
   <plugin>
   <groupId>org.apache.maven.plugins</groupId>
   <artifactId>maven-compiler-plugin</artifactId>
   <configuration>
   <source>1.8</source>
   <target>1.8</target>
   </configuration>
   </plugin>
   <plugin>
   <!-- Build an executable JAR -->
   <groupId>org.apache.maven.plugins</groupId>
   <artifactId>maven-jar-plugin</artifactId>
   <version>3.1.0</version>
   <configuration>
   <archive>
   <manifest>
   <addClasspath>true</addClasspath>
   <classpathPrefix>lib/</classpathPrefix>
   <mainClass>Main</mainClass>
   </manifest>
   </archive>
   </configuration>
   </plugin>
   <plugin>
   <artifactId>maven-assembly-plugin</artifactId>
   <configuration>
   <archive>
   <manifest>
   <mainClass>Main</mainClass>
   </manifest>
   </archive>
   <descriptorRefs>
   <descriptorRef>jar-with-dependencies</descriptorRef>
   </descriptorRefs>
   </configuration>
   </plugin>
   </plugins>
   </build>
\end{lstlisting}
\end{latin}
\captionof{lstlisting}{پیکربندی pom.xml}
\label{code:pom}

\lstset{style=appstyle}
\begin{latin}
\begin{lstlisting}[language=bash]
#!/bin/bash +x
cd ~/IdeaProjects/bug_predict
echo 2 | sudo update-alternatives --config javac
mvn clean compile assembly:single
echo 1 | sudo update-alternatives --config javac
/usr/lib/jvm/java-8-openjdk-amd64/bin/java -jar target/com.bug.predict-1.0-SNAPSHOT-jar-with-dependencies.jar
\end{lstlisting}
\end{latin}
\captionof{lstlisting}{ساخت و اجرای پروژه}
\label{code:run}


\chapter{معیارهای استخراج شده}

	\input{appendix/table-met}







\PrepareForBibliography

\setlatintextfont[Scale=1]{Linux Libertine}
\setlength{\baselineskip}{0.8cm}
%\setromantextfont[Scale=1.2]{XB Niloofar}

%\bibliographystyle{IEEEtran}
%\bibliographystyle{is-unsrt}
%\bibliographystyle{ieeetr-fa}
%\bibliographystyle{amsplain}

%\bibliography{resources/resources}
\latin
\printbibliography[title=\bibliographytitle,heading=bibintoc]
\persian

\printglossary
\PrepareForLatinPages
\date{August 2018}
\logo{\includegraphics[scale=.4]{logo-en}}
\title{\sffamily\enTitle}
% uncomment following lines only if you have defined commands for two-lines-title at the beginning of this file
%\titlelineone{\enTitleLineOne}
%\titlelinetwo{\enTitleLineTwo}
\author{\sffamily\enAuthor}
\university{\normalfont\bfseries Sharif University of Technology\\Computer Engineering Department}
\subject{Software Engineering}
\supervisor{\sffamily Dr.  Hassan Mirian}
%If you have a consultant/advisor professor, uncomment the following line
%\consult{\sffamily Dr. <name of consultant prof.>}
\begin{abstract}{\enKeywords}
%latin abstract
% write it at the end...

Software developers notice existance of  faults by report of a fault in issue  tracking systems  or a failure in software tests. Then they try to locate the bug and understand the problem. Early detection of dault results in saving time and money and faciliates debugging process. Prediction models can be built and used easly by modern statestical tools. Software metrics are the most important part of prediction models. Therfore higher perfomance in models can be achived using new and effective metrics. In this study, process metrics and metrics that built base on mutation analysis used and resulting models evaluted. In addition to using process metrics with mutation metrics, two group of metrics named \textit{mutation base process}  metrics and \textit{mutation-process hybrid} introduced for building prediction models. Results showed that  mutation metrics can improve prediction prefromance of process metrics. Although mutation based process metrics have a predective value, they can not perform better than mutation metrics. Also mutation-process hybrid metrics can improve performance in prediction models significantly. 
\end{abstract}
\makethesistitle
\پایان{نوشتار}
