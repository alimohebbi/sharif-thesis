\chapter{سرآغاز}
\برچسب{chap:intro}
سامانه‌های نرم‌افزاری بسیار فراگیر شده‌اند و زندگی امروزی را ارتقا داده‌اند. در نتیجه کاربران کیفیت نرم‌افزار بالایی را تقاضا می‌کنند. کشف و بر طرف کردن خطاها پرهزینه است و مدل‌های پیش‌بینی خطا  از طریق اولویت‌دهی به فعالیت‌های تضمین کیفیت موجب افزایش بازدهی می‌گردند. پیش‌بینی خطا از سال ۱۹۹۲ تا کنون یک زمینه‌ی فعال تحقیقاتی بوده است. محققان همواره به دنبال روش‌هایی بوده‌اند که پیش‌بینی خطا را با کیفیت بهتری انجام دهند و یا دامنه‌ی کاربرد آن را گسترش بخشند. 

به  منظور افزایش کارایی پیش‌بینی‌خطا محققان معیارهای نوینی را ارائه داده‌اند \cite{bacchelli2010popular}، سعی داشته‌اند محدودیت‌های یادگیری ماشین را تقلیل بخشند \cite{limsettho2018cross} و یا روش‌های بروزتری را به منظور \واژه{دسته‌بندی} به کار گیرند\cite{chen2016software}. 

\section{ تعاریف مقدماتی}
در این قسمت چند اصطلاح رایج در مبحث پیش‌بینی‌ خطا و مورد استفاده در پایانامه نوشته شده است.
\begin{itemize}
	\item 
	\واژه{مورد آزمون}: \\
	یک مورد آزمون متشکل است از مقادیر مورد آزمون، نتایج مورد انتظار که با اجرای برنامه تحت آزمون یک یا چند عملکرد آنرا ارزیابی می‌کند. 
	\item
	\واژه{سامانه‌ی کنترل نسخه}:\\
	این سامانه تغییرات اعمال شده بر روی یک یا چندین \واژه{پرونده} را ذخیره می‌کند تا در آینده بتوان یک نسخه‌ی خاص را بازخوانی کرد. 
	\item
	\واژه{ثبت}:\\
	ذخیره‌ی تغییرات ایجاد شده بر روی پرونده‌ها در سامانه‌ی کنترل نسخه‌ را ثبت می‌نامند. یک ثبت را می‌تواند معادل یک نسخه از برنامه در نظر گرفت که البته این نسخه می‌تواند ناکامل باشد.
	\item
	\واژه{انتشار}:\\
	انتشار به معنی توزیع نسخه‌ی نهایی یک نرم‌افزار است که قابل استفاده برای کاربر می‌باشد. یک انتشار ممکن است نسخه‌ای از یک برنامه‌ی جدید باشد و یا ارتقاء یافته‌ی نرم‌افزار موجود باشد. قبل از یک انتشار معمولا به ترتیب نسخه‌های \نام{آلفا}{Alpha} و \نام{بتا}{Beta} توزیع می‌شود. 
\end{itemize}

\section{بیان مسئله}
آزمون نرم‌افزار اصلی‌ترین فعالیت تیم  تضمین کیفیت می‌باشد. آزمون نرم‌افزار می‌تواند تا ۵۰ درصد هزینه‌ی تولید نرم‌افزار را به خود اختصاص دهد. هدف از پیش‌بینی خطا افزایش بازدهی این فرآیند می‌باشد. حال با بهبود پیش‌بینی خطا می‌توان به دستیابی به این هدف کمک نمود. به منظور پیش‌بینی خطا معیارهایی  در سطح مورد نظر استخراج می‌گردد و با استفاده از دسته‌بندی، خطادار بودن یا نبودن قطعه‌ی مورد بررسی پیش‌بینی می‌شود. یک دسته از معیارهای مورد استفاده در این زمینه معیارهای فرآیند است و معیارهای جهش نیز به تازگی در این راستا استفاده شده‌اند. این پژوهش قصد دارد تا بررسی کند که معیارهای جهش در کنار فرآیند  چه میزان در پیش‌بینی خطا تاثیر گذار است و همچنین بر اساس مفاهیم تحلیل جهش معیارهای جدیدی ارائه دهد تا پیش‌بینی خطا بهبود یابد. \\

با توجه به اینکه معیارهای جهش به تازگی در پیش‌بینی خطا مورد استفاده قرار گرفته‌اند لازم است تا تحقیقات بیشتری در مورد آنها صورت گیرد و عملکرد آنها از ابعاد مختلف مورد بررسی قرار گیرد. همچنین با بررسی مطالعات پیشین نقاط ضعف و قوت معیارهایی که تا کنون ارائه شده‌اند مورد بررسی قرار گرفتند. در این پژوهش عملکرد معیارهای فرآیند و جهش مورد بررسی بیشتری قرار گرفتند و با توجه به نقاط ضعف و قدرت معیارهای قبلی، معیارهایی ارائه شده تا بخشی از نقاط ضعف را پوشش دهند و به پیش‌بینی بهتری بیانجامند. 
\section{ساختار پایانامه}

این پایانامه در ۶ فصل تهیه گردیده است. در فصل \ref{chap:survey} به مرور مطالعات پیشین پرداخته می‌شود که در  قسمت \ref{sec:bug-predict}  مباحث مربوط به پیش‌بینی خطا از جمله فرآیند پیش‌بینی، معیارهای ارزیابی، معیارهای پیش‌بینی و مدل‌های پیش‌بینی بررسی می‌شوند. در قسمت \ref{sec:mutation} مباحث مربوط به آزمون جهش بررسی شده‌اند و در قسمت \ref{sec:conclustion} مطالعات مروری جمع‌بندی شده‌اند. در فصل \ref{chap:method} معیارهای مورد استفاده و ارائه شده در این پژوهش معرفی می‌شوند. در فصل \ref{chap:case-study} پنج پروژه‌ی صنعتی مورد مطالعه قرار گرفته‌اند و در فصل \ref{chap:evaluation} معیارها مورد ارزیابی قرار گرفته‌اند. در فصل \ref{chap:future} مباحث مطرح شده در این پایانامه جمع‌بندی شده و کارهای آتی شرح داده شده است. 
	
