\chapter{مورد مطالعاتی}
\label{chap:case-study}
در این فصل مطالعه‌ی موردی بر روی مجموعه‌داده‌ی \lr{defects4j} \cite{Just:2014:DDE:2610384.2628055} انجام می‌گیرد. ابتدا نحوه‌ی کلی برپایی آزمایش شرح داده می‌شود و سپس چگونگی استخراج معیارها و پیاده‌سازی ارمایش توضیح داده خواهد شد. 
\section{طراحی آزمایش}
به منظور ارزیابی رویکردهای گفته شده لازم است که برای مجموعه معیارهای هر رویکرد مدلهای پیش‌بینی ساخته شود و هر عملکرد هر مدل نسبت به پژوهش‌های گذشته مقایسه شود. به این ترتیب ابتدا لازم است از مجموعه‌داده‌ی فراهم شده معیارهای بیان شده در فصل \ref{sec:method} استخراج شوند. مجموعه‌داده‌ی \lr{defects4j} که در قسمت‌های آتی معرفی می‌شود شامل اطلاعات خطا در چندین فایل است و به همین تعداد، فایل بدون خطا در ثبت و پروژه‌ی متناظر به طور تصادفی انتخاب می‌گردد. برای فایلهای حاوی خطا و سالم، معیارها استخراج می‌شود. معیارهای استخراج شده  برای هر فایل به عنوان بردار ویژگی در مدلهای دسته‌بندی عمل می‌کند. مدلهای دسته‌بندی به منظور پیش‌بینی حاوی خطا بودن ساخته می‌شود و عملکرد آنها مقایسه می‌گردد. مدلهایی که با هم مقایسه می‌شود در الگوریتم و \واژه{پیکربندی} یکسان هستند و تنها تفاوت آنها در معیارهای استفاده شده به منظور یادگیری است. بدین ترتیب تاثیر معیارها بر پیش‌بینی خطا سنجیده می‌شود. 

\section{ آشنایی با ابزارها و مجموعه داده}
این قسمت به معرفی ابزارهای استفاده شده در این پژوهش می‌پردازد. آشنایی با این ابزارها به درک هرچه بهتر  نحوه‌ی استخراج معیارها  و روند آزمایش کمک می‌کند.

\subsection{مجموعه داده \lr{defect4j}}
 مجموعه‌‌داده‌ی انتخابی به منظور انجام مورد مطالعاتی لازم است که دارای ویژگی‌های زیر باشد:
 \begin{itemize}
 	\item
 	اطلاعات خطاهای پروژه وجود داشته باشد و این اطلاعات نشان دهد که خطا متعلق به کدام پرونده در کدام ثبت است. 
 	\item
 	پروژه‌ها متن-باز باشد تا بتوان با استفاده از \واژه{کد منبع} آنها معیارها را استخراج نمود.
 	\item
 	برای پروژه‌ها موارد آزمون مناسب وجود داشته باشد تا بتوان معیارهای جهش را استخراج کرد.
 \end{itemize}
 در میان مجموعه‌داده‌های موجود  مانند 
 Promise \cite{boetticher2007promise} و BDP
 \cite{d2012evaluating}
  مجموعه‌داده‌ی \lr{defects4j} تنها موردی است که تمام ویژگی‌ها را دارد.\\
 
این مجموعه شامل  شش پروژه می‌باشد که این پروژه‌ها  \واژه{متن-باز} هستند و با استفاده از نرم‌افزارهای کنترل نسخه‌ی گیت و svn می‌توان به کدهای آن‌ها در طول فرآیند توسعه‌ی آنها دسترسی پیدا کرد. بجز پروژه‌ی Chart سایرین از سیستم گیت استفاده می‌کنند. همچنین این پروژه به دلیل نداشتن ساختار مناسب کنار گذاشته شد و از پرونده‌های حاوی خطای  موجود در آن استفاده نشد. 
مجموعه‌داده‌ی \چر{defects4j} به صورت یک \واژه{چارچوب} ارائه شده است که کارهایی بیش از نگهداری اطلاعات درباره‌ی پروژه‌ها انجام می‌دهد. مهم‌ترین  کارهایی که می‌توان به وسیله‌ی این ابزار انجام داد در جدول \ref{tab:defects4j-ops} آمده است. 


\begin{table}[H] 
	\renewcommand*{\arraystretch}{1.3}	
	\centering \caption{عملیات موجود در \چر{defects4j}  }
	\label{tab:defects4j-ops}
	\newcolumntype{C}{>{\centering\arraybackslash} m } 
	\begin{tabular}{ |c|c|}
		
		\hline
		\hline
		نام عمل  & توضیح
		\\
		\hline
		\hline
		\lr{info } &   نمایش پیکربندی یک پروژه‌ی خاص یا خلاصه‌ی یک خطای خاص
		\\
		\hline
		\lr{checkout} &   وارسی یک نسخه‌ی حاوی خطا یا تعمیر شده از پروژه
		\\
		\hline
		\lr{compile} &   کامپایل کدها و آزمون‌های نوشته شده توسط توسعه‌دهندگان
		\\
		\hline
		\lr{test} &   اجرای یک آزمون یا مجموعه‌ی آزمون در یک نسخه‌ی حاوی خطا یا تعمیر شده از پروژه
		\\
		\hline
		\lr{mutation} &   اجرای تحلیل جهش در یک نسخه‌ی حاوی خطا یا تعمیر شده از پروژه
		\\
		\hline
		
	\end{tabular}
\end{table}

این ابزار در اجرای عملیات بالا دارای محدودیت است و تنها آن‌ها را بر روی ثبت‌های از پیش تعیین شده انجام می‌دهد. ثبت‌های از پیش تعیین شده شامل ثبت‌های حاوی خطا و تعمیر آن خطا می‌باشد. در جدول \ref{tab:defects4j-bugs} اطلاعات مربوط به تعداد خطاهای هر پروژه آمده است. 

\begin{table}[H] 
	\renewcommand*{\arraystretch}{1.3}	
	\centering \caption{پروژه‌های موجود در \چر{defects4j}  }
	\label{tab:defects4j-bugs}
	\newcolumntype{C}{>{\centering\arraybackslash} m } 
	\begin{tabular}{ |c|c|c|}
		
		\hline
		\hline
	نام مختصر &	نام کامل  & تعداد خطا
		\\
		\hline
		\hline
		\lr{Chart } & JFreeChart &	26
		\\
		\hline
		\lr{Closure} & \lr{Closure compiler}	& 133
		\\
		\hline
		\lr{Lang} &   \lr{Apache commons-lang} &	65
		\\
		\hline
		\lr{Math} &  \lr{Apache commons-math} &	106
		\\
		\hline
		\lr{Mockito} &   Mockito &	38
		\\
		\hline
		Time & Joda-Time &	27
		\\
		\hline
	-	& کل  پروژه‌ها &   395
		\\
		\hline
		
	\end{tabular}
\end{table}


به منظور نصب و راه اندازی ابزار  \lr{defects4j} ابتدا از صفحه‌ی \نام{گیت‌هاب}{Github} آن  کدهای مربوطه دریافت می‌شود. سپس باید  دستوراتی را اجرا کرد تا سایر متعلقات دریافت شود. این تعلقات شامل مخزن نرم‌افزاری مربوط به شش پروژه‌ی یاد شده است که کدهای پروژه‌ها در آن قرار دارد. نکته‌ی قابل توجه در این ابزار این است  که بجز  دستور info سایر دستورات عملیات مربوط را بر روی کامپیوتر کاربر انجام می‌دهد و خروجی را نمایش  داده می‌شود، نه اینکه از یک پایگاه داده اطلاعات را صرفاً بازخوانی کند. \\
در نیازمندی‌های این ابزار اشاره شده که باید از جاوا نسخه‌ی ۷ استفاده شود. اما مسأله‌ای که به آن اشاره نشده توزیع‌کننده‌ی جاوا است. جاوا دو توزیع‌کننده‌ی عمده دارد. یکی OpenJDK و دیگری Oracle می‌باشد. با استفاده از OpenJDK ابزار \lr{defects4j} و ابزارهایی که به آن وابسته است به خوبی کار نمی‌کنند. به عنوان مثال برخی مجموعه آزمون‌ها که باید بدون خطا اجرا شوند به دلیل نبود  \واژه[وابستگی‌های]{وابستگی} لازم با شکست مواجه می شوند. در این پایانامه از توزیع‌کننده‌ی Oracle استفاده شده است.\\ 
راه ارتباط با  ابزار defects4j \واژه{خط دستور}‌ می‌باشد و  یک نمونه‌  از دستورات قابل استفاده در این ابزار  در شکل  \ref{fig:d4j-info-command} است که این دستور اطلاعات مربوط به پروژه‌ی Lang و خطای شماره‌ی یک را خواهد داد. 
\begin{latin}
	\flushleft
defects4j info -p Lang -b 1
\end{latin}

\begin{figure}[H]
	\centering
	\includegraphics[width=.8\textwidth]{img/case_study/d4j-info-commadn.png}
	\caption{اجرای دستور info در \lr{defects4j}}
	\label{fig:d4j-info-command}
\end{figure}

\subsection{ابزار Major}
\label{sec:tools-major}

این ابزار جهت تولید جهش‌یافته و تحلیل جهش استفاده می‌شود. یک ابزار دیگر در این حوزه 
\نام{PIT}{ \url{http://pitest.org/}} می‌باشد اما به دلیل سازگاری ابزار \lr{defects4j} با Major و نیز قابلیت‌های ویژه‌ی آن از این ابزار استفاده شد.
چند مورد از ویژگی‌های مهم ابزار Major عبارتند از :
\begin{itemize}

\item
 راحتی استفاده  به دلیل نیاز به دستورات کمتر نسبت به PIT
\item
امکان اجرای تحلیل جهش در پروژه‌هایی که از ابزار \واژه{ساخت} گریدل استفاده می‌کنند
\item
 مجموعه عملگرهای کاملتر
\item
انعطاف در پیکربندی : امکان انجام تحلیل تنها برای یک کلاس یا تابع، تنظیمات ساده و کامل جهت مشخص کردن مجموعه عملگرها
\end{itemize}
لازم به ذکر است که این ابزار از کامپایلر‌ مخصوص به خود جهت کامپایل برنامه و ساخت جهش‌یافته استفاده می‌کند که گسترش یافته‌ی یک کامپایلر جاوا است. 
استفاده از این ابزار را می‌توان در سه مرحله خلاصه کرد :
\begin{enumerate}
\item
\textbf{پیکربندی تولید جهش‌یافته به وسیله‌ی دستورات \lr{MML}: }
این ابزار برای مشخص نمودن اینکه از چه عملگرهایی استفاده شود و آن‌ها در چه محل‌هایی از برنامه به کار گرفته شوند یک زبان ساده ابداع کرده است به نام MML که یک کامپایلر نیز دارد. ابتدا کد MML نوشته می‌شود و سپس با MMLC کامپایل می‌شود و نتیجه به عنوان یکی از پارامترها  در هنگام فراخوانی به ابزار  Major ارسال می شود.

نمونه‌ای از این کد در شکل \ref{fig:major-mml} آمده است. در بلاک targetOp مجموعه‌ی عملگرها مشخص می‌شود و در خط آخر محلی که عملگرهای جهش مشخص شده عمل می‌کند. در واقع تنها برای محل مشخص شده جهش‌یافته تولید می‌شود. در بلاک targetOP ابتدا مشخص می‌شود که عملگرهای باینری موجود در برنامه به چه عملگرهای دیگری تبدیل شود. همچنین امکان مشخص کردن نحوه‌ی تبدیل عملگرهای یگانی با دستو UNI به جای BIN وجود دارد. سپس عباراتی که امکان حذف آنها وجود دارد مشخص می‌شود که در شکل امکان حدف خروجی توابع مشخص شده است. در قسمت آخر مجموعه عملگرهای جهش قابل استفاده مشخص شده که در شکل عملگرهای حذف عبارت، جایگزینی عملگر شرطی و جایگزینی عملگر رابطه‌ای به کار گرفته شده است. مجموعه‌ی عملگرهای موجود در این ابزار در \cite{just2014major} آمده است. در خط آخر بیان شده که این عملگرها به نحو مشخص شده در targetOp بر روی تابعی به نام classify که سه آرگومان از نوع int دارد و در کلاس Triangle و پکیج trinangle قرار دارد اعمال شوند.

\begin{figure}[H]
	\centering
	\includegraphics[width=.8\textwidth]{img/case_study/major-mml.png}
	\caption{نمونه کد MML در Major}
	\label{fig:major-mml}
\end{figure}
\item 	
\textbf{تولید جهش‌یافته‌ها:}
 همانطور که اشاره شد ابزار Major جهت تولید نسخ جهش‌یافته نیاز به کامپایل پروژه دارد. امروزه پروژه‌های نرم‌افزاری از جمله پروژه‌های موجود در \lr{defects4j} از ابزارهایی استفاده می‌کنند که فرآیند ساخت را خودکارسازی می‌کنند. فرآیند ساخت به طور کلی شامل مراحل زیر است:
\begin{itemize}
	\item پاک سازی  پوشه‌های کاری، از پرونده‌های ساخت‌های قبلی
	\item 
	معرفی‌ وابستگی‌ها و کامپال پروژه
	\item
	معرفی وابستگی‌ها و کامپایل موارد آزمون
	\item
	اجرای موارد آزمون و ارائه‌ی گزارش
\end{itemize}
سه نوع از مهم ترین ابزارهای خودکارسازی مورد استفاده در صنعت عبارتند از Ant ،  Maven و \lr{Gradle}. در پروژه‌های مورد مطالعه نیز این سه نوع به کار گرفته شده است.  
هر یک از روش‌ها خودکارسازی دارای دستورات مربوط به خود می‌باشد  و برای تولید جهش‌یافته باید متناسب با آنها عمل نمود که در زیر خلاصه شده است. 
\begin{itemize}
	\item
	Ant : 
	این دسته از پروژه‌ها دارای یک پرونده به نام build.xml است که دستورات لازم جهت پیکربندی و انجام عملیات ساخت در آن قرار دارد. به منظور تولید جهش‌یافته کافیست کامپایلر مورد استفاده در قسمت کامپایل پروژه را کامپایلر توسعه‌یافته‌ی Major قرار داد و پارامترهای لازم به آن ارسال شود. 
	\item Maven : 
	در این دسته از پروژه‌ها دستورات لازم در پرونده‌ی pom.xml قرار دارد.  به وسیله‌ی  یک \واژه{افزونه} در ابزار Maven این پرونده تبدیل به یک پرونده build.xml می‌شود که قابل استفاده توسط ابزار Ant است. پس از این تبدیل مشابه حالت قبل عمل می‌شود. 	
	\item Gradle : 
	در این دسته از پروژه‌ها دستورات لازم در پرونده‌ی  build.gradle قرار دارد.  به منظور تولید جهش‌یافته کافیست کامپایلر مورد استفاده در قسمت کامپایل پروژه را کامپایلر توسعه‌یافته‌ی Major قرار داد و پارامترهای لازم به آن ارسال شود. 
\end{itemize}

نمونه‌ای از حاصل اجرای  عملیات جهش برای یک پرونده در شکل \ref{fig:major-mutant}  آمده است   که نشان می‌دهد ۸۶ جهش یافته تولید شده است.  همچنین ابزار یک پرونده به نام mutants.log تولید می‌کند که نشان می‌دهد چه جهش‌یافته‌هایی در کجا تولید شده‌اند. نمونه‌ای از محتویات این پرونده در شکل \ref{fig:major-log}  آمده است. 

\begin{figure}[H]
	\centering
	\includegraphics[width=.8\textwidth]{img/case_study/major-mutant.png}
	\caption{اجرای عملیات جهش برای یک پرونده}
	\label{fig:major-mutant}
\end{figure}

\begin{figure}[H]
	\centering
	\includegraphics[width=\textwidth]{img/case_study/major-log.png}
	\caption{نمونه‌ای از پرونده‌ی mutants.log}
	\label{fig:major-log}
\end{figure}

\item
\textbf{ اجرای تحلیل جهش:}
 ابتدا پرونده‌های آزمون کامپایل می‌شود و سپس هر مجموعه آزمون بر روی جهش‌یافته‌هایی که تاکنون کشته نشده‌اند اجرا می‌شود. در پایان نتایج  در خروجی چاپ می‌شوند. همچنین نتایج  در پرونده‌های با پسوند csv قرار می‌گیرد. نمونه‌ای از اجرای تحلیل جهش در شکل \ref{fig:major-analysis} و پرونده‌ی نتایج خروجی در شکل \ref{fig:major-results} نشان داده شده.
 
 \begin{figure}[H]
 	\centering
 	\includegraphics[width=.8\textwidth]{img/case_study/major-analysis.png}
 	\caption{اجرای تحلیل جهش}
 	\label{fig:major-analysis}
 \end{figure}
 
 \begin{figure}[H]
 	\centering
 	\includegraphics[width=.8\textwidth]{img/case_study/major-results.png}
 	\caption{نتایج خروجی تحلیل جهش}
 	\label{fig:major-results}
 \end{figure}

\end{enumerate}


\subsection{کتابخانه‌ی Jgit}
این کتابخانه جهت کار با مخازن نرم‌افزاری  که از نوع گیت هستند به کار گرفته می‌شود و به زبان جاوا است. تمام عملیات مهم و اساسی که در نرم‌افزار اصلی گیت وجود دارد در این کتابخانه نیز قابل انجام است. مشکلی که کار با این کتابخانه دارد نبود منابع آموزشی به اندازه‌ی کافی است. چراکه کاربران زیادی ندارد و آموزش‌های ابتدایی معمولاً نیازهای عموم کاربران را بر طرف می‌کند. 
\subsection{چارچوب Hibernate }
به وسیله‌ی این چارچوب می‌توان  اشیاء موجود در برنامه‌ی جاوا را به داده‌های موجود در پایگاه داده تبدیل کرد. اصطلاحاً به این ابزار ها \نام{ORM}{Object Relational Mapping} می‌گویند. مزیت استفاده از این نوع از ابزارها این است که ارتباط با پایگاه داده ساده‌تر خواهد شد و حجم کدهای لازم کاهش چشم‌گیری خوهد داشت. همچنین این ابزارها اشیاء را در حافظه‌ی موقت مدیریت می‌کنند و حجم کاری پایگاه داده کاسته می‌شود. 
\section{نکات پیاده‌سازی پروژه}

پیاده‌سازی پروژه در زبان جاوا انجام گرفت. یکی از نکات مهم و قابل توجه در پیاده‌سازی این پروژه این است که تمام مراحل انجام کار  به طور کاملاً خودکار انجام شود و در هیچ مرحله‌ای نیاز به دخالت عامل خارجی  ندارد بجز پیکربندی اولیه مانند آدرس پایگاه داده. همچنین در تمام مراحل سعی شده است که تمام اصول لازم در طراحی معماری نرم‌افزار به کار گرفته شود و نیازمندی‌های کیفی پروژه نیز مد نظر قرار گیرد. این نیازمندی‌ها شامل موارد زیر است:
\begin{enumerate}
\item 
\واژه{کارایی} : جهت پاسخ به این نیازمندی از پایگاه داده استفاده شده است.
\item
قابلیت نگهداری: این قابلیت از سایرین بیشتر حائز اهمیت است. زیرا پروژه های پروژهشی خود به صورت مستقیم کاربران عمومی ندارند و از این جهت نیازمند کارایی بالا یا رابط گرافیکی کاربر پسند نیستند. استفاده آن‌ها معمولاً در گسترش آن‌ها توسط سایر محققین است که راه را ادامه خواهند داد. 
\begin{itemize}
\item

 برای پاسخ به این نیازمندی اصول مربوط به کدنویسی  در فصل سوم و چهارم کتاب  \cite{martin2009clean} به کار گرفته شده است.
 \item
 از الگوهای نرم افزاری پر کاربرد مانند \نام{اداپتور}{Adaptor}،  \نام{فکتوری}{Factory} و \نام{سینگلتون}{Singelton} استفاده شده است.
 \item
 به منظور جلوگیری از قطعه کد تکراری از وراثت و توابع \واژه{عمومی} استفاده شده است. همینطور عمق وراثت از عدد ۳ بیشتر نشده است زیرا وراثت عمیق از خوانایی کد می‌کاهد و محل اشتباه خواهد بود. 
\end{itemize}
\item
امنیت: از آنجا که پروژه قرار نیست به استفاده ی عموم برسد و کاربران عمل متخاصمانه‌ای انجام نخواهند داد   به نوع خاصی از امنیت  نسبت به انواع متداول دارد.  باید روند توسعه‌ی پروژه دارای امنیت باشد. از این نظر که کدها مفقود نشوند یا در صورت اشتباه در توسعه بتوان پروژه را به حالت قبل بازگرداند. در این راستا کدهای پروژه در مخزن نرم افزاری از نوع گیت نگهداری شده که یک مخزن در کامپیوتر شخصی و دیگری در سایت \نام{بیت‌باکت}{Bitbucket - \url{https://bitbucket.org/alimohebbi/bug_predict }} 
قرار دارد. مزیت این سایت نسبت به گیت‌هاب این است مخازن خصوصی  را به صورت رایگان ارائه می دهد. در مخازن خصوصی  اجازه‌ی دسترسی تنها به  افراد تعیین شده از طرف مالک  داده می‌شود و عموم کاربران به آن دسترسی ندارند. از ابتدای شروع پیاده‌سازی کدها در مخازن بروزرسانی شده است. نمایی از ثبت‌های مختلف پروژه در مخزن در شکل \ref{fig:bitbucket}
آورده شده است. 
\end{enumerate}

 \begin{figure}[H]
	\centering
	\includegraphics[width=1\textwidth]{img/case_study/bitbucket.png}
	\caption{نمایی از مخزن نرم‌افزاری}
	\label{fig:bitbucket}
\end{figure}


\section{رویکرد اول : معیارهای فرآیند در کنار جهش}
در این قسمت  چگونگی استخراج معیارهای رویکرد اول شرح داده می‌شود. ابتدا بازم است اطلاعات مربوط به ثبت‌های  حاوی خطا از ابزار \lr{defects4j} بازیابی شود و سپس این اطلاعات با استفاده از مخرن نرم‌افزاری تکمیل شود. در مراحل بعد ابتدا معیار‌های فرآیند و سپس معیارهای جهش استخراج خواهند شد. 

\subsection{ استخراج اطلاعات مربوط به ثبت‌های   حاوی خطا}
اطلاعاتی که درباره‌ی ثبت‌های حاوی خطا قابل بازیابی است در زیر آمده است:
\begin{enumerate}
\item شناسه‌ی ثبت در مخزن 
\item نام فایل حاوی خطا
\item شماره ی خطا در ابزار \lr{defects4j}
\item شماره‌ی ثبت تعمیر خطا
\item نام پروژه
\item نام انتشار قبلی پروژه
\item شماره‌ی ثبت انتشار قبلی پروژه
\end{enumerate}

از میان اطلاعات بالا همگی به سادگی با استفاده از ابزار defect4j قابل استخراج است بجز دو مورد آخر. همچنین شماره‌ی ثبت تعمیر مورد استفاده قرار نگرفت ولی نگهداری شد چراکه ممکن بود لازم شود. 
برای بدست آوردن اطلاعات مربوط به هر انتشار لازم است که مخرن نرم افزاری هر پروژه مورد بررسی قرار گیرد. در  مخازن پروژه‌های نرم‌افزاری  از نوع گیت برای مشخص کردن یک رویداد مهم از \نام{تگ}{Tag} استفاده می‌شود. هر تگ می‌تواند به یک ثبت از برنامه اشاره کند. تگ می‌تواند نمایانگر رویدادهایی چون انتشار برنامه، انتشار بتا، و یا کاندید انتشار باشد. بنابرین با استفاده از تگ می‌تواند انتشار را پیدا کرد.\\

تگ‌های مخازن گیت دو نوع \واژه{سبک‌وزن} و \واژه{حاشیه‌نویسی شده} که در میان پروژه‌های مورد مطالعه از هر دو نوع جهت مشخص کردن انتشار استفاده شده است.  کار کردن با این دو نوع تگ دارای تفاوتهایی در پیاده‌سازی است که در ایتجا از پرداختن به جزییات صرف نظر می‌شود. \\ ابتدا همه‌ی تگ‌های موجود در مخازن نرم‌افزاری استخراج می‌شود و در پایگاه داده قرار می‌گیرد. از میان تگ‌های استخراج شده تگ‌های نا مرتبط با انتشار از پایگاه داده حذف می‌شود. تگ‌های نامرتبط با توجه به نام آنها مشخص می‌شود به عنوان مثال تگ‌هایی که حاوی لغات Beta یا Dev هستند نامرتبط  محسوب می‌شوند. در نهایت جدولی به نام ReleaseProject ساخته می‌شود که در آن اطلاعات انتشارهای مختلف وجود دارد. نمایی از این جدول در شکل \ref{fig:project-release} آمده است. \\
\begin{figure}[H]
	\centering
	\includegraphics[width=1\textwidth]{img/case_study/project-release.png}
	\caption{نمایی از جدول محتوای انتشارها}
	\label{fig:project-release}
\end{figure}


در قدم بعدی باید مشخص شود اولین انتشار ما قبل هر ثبت حاوی خطا کدام است. برای این منظور لیست ثبت‌ها  در یک پروژه به ترتیب زمانی بررسی می‌شود. اولین ثبت  ماقبل ثبت مورد نظر که مربوط به یک انتشار است یافت می‌شود و به عنوان انتشار ماقبل آن ثبت در نظر گرفته می‌شود. \\

 در نهایت جدولی به نام BugInfo تولید شده که نمایی از آن در شکل \ref{fig:bug-info} آمده است. این جدول ۴۰۵ سطر دارد که بیشتر از تعداد کل خطاهای ذکر شده در مجموعه داده‌ی \lr{defects4j} است. علت این است که یک خطا می‌تواند خطا در چندین پرونده به طور همزمان باشد و از آنجا که پیش‌بینی در سطح پرونده انجام می‌شود لازم است اطلاعات برای پرونده‌ها ذخیره شود.

\begin{figure}[H]
\centering
\includegraphics[width=1\textwidth]{img/case_study/bug-info.png}
\caption{نمایی از جدول محتوای اطلاعات پرونده‌های حاوی خطا}
\label{fig:bug-info}
\end{figure}


\subsection{ استخراج معیارهای فرآیند}
در  این قسمت نحوه‌ی استخراج هر یک از معیارهای ذکر شده در قسمت \ref{sec:method-phase1} بیان می‌شود. \\
\textbf{‫تعداد ثبت در سیستم کنترل نسخه‬:}
اولین راه حلی که به ذهن می‌رسد استفاده ی مستقیم از Jgit برای این کار است.  به این صورت که تعداد ثبت‌های بین ثبت کنونی و انتشار قبلی  بررسی کرده و تعداد  ثبت‌هایی که در آن‌ها فایل حاوی خطا تغییر کرده است شمرده شوند. مشکل این راه این است که بسیار پر هزینه  خواهد بود زیرا مرتبا باید عملیات \واژه{ورودی/خروجی} بر روی دیسک انجام پذیرد و همچنین بررسی‌های تکراری بسیاری انجام می‌گیرد. به عنوان مثال دو ثبت حاوی خطا را در نظر بگیرید که دارای انتشار ما قبل یکسانی هستند. تعدادی از بررسی‌های ثبت‌های ما بین آن‌ها تا ثبت مربوط به انتشار دارای همپوشانی خواهد بود. از طرف دیگر می‌توان اطلاعاتی که در بررسی ثبت‌ها بدست می‌آید در محاسبه‌ی معیارهای دیگر نیز مورد استفاده قرار گیرد.\\
همچنین برای یافتن ثبت‌های بین انتشار و ثبت مورد نظر نمی‌توان از تاریخ ثبت آنها استفاده کرد. زیرا تعداد زیادی از ثبت‌های ابتدای  برخی پروژه ‌های مورد مطالعه دارای تاریخ یکسانی هستند استفاده از تاریخ غیر ممکن می‌شود. علت داشتن تاریخ یکسان احتمالاً مهاجرت از یک نوع مخرن نرم‌افزاری به نوع گیت بوده است. \\

 بنابرین کل ثبت‌های پروژه‌ها مورد بررسی قرار گرفت و دو جدول تولید شد.
جدول اول به نام CommitInfo‌ که اطلاعات کلی ثبت‌ها را در بر می‌گیرد و جدول دوم  CommitChangedFile که اطلاعات مربوط به پرونده‌هایی که در یک ثبت از برنامه نسبت به ثبت قبلی تغییر کرده است نگهداری می‌شود.  در این جدول برای هر پرونده تعداد خطوط اضافه و کم شده نسبت به ثبت قبلی ذخیره شده است. در جدول اول \lr{Sequence\textunderscore Number} نشان می‌دهد که چندمین نسخه از ابتدای پروژه می‌باشد و این عدد در هنگام بررسی‌ها به آن ثبت داده‌ شده زیرا برای یافتن ثبت‌های بین ثبت کنونی و ثبت مربوط به انتشار قبلی لازم است از آنها استفاده شود. \\
  هر سطر از جدول دوم یک کلید خارجی دارد به سطری از جدول اول. قسمتی از جدول CommitInfo  در شکل \ref{fig:commit-info} و جدول CommitChangeFile در شکل \ref{fig:change-file-info}  زیر آمده است:

\begin{figure}[H]
	\centering
	\includegraphics[width=1\textwidth]{img/case_study/commit-info.png}
	\caption{نمایی از جدول اطلاعات ثبت‌ها}
	\label{fig:commit-info}
\end{figure}


\begin{figure}[H]
	\centering
	\includegraphics[width=1\textwidth]{img/case_study/change-file-info.png}
	\caption{ نمایی از جدول تغییرات پرونده‌ها در ثبت‌ها }
	\label{fig:change-file-info}
\end{figure}

در نهایت با استفاده از قطعه کد \ref{code:commit-info} اطلاعات مربوط به ثبت مورد نظر و ثبت انتشار بازیابی می‌شوند و سپس  از شماره‌ی دنباله‌ی آنها در پرسمان موجود در قطعه کد \ref{code:ncomm} استفاده می‌شود و معیار محاسبه می‌گردد.



\begin{latin}
	\begin{lstlisting}[language=SQL]
SELECT  * from CommitInfo CI where CI.COMMIT_GIT_ID = :gitId AND CI.PROJECT = :project
\end{lstlisting}
\end{latin}
\captionof{lstlisting}{بازیابی اطلاعات ثبت}
\label{code:commit-info}

\begin{latin}
\begin{lstlisting}[language=SQL]
SELECT  count(*) from CommitChangedFile CC where CC.COMMIT_INFO_ID IN
	(SELECT CI.ID from CommitInfo CI WHERE CI.SEQUENCE_NUMBER BETWEEN 
	 :startSeq AND :endSeq AND CI.PROJECT = :project)
AND CC.FILE_NAME = :fileName
\end{lstlisting}
\end{latin}
\captionof{lstlisting}{محاسبه‌ی معیار تعداد ثبت‌ در سیستم کنترل نسخه}
\label{code:comm}

\متن‌سیاه{‫تعداد توسعه‌دهندگان فعال:‬}
به منظور محاسبه‌ی این معیار تعداد آدرس ایمیل‌های ثبت‌کننده‌های ثبت‌هایی شمرده می‌شود که آن ثبت‌ها شماره‌ی دنباله‌ی آنها بین شماره‌ی دنباله‌ی ثبت پرونده‌ی مورد نظر و ثبت انتشار قبلی است و همچنین در آن ثبت پرونده‌ی مورد نظر در آن ثبت‌ها تغییر کرده است. به عبارت دیگر ثبت‌هایی که نام پرونده در جدول  CommitChangeFile  برای آن‌ها وجود دارد. 

\begin{latin}
\begin{lstlisting}[language=SQL]
SELECT  count(DISTINCT CI.COMMITTER_MAIL) from CommitInfo CI WHERE
CI.SEQUENCE_NUMBER BETWEEN :startSeq AND :endSeq AND CI.PROJECT = 
project AND CI.ID IN 
	(SELECT CC.COMMIT_INFO_ID from CommitChangedFile CC where CC.FILE_NAME = :fileName)
\end{lstlisting}
\end{latin}
\captionof{lstlisting}{محاسبه‌ی تعداد توسعه‌دهندگان فعال}

\textbf{تعداد توسعه‌دهندگان متمایز:}
برای محاسبه‌ی معیار از پرسمان قبلی استفاده می‌شود اما اینبار به جای استفاده \lr{Sequence\_Number} انتشار قبلی، عدد یک  قرار داده می‌شود که از ابتدای پروژه توسعه دهندگان شمرده شوند. 
\\

\متن‌سیاه{‫مقدار نرمال‌سازی شده‌ی تعداد خطوط اضافه شده:‬}
از پرسمان \ref{code:added-line-file} جهت محاسبه‌ی مجموع تعداد خطوط اضافه شده به پرونده در طول انتشار استفاده می‌شود و از پرسمان \ref{code:added-line-project} جهت محاسبه‌ی مجموع خطوط اضافه شده به پروژه استفاده می‌شود. سرانجام حاصل پرسمان اول بر دوم تقسیم می‌شود. 

\begin{latin}
	\begin{lstlisting}[language=SQL]
SELECT sum(CC.ADDED_LINES) from CommitChangedFile CC where
CC.COMMIT_INFO_ID IN
	(SELECT CI.ID from CommitInfo CI WHERE CI.SEQUENCE_NUMBER BETWEEN 
	:startSeq AND :endSeq AND CI.PROJECT = :project)
AND CC.FILE_NAME = :fileName
	\end{lstlisting}
\end{latin}
\captionof{lstlisting}{محاسبه‌ی تعداد خطوط اضافه شده به پرونده}
\label{code:added-line-file}

\begin{latin}
\begin{lstlisting}[language=SQL]
SELECT  sum(CC.ADDED_LINES) from CommitChangedFile CC where 
CC.COMMIT_INFO_ID IN
	(SELECT CI.ID from CommitInfo CI WHERE CI.SEQUENCE_NUMBER
	 BETWEEN :startSeq AND :endSeq AND CI.PROJECT = :project)
\end{lstlisting}
\end{latin}
\captionof{lstlisting}{محاسبه‌ی تعداد خطوط اضافه شده به پروژه}
\label{code:added-line-project}

\متن‌سیاه{‫مقدار نرمال‌سازی شده‌ی تعداد خطوط حذف شده:‬} به طور مشابه معیار قبلی محاسبه می‌گردد.

\textbf{درصد خطوطی که مالک فایل مشارکت کرده:}
دستور Blame در Jgit نشان می‌دهد که هر خط از پرونده در یک ثبت  در کدام یک از ثبت‌های گذشته اضافه شده است.  با یافتن ثبت مسئول اضافه کردن آن خط نویسنده‌ی آن خط مشخص می‌شود که همان ثبت‌کننده است. با کمک این دستور به دلایل مشابه ساخت جداول مربوط به ثبت‌ها، جدولی با عنوان Participation ساخته شده که در آن هر سطر نشان می‌دهد که یک نویسنده در یک نسخه از برنامه چند درصد از خطوط به وی اختصاص دارد. در شکل \ref{fig:participation} نمایی از این جدول آورده شده است.  از این جدول علاوه بر محاسبه‌ی این معیار برای یافت سایر معیارها نیز استفاده خواهد شد. در نهایت معیاری که در ابتدا بسیار پیچیده به نظر می رسید به کمک پرسمان ساده‌ی \ref{code:own} محاسبه خواهد شد. 

\begin{figure}[H]
	\centering
	\includegraphics[width=1\textwidth]{img/case_study/participation.png}
	\caption{نمایی از مخزن نرم‌افزاری}
	\label{fig:participation}
\end{figure}


\begin{latin}
	\begin{lstlisting}[language=SQL]
SELECT max(PARTICIPATION_PERCENT) from Participation P 
where COMMIT_ID = :commitId AND FILE_NAME = :fileName")
\end{lstlisting}
\end{latin}
\captionof{lstlisting}{محاسبه‌ی درصد خطوط مالک پرونده}
\label{code:own}

\textbf{تعداد مشارکت‌کنندگان جزئی:‬}
 با استفاده از جدول Participation و پرسمان  \ref{code:minor}  معیار محاسبه می‌شود. مقدار minorThereshold برابر ۵ درصد قرار می‌گیرد.

\begin{latin}
	\begin{lstlisting}[language=SQL]
SELECT count(AUTHOR_EMAIL) from Participation P 
where COMMIT_ID = :commitId AND FILE_NAME = :fileName
and PARTICIPATION_PERCENT < :minorThreshold
\end{lstlisting}
\end{latin}
\captionof{lstlisting}{محاسبه‌ي تعداد مشارکت‌کنندگان جزئی}
\label{code:minor}

\textbf{تعداد ثبت‌های همسایگان:‬}
ابتدا لازم است که همسایگان پرونده در یک ثبت و نیز تعداد دفعات همسایگی در طول انتشار مشخص شود. این عمل به وسیله‌ی پرسمان \ref{code:neighbor} انجام می‌شود. سپس معیار تعداد ثبت‌ها در سیستم کنترل نسخه مشابه قبل با استفاده از کد \ref{code:comm} محاسبه می‌گردد و از آنها میانگین وزن‌دهی شده گرفته می‌شود.

\begin{latin}
\begin{lstlisting}[language=SQL]
SELECT FILE_NAME as `name`, count(ID) as `frequency` FROM
CommitChangedFile WHERE COMMIT_INFO_ID IN
	(SELECT COMMIT_INFO_ID FROM CommitChangedFile WHERE FILE_NAME = :fileName) 
AND COMMIT_INFO_ID IN
	(SELECT CI.ID from CommitInfo CI WHERE CI.SEQUENCE_NUMBER BETWEEN :startSeq AND :endSeq AND PROJECT = :project)
AND FILE_NAME != :fileName GROUP BY FILE_NAME
\end{lstlisting}
\end{latin}
\captionof{lstlisting}{یافتن همسایگان و تعدد همسایگی}
\label{code:neighbor}

\textbf{تعداد توسعه‌دهندگان فعال همسایگان:‬}
 به طور مشابه با معیار قبلی محاسبه می‌شود.\\
\textbf{‫تعداد توسعه‌دهندگان متمایز همسایگان:‬}
 به طور مشابه با معیار قبلی محاسبه می‌شود. \\

\textbf{تجربه‌ی مالک فایل:‬‬}
 برای محاسبه‌ی معیار ابتدا   با استفاده از پرسمان \ref{code:find-owner} مالک پرونده مشخص می‌شود. سپس تعداد ثبت‌هایی که مالک پرونده از ابتدای پروژه تا آن زمان ثبت کرده است  با استفاده از پرسمان \ref{code:commit-of-commiter} شمرده می‌شود. به ترتیب از دو جدول Participation 
 و 
 CommitInfo  استفاده می‌شود.

\begin{latin}
\begin{lstlisting}[language=SQL]
SELECT AUTHOR_EMAIL FROM Participation P WHERE COMMIT_ID = :commitId 
AND FILE_NAME = :fileName AND PARTICIPATION_PERCENT  = 
	(SELECT max(PARTICIPATION_PERCENT) FROM Participation P2 
	 WHERE P2.COMMIT_ID = :commitId  AND P2.FILE_NAME = :fileName)
\end{lstlisting}
\end{latin}
\captionof{lstlisting}{یافتن مالک پرونده}
\label{code:find-owner}



\begin{latin}
\begin{lstlisting}[language=SQL]
SELECT  count(*) from CommitInfo CI where CI.SEQUENCE_NUMBER BETWEEN
:startSeq AND :endSeq AND CI.PROJECT = :project AND CI.COMMITTER_MAIL =
:authorEmail
\end{lstlisting}
\end{latin}
\captionof{lstlisting}{شمارش تعداد ثبت‌های یک ثبت کننده در بازه‌ی زمانی داده شده}
\label{code:commit-of-commiter}

\textbf{‫تجربه‌ی تمام مشارکت‌کنندگان:‬}
ابتدا همه‌ی توسعه‌دهندگان پرونده با استفاده از پرسمان \ref{code:contributers} مشخص می‌شوند.  سپس میزان تجربه‌ی هر یک با استفاده از پرسمان \ref{code:commit-of-commiter} جداگانه محاسبه می‌شود و از آن‌ها میانگین هندسی گرفته می شود. 
\begin{latin}
\begin{lstlisting}[language=SQL]
SELECT AUTHOR_EMAIL FROM Participation P WHERE COMMIT_ID = :commitId 
AND FILE_NAME = :fileName
\end{lstlisting}
\end{latin}
\captionof{lstlisting}{یافتن مشارکت‌کنندگان در پرونده}
\label{code:contributers}


در نهایت جدولی برای معیارهای فرآیند تولید می‌شود که نمایی از آن در شکل \ref{fig:process-metics} آورده شده است. 
\begin{figure}[H]
	\centering
	\includegraphics[width=1\textwidth]{img/case_study/process-metrics.png}
	\caption{نمایی از جدول معیارهای فرآیند }
	\label{fig:process-metics}
\end{figure}


 
\section{رویکرد دوم: معیارهای فرآیند مبتنی بر جهش}
همانطور که در قسمت \ref{sec:method-phase-two} اشاره شده چهار معیار معرفی شدند و مبتنی بر جهش نامیده شدند. این قسمت به نحوه‌ی پیاده‌سازی دسته‌ی دوم از معیارها را شرح خواهد داد. 
\begin{itemize}
\item
\متن‌سیاه{	تعداد جهش‌یافته‌های تولید شده‌ی جدید نسبت به انتشار قبلی برنامه:}
به منظور محاسبه‌ی این معیار ابتدا لازم است که مشخص شود که پرونده‌ی مورد نظر نسبت به انتشار قبلی چه تغییراتی داشته است. این کار با استفاده از ابزار JGit انجام  می‌شود. JGit این امکان را فراهم می‌کند که دو پرونده در دو ثبت متفاوت مقایسه شوند و مشخص می‌کند که کدام خطوط حذف شده‌اند و کدام خطوط اضافه شده‌اند. در اینجا لازم است خطوط اضافه شده  مشخص شود. سپس با استفاده از ابزار Major جهش‌یافته‌ها تولید می‌شود. در  قسمت  \ref{sec:tools-major} توضیح داده شد که پس تولید جهش‌یافته‌ها یک فایل خروجی نیز به نام mutant.log تولید می‌شود که در آن مشخص شده در هر خط از برنامه چه جهش‌یافته‌هایی تولید شده است. حال کافیست تعداد جهش‌یافته‌های تولید شده در خطوطی شمرده شوند که ابزار Jgit آن‌ها را به عنوان خطوط جدید نسبت به انتشار قبلی معرفی کرده است. بدین ترتیب این معیار محاسبه خواهد شد.\\
لازم به ذکر است روش یاد شده پایه ی محاسبه‌ی معیار بعدی و معیارهای رویکرد سوم است.
\end{itemize}

\section{رویکرد سوم: معیارهای ترکیبی جهش-فرآیند}

نحوه‌ی محاسبه به این صورت خواهید بود که ابتدا ثبت‌هایی از برنامه در طول آخرین انتشار که در آن فایل مورد نظر تغییر کرده است  توسط  پرسمان \ref{code:commit-during-release}بازیابی می شود. سپس برای هر ثبت تعداد جهش یافته‌های جدید نسبت به ثبت قبلی محاسبه می‌شود و برای محاسبه‌ی جهش‌یافته‌های حذف شده تعداد جهش یافته‌ها در  ثبت قبلی را یافته و آن‌ها که جز خطوط حذف شده در ثبت بعدی است شمرده می شود. تعداد جهش‌یافته‌های اضافه و حذف شده در ثبت‌ها جمع شده و بر تعداد ثبت‌های کل پروژه در طول انتشار تقسیم می گردد.



\begin{latin}
\begin{lstlisting}[language=SQL]
SELECT CC.* from CommitChangedFile CC, CommitInfo CI where CC.COMMIT_INFO_ID = CI.ID
AND CI.SEQUENCE_NUMBER BETWEEN :startSeq AND :endSeq
AND CI.PROJECT = :project
AND CC.FILE_NAME = :fileName ORDER BY CI.SEQUENCE_NUMBER asc
\end{lstlisting}
\end{latin}
\captionof{lstlisting}{بازیابی اطلاعات ثبت‌هایی که یک فایل در بازه‌ی مشخص در آنها تغییر کرده است}
\label{code:commit-during-release}
