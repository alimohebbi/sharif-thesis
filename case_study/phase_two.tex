\section{رویکرد دوم: معیارهای فرآیند مبتنی بر جهش}
همانطور که در قسمت \ref{sec:method-phase-two} اشاره شده چهار معیار معرفی شدند و مبتنی بر جهش نامیده شدند. این قسمت به نحوه‌ی پیاده‌سازی دسته‌ی دوم از معیارها را شرح خواهد داد. 
\begin{itemize}
\item
\متن‌سیاه{	تعداد جهش‌یافته‌های تولید شده‌ی جدید نسبت به انتشار قبلی برنامه:}
به منظور محاسبه‌ی این معیار ابتدا لازم است که مشخص شود که پرونده‌ی مورد نظر نسبت به انتشار قبلی چه تغییراتی داشته است. این کار با استفاده از ابزار JGit انجام  می‌شود. JGit این امکان را فراهم می‌کند که دو پرونده در دو ثبت متفاوت مقایسه شوند و مشخص می‌کند که کدام خطوط حذف شده‌اند و کدام خطوط اضافه شده‌اند. در اینجا لازم است خطوط اضافه شده  مشخص شود. سپس با استفاده از ابزار Major جهش‌یافته‌ها تولید می‌شود. در  قسمت  \ref{sec:tools-major} توضیح داده شد که پس تولید جهش‌یافته‌ها یک فایل خروجی نیز به نام mutant.log تولید می‌شود که در آن مشخص شده در هر خط از برنامه چه جهش‌یافته‌هایی تولید شده است. حال کافیست تعداد جهش‌یافته‌های تولید شده در خطوطی شمرده شوند که ابزار Jgit آن‌ها را به عنوان خطوط جدید نسبت به انتشار قبلی معرفی کرده است. بدین ترتیب این معیار محاسبه خواهد شد.\\
لازم به ذکر است روش یاد شده پایه ی محاسبه‌ی معیار بعدی و معیارهای رویکرد سوم است.
\end{itemize}
