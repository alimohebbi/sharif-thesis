\section{نکات پیاده‌سازی پروژه}

پیاده‌سازی پروژه در زبان جاوا انجام گرفت. یکی از نکات مهم و قابل توجه در پیاده‌سازی این پروژه این است که تمام مراحل انجام کار  به طور کاملاً خودکار انجام شود و در هیچ مرحله‌ای نیاز به دخالت عامل خارجی  ندارد بجز پیکربندی اولیه مانند آدرس پایگاه داده. همچنین در تمام مراحل سعی شده است که تمام اصول لازم در طراحی معماری نرم‌افزار به کار گرفته شود و نیازمندی‌های کیفی پروژه نیز مد نظر قرار گیرد. این نیازمندی‌ها شامل موارد زیر است:
\begin{enumerate}
\item 
\واژه{کارایی} : جهت پاسخ به این نیازمندی از پایگاه داده استفاده شده است.
\item
قابلیت نگهداری: این قابلیت از سایرین بیشتر حائز اهمیت است. زیرا پروژه های پروژهشی خود به صورت مستقیم کاربران عمومی ندارند و از این جهت نیازمند کارایی بالا یا رابط گرافیکی کاربر پسند نیستند. استفاده آن‌ها معمولاً در گسترش آن‌ها توسط سایر محققین است که راه را ادامه خواهند داد. 
\begin{itemize}
\item

 برای پاسخ به این نیازمندی اصول مربوط به کدنویسی  در فصل سوم و چهارم کتاب  \cite{martin2009clean} به کار گرفته شده است.
 \item
 از الگوهای نرم افزاری پر کاربرد مانند \نام{اداپتور}{Adaptor}،  \نام{فکتوری}{Factory} و \نام{سینگلتون}{Singelton} استفاده شده است.
 \item
 به منظور جلوگیری از قطعه کد تکراری از وراثت و توابع \واژه{عمومی} استفاده شده است. همینطور عمق وراثت از عدد ۳ بیشتر نشده است زیرا وراثت عمیق از خوانایی کد می‌کاهد و محل اشتباه خواهد بود. 
\end{itemize}
\item
امنیت: از آنجا که پروژه قرار نیست به استفاده ی عموم برسد و کاربران عمل متخاصمانه‌ای انجام نخواهند داد   به نوع خاصی از امنیت  نسبت به انواع متداول دارد.  باید روند توسعه‌ی پروژه دارای امنیت باشد. از این نظر که کدها مفقود نشوند یا در صورت اشتباه در توسعه بتوان پروژه را به حالت قبل بازگرداند. در این راستا کدهای پروژه در مخزن نرم افزاری از نوع گیت نگهداری شده که یک مخزن در کامپیوتر شخصی و دیگری در سایت \نام{بیت‌باکت}{Bitbucket - \url{https://bitbucket.org/alimohebbi/bug_predict }} 
قرار دارد. مزیت این سایت نسبت به گیت‌هاب این است مخازن خصوصی  را به صورت رایگان ارائه می دهد. در مخازن خصوصی  اجازه‌ی دسترسی تنها به  افراد تعیین شده از طرف مالک  داده می‌شود و عموم کاربران به آن دسترسی ندارند. از ابتدای شروع پیاده‌سازی کدها در مخازن بروزرسانی شده است. نمایی از ثبت‌های مختلف پروژه در مخزن در شکل \ref{fig:bitbucket}
آورده شده است. 
\end{enumerate}

 \begin{figure}[H]
	\centering
	\includegraphics[width=1\textwidth]{img/case_study/bitbucket.png}
	\caption{نمایی از مخزن نرم‌افزاری}
	\label{fig:bitbucket}
\end{figure}


\section{رویکرد اول : معیارهای فرآیند در کنار جهش}
در این قسمت  چگونگی استخراج معیارهای رویکرد اول شرح داده می‌شود. ابتدا بازم است اطلاعات مربوط به ثبت‌های  حاوی خطا از ابزار \lr{defects4j} بازیابی شود و سپس این اطلاعات با استفاده از مخرن نرم‌افزاری تکمیل شود. در مراحل بعد ابتدا معیار‌های فرآیند و سپس معیارهای جهش استخراج خواهند شد. 

\subsection{ استخراج اطلاعات مربوط به ثبت‌های   حاوی خطا}
اطلاعاتی که درباره‌ی ثبت‌های حاوی خطا قابل بازیابی است در زیر آمده است:
\begin{enumerate}
\item شناسه‌ی ثبت در مخزن 
\item نام فایل حاوی خطا
\item شماره ی خطا در ابزار \lr{defects4j}
\item شماره‌ی ثبت تعمیر خطا
\item نام پروژه
\item نام انتشار قبلی پروژه
\item شماره‌ی ثبت انتشار قبلی پروژه
\end{enumerate}

از میان اطلاعات بالا همگی به سادگی با استفاده از ابزار defect4j قابل استخراج است بجز دو مورد آخر. همچنین شماره‌ی ثبت تعمیر مورد استفاده قرار نگرفت ولی نگهداری شد چراکه ممکن بود لازم شود. 
برای بدست آوردن اطلاعات مربوط به هر انتشار لازم است که مخرن نرم افزاری هر پروژه مورد بررسی قرار گیرد. در  مخازن پروژه‌های نرم‌افزاری  از نوع گیت برای مشخص کردن یک رویداد مهم از \نام{تگ}{Tag} استفاده می‌شود. هر تگ می‌تواند به یک ثبت از برنامه اشاره کند. تگ می‌تواند نمایانگر رویدادهایی چون انتشار برنامه، انتشار بتا، و یا کاندید انتشار باشد. بنابرین با استفاده از تگ می‌تواند انتشار را پیدا کرد.\\

تگ‌های مخازن گیت دو نوع \واژه{سبک‌وزن} و \واژه{حاشیه‌نویسی شده} که در میان پروژه‌های مورد مطالعه از هر دو نوع جهت مشخص کردن انتشار استفاده شده است.  کار کردن با این دو نوع تگ دارای تفاوتهایی در پیاده‌سازی است که در ایتجا از پرداختن به جزییات صرف نظر می‌شود. \\ ابتدا همه‌ی تگ‌های موجود در مخازن نرم‌افزاری استخراج می‌شود و در پایگاه داده قرار می‌گیرد. از میان تگ‌های استخراج شده تگ‌های نا مرتبط با انتشار از پایگاه داده حذف می‌شود. تگ‌های نامرتبط با توجه به نام آنها مشخص می‌شود به عنوان مثال تگ‌هایی که حاوی لغات Beta یا Dev هستند نامرتبط  محسوب می‌شوند. در نهایت جدولی به نام ReleaseProject ساخته می‌شود که در آن اطلاعات انتشارهای مختلف وجود دارد. نمایی از این جدول در شکل \ref{fig:project-release} آمده است. \\
\begin{figure}[H]
	\centering
	\includegraphics[width=1\textwidth]{img/case_study/project-release.png}
	\caption{نمایی از جدول محتوای انتشارها}
	\label{fig:project-release}
\end{figure}


در قدم بعدی باید مشخص شود اولین انتشار ما قبل هر ثبت حاوی خطا کدام است. برای این منظور لیست ثبت‌ها  در یک پروژه به ترتیب زمانی بررسی می‌شود. اولین ثبت  ماقبل ثبت مورد نظر که مربوط به یک انتشار است یافت می‌شود و به عنوان انتشار ماقبل آن ثبت در نظر گرفته می‌شود. \\

 در نهایت جدولی به نام BugInfo تولید شده که نمایی از آن در شکل \ref{fig:bug-info} آمده است. این جدول ۴۰۵ سطر دارد که بیشتر از تعداد کل خطاهای ذکر شده در مجموعه داده‌ی \lr{defects4j} است. علت این است که یک خطا می‌تواند خطا در چندین پرونده به طور همزمان باشد و از آنجا که پیش‌بینی در سطح پرونده انجام می‌شود لازم است اطلاعات برای پرونده‌ها ذخیره شود.

\begin{figure}[H]
\centering
\includegraphics[width=1\textwidth]{img/case_study/bug-info.png}
\caption{نمایی از جدول محتوای اطلاعات پرونده‌های حاوی خطا}
\label{fig:bug-info}
\end{figure}


\subsection{ استخراج معیارهای فرآیند}
در  این قسمت نحوه‌ی استخراج هر یک از معیارهای ذکر شده در قسمت \ref{sec:method-phase1} بیان می‌شود. \\
\textbf{‫تعداد ثبت در سیستم کنترل نسخه‬:}
اولین راه حلی که به ذهن می‌رسد استفاده ی مستقیم از Jgit برای این کار است.  به این صورت که تعداد ثبت‌های بین ثبت کنونی و انتشار قبلی  بررسی کرده و تعداد  ثبت‌هایی که در آن‌ها فایل حاوی خطا تغییر کرده است شمرده شوند. مشکل این راه این است که بسیار پر هزینه  خواهد بود زیرا مرتبا باید عملیات \واژه{ورودی/خروجی} بر روی دیسک انجام پذیرد و همچنین بررسی‌های تکراری بسیاری انجام می‌گیرد. به عنوان مثال دو ثبت حاوی خطا را در نظر بگیرید که دارای انتشار ما قبل یکسانی هستند. تعدادی از بررسی‌های ثبت‌های ما بین آن‌ها تا ثبت مربوط به انتشار دارای همپوشانی خواهد بود. از طرف دیگر می‌توان اطلاعاتی که در بررسی ثبت‌ها بدست می‌آید در محاسبه‌ی معیارهای دیگر نیز مورد استفاده قرار گیرد.\\
همچنین برای یافتن ثبت‌های بین انتشار و ثبت مورد نظر نمی‌توان از تاریخ ثبت آنها استفاده کرد. زیرا تعداد زیادی از ثبت‌های ابتدای  برخی پروژه ‌های مورد مطالعه دارای تاریخ یکسانی هستند استفاده از تاریخ غیر ممکن می‌شود. علت داشتن تاریخ یکسان احتمالاً مهاجرت از یک نوع مخرن نرم‌افزاری به نوع گیت بوده است. \\

 بنابرین کل ثبت‌های پروژه‌ها مورد بررسی قرار گرفت و دو جدول تولید شد.
جدول اول به نام CommitInfo‌ که اطلاعات کلی ثبت‌ها را در بر می‌گیرد و جدول دوم  CommitChangedFile که اطلاعات مربوط به پرونده‌هایی که در یک ثبت از برنامه نسبت به ثبت قبلی تغییر کرده است نگهداری می‌شود.  در این جدول برای هر پرونده تعداد خطوط اضافه و کم شده نسبت به ثبت قبلی ذخیره شده است. در جدول اول \lr{Sequence\textunderscore Number} نشان می‌دهد که چندمین نسخه از ابتدای پروژه می‌باشد و این عدد در هنگام بررسی‌ها به آن ثبت داده‌ شده زیرا برای یافتن ثبت‌های بین ثبت کنونی و ثبت مربوط به انتشار قبلی لازم است از آنها استفاده شود. \\
  هر سطر از جدول دوم یک کلید خارجی دارد به سطری از جدول اول. قسمتی از جدول CommitInfo  در شکل \ref{fig:commit-info} و جدول CommitChangeFile در شکل \ref{fig:change-file-info}  زیر آمده است:

\begin{figure}[H]
	\centering
	\includegraphics[width=1\textwidth]{img/case_study/commit-info.png}
	\caption{نمایی از جدول اطلاعات ثبت‌ها}
	\label{fig:commit-info}
\end{figure}


\begin{figure}[H]
	\centering
	\includegraphics[width=1\textwidth]{img/case_study/change-file-info.png}
	\caption{ نمایی از جدول تغییرات پرونده‌ها در ثبت‌ها }
	\label{fig:change-file-info}
\end{figure}

در نهایت با استفاده از قطعه کد \ref{code:commit-info} اطلاعات مربوط به ثبت مورد نظر و ثبت انتشار بازیابی می‌شوند و سپس  از شماره‌ی دنباله‌ی آنها در پرسمان موجود در قطعه کد \ref{code:ncomm} استفاده می‌شود و معیار محاسبه می‌گردد.



\begin{latin}
	\begin{lstlisting}[language=SQL]
SELECT  * from CommitInfo CI where CI.COMMIT_GIT_ID = :gitId AND CI.PROJECT = :project
\end{lstlisting}
\end{latin}
\captionof{lstlisting}{بازیابی اطلاعات ثبت}
\label{code:commit-info}

\begin{latin}
\begin{lstlisting}[language=SQL]
SELECT  count(*) from CommitChangedFile CC where CC.COMMIT_INFO_ID IN
	(SELECT CI.ID from CommitInfo CI WHERE CI.SEQUENCE_NUMBER BETWEEN 
	 :startSeq AND :endSeq AND CI.PROJECT = :project)
AND CC.FILE_NAME = :fileName
\end{lstlisting}
\end{latin}
\captionof{lstlisting}{محاسبه‌ی معیار تعداد ثبت‌ در سیستم کنترل نسخه}
\label{code:comm}

\متن‌سیاه{‫تعداد توسعه‌دهندگان فعال:‬}
به منظور محاسبه‌ی این معیار تعداد آدرس ایمیل‌های ثبت‌کننده‌های ثبت‌هایی شمرده می‌شود که آن ثبت‌ها شماره‌ی دنباله‌ی آنها بین شماره‌ی دنباله‌ی ثبت پرونده‌ی مورد نظر و ثبت انتشار قبلی است و همچنین در آن ثبت پرونده‌ی مورد نظر در آن ثبت‌ها تغییر کرده است. به عبارت دیگر ثبت‌هایی که نام پرونده در جدول  CommitChangeFile  برای آن‌ها وجود دارد. 

\begin{latin}
\begin{lstlisting}[language=SQL]
SELECT  count(DISTINCT CI.COMMITTER_MAIL) from CommitInfo CI WHERE
CI.SEQUENCE_NUMBER BETWEEN :startSeq AND :endSeq AND CI.PROJECT = 
project AND CI.ID IN 
	(SELECT CC.COMMIT_INFO_ID from CommitChangedFile CC where CC.FILE_NAME = :fileName)
\end{lstlisting}
\end{latin}
\captionof{lstlisting}{محاسبه‌ی تعداد توسعه‌دهندگان فعال}

\textbf{تعداد توسعه‌دهندگان متمایز:}
برای محاسبه‌ی معیار از پرسمان قبلی استفاده می‌شود اما اینبار به جای استفاده \lr{Sequence\_Number} انتشار قبلی، عدد یک  قرار داده می‌شود که از ابتدای پروژه توسعه دهندگان شمرده شوند. 
\\

\متن‌سیاه{‫مقدار نرمال‌سازی شده‌ی تعداد خطوط اضافه شده:‬}
از پرسمان \ref{code:added-line-file} جهت محاسبه‌ی مجموع تعداد خطوط اضافه شده به پرونده در طول انتشار استفاده می‌شود و از پرسمان \ref{code:added-line-project} جهت محاسبه‌ی مجموع خطوط اضافه شده به پروژه استفاده می‌شود. سرانجام حاصل پرسمان اول بر دوم تقسیم می‌شود. 

\begin{latin}
	\begin{lstlisting}[language=SQL]
SELECT sum(CC.ADDED_LINES) from CommitChangedFile CC where
CC.COMMIT_INFO_ID IN
	(SELECT CI.ID from CommitInfo CI WHERE CI.SEQUENCE_NUMBER BETWEEN 
	:startSeq AND :endSeq AND CI.PROJECT = :project)
AND CC.FILE_NAME = :fileName
	\end{lstlisting}
\end{latin}
\captionof{lstlisting}{محاسبه‌ی تعداد خطوط اضافه شده به پرونده}
\label{code:added-line-file}

\begin{latin}
\begin{lstlisting}[language=SQL]
SELECT  sum(CC.ADDED_LINES) from CommitChangedFile CC where 
CC.COMMIT_INFO_ID IN
	(SELECT CI.ID from CommitInfo CI WHERE CI.SEQUENCE_NUMBER
	 BETWEEN :startSeq AND :endSeq AND CI.PROJECT = :project)
\end{lstlisting}
\end{latin}
\captionof{lstlisting}{محاسبه‌ی تعداد خطوط اضافه شده به پروژه}
\label{code:added-line-project}

\متن‌سیاه{‫مقدار نرمال‌سازی شده‌ی تعداد خطوط حذف شده:‬} به طور مشابه معیار قبلی محاسبه می‌گردد.

\textbf{درصد خطوطی که مالک فایل مشارکت کرده:}
دستور Blame در Jgit نشان می‌دهد که هر خط از پرونده در یک ثبت  در کدام یک از ثبت‌های گذشته اضافه شده است.  با یافتن ثبت مسئول اضافه کردن آن خط نویسنده‌ی آن خط مشخص می‌شود که همان ثبت‌کننده است. با کمک این دستور به دلایل مشابه ساخت جداول مربوط به ثبت‌ها، جدولی با عنوان Participation ساخته شده که در آن هر سطر نشان می‌دهد که یک نویسنده در یک نسخه از برنامه چند درصد از خطوط به وی اختصاص دارد. در شکل \ref{fig:participation} نمایی از این جدول آورده شده است.  از این جدول علاوه بر محاسبه‌ی این معیار برای یافت سایر معیارها نیز استفاده خواهد شد. در نهایت معیاری که در ابتدا بسیار پیچیده به نظر می رسید به کمک پرسمان ساده‌ی \ref{code:own} محاسبه خواهد شد. 

\begin{figure}[H]
	\centering
	\includegraphics[width=1\textwidth]{img/case_study/participation.png}
	\caption{نمایی از مخزن نرم‌افزاری}
	\label{fig:participation}
\end{figure}


\begin{latin}
	\begin{lstlisting}[language=SQL]
SELECT max(PARTICIPATION_PERCENT) from Participation P 
where COMMIT_ID = :commitId AND FILE_NAME = :fileName")
\end{lstlisting}
\end{latin}
\captionof{lstlisting}{محاسبه‌ی درصد خطوط مالک پرونده}
\label{code:own}

\textbf{تعداد مشارکت‌کنندگان جزئی:‬}
 با استفاده از جدول Participation و پرسمان  \ref{code:minor}  معیار محاسبه می‌شود. مقدار minorThereshold برابر ۵ درصد قرار می‌گیرد.

\begin{latin}
	\begin{lstlisting}[language=SQL]
SELECT count(AUTHOR_EMAIL) from Participation P 
where COMMIT_ID = :commitId AND FILE_NAME = :fileName
and PARTICIPATION_PERCENT < :minorThreshold
\end{lstlisting}
\end{latin}
\captionof{lstlisting}{محاسبه‌ي تعداد مشارکت‌کنندگان جزئی}
\label{code:minor}

\textbf{تعداد ثبت‌های همسایگان:‬}
ابتدا لازم است که همسایگان پرونده در یک ثبت و نیز تعداد دفعات همسایگی در طول انتشار مشخص شود. این عمل به وسیله‌ی پرسمان \ref{code:neighbor} انجام می‌شود. سپس معیار تعداد ثبت‌ها در سیستم کنترل نسخه مشابه قبل با استفاده از کد \ref{code:comm} محاسبه می‌گردد و از آنها میانگین وزن‌دهی شده گرفته می‌شود.

\begin{latin}
\begin{lstlisting}[language=SQL]
SELECT FILE_NAME as `name`, count(ID) as `frequency` FROM
CommitChangedFile WHERE COMMIT_INFO_ID IN
	(SELECT COMMIT_INFO_ID FROM CommitChangedFile WHERE FILE_NAME = :fileName) 
AND COMMIT_INFO_ID IN
	(SELECT CI.ID from CommitInfo CI WHERE CI.SEQUENCE_NUMBER BETWEEN :startSeq AND :endSeq AND PROJECT = :project)
AND FILE_NAME != :fileName GROUP BY FILE_NAME
\end{lstlisting}
\end{latin}
\captionof{lstlisting}{یافتن همسایگان و تعدد همسایگی}
\label{code:neighbor}

\textbf{تعداد توسعه‌دهندگان فعال همسایگان:‬}
 به طور مشابه با معیار قبلی محاسبه می‌شود.\\
\textbf{‫تعداد توسعه‌دهندگان متمایز همسایگان:‬}
 به طور مشابه با معیار قبلی محاسبه می‌شود. \\

\textbf{تجربه‌ی مالک فایل:‬‬}
 برای محاسبه‌ی معیار ابتدا   با استفاده از پرسمان \ref{code:find-owner} مالک پرونده مشخص می‌شود. سپس تعداد ثبت‌هایی که مالک پرونده از ابتدای پروژه تا آن زمان ثبت کرده است  با استفاده از پرسمان \ref{code:commit-of-commiter} شمرده می‌شود. به ترتیب از دو جدول Participation 
 و 
 CommitInfo  استفاده می‌شود.

\begin{latin}
\begin{lstlisting}[language=SQL]
SELECT AUTHOR_EMAIL FROM Participation P WHERE COMMIT_ID = :commitId 
AND FILE_NAME = :fileName AND PARTICIPATION_PERCENT  = 
	(SELECT max(PARTICIPATION_PERCENT) FROM Participation P2 
	 WHERE P2.COMMIT_ID = :commitId  AND P2.FILE_NAME = :fileName)
\end{lstlisting}
\end{latin}
\captionof{lstlisting}{یافتن مالک پرونده}
\label{code:find-owner}



\begin{latin}
\begin{lstlisting}[language=SQL]
SELECT  count(*) from CommitInfo CI where CI.SEQUENCE_NUMBER BETWEEN
:startSeq AND :endSeq AND CI.PROJECT = :project AND CI.COMMITTER_MAIL =
:authorEmail
\end{lstlisting}
\end{latin}
\captionof{lstlisting}{شمارش تعداد ثبت‌های یک ثبت کننده در بازه‌ی زمانی داده شده}
\label{code:commit-of-commiter}

\textbf{‫تجربه‌ی تمام مشارکت‌کنندگان:‬}
ابتدا همه‌ی توسعه‌دهندگان پرونده با استفاده از پرسمان \ref{code:contributers} مشخص می‌شوند.  سپس میزان تجربه‌ی هر یک با استفاده از پرسمان \ref{code:commit-of-commiter} جداگانه محاسبه می‌شود و از آن‌ها میانگین هندسی گرفته می شود. 
\begin{latin}
\begin{lstlisting}[language=SQL]
SELECT AUTHOR_EMAIL FROM Participation P WHERE COMMIT_ID = :commitId 
AND FILE_NAME = :fileName
\end{lstlisting}
\end{latin}
\captionof{lstlisting}{یافتن مشارکت‌کنندگان در پرونده}
\label{code:contributers}


در نهایت جدولی برای معیارهای فرآیند تولید می‌شود که نمایی از آن در شکل \ref{fig:process-metics} آورده شده است. 
\begin{figure}[H]
	\centering
	\includegraphics[width=1\textwidth]{img/case_study/process-metrics.png}
	\caption{نمایی از جدول معیارهای فرآیند }
	\label{fig:process-metics}
\end{figure}


 