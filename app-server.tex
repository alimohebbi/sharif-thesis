 
\chapter{آماده‌سازی رایانه به عنوان سرور}
\label{app:server}
انجام تحلیل جهش امری زمان‌بر است. به همین علت لازم است که رایانه‌ای به این فرآیند اختصاص یابد تا  این کار بدون وقفه انجام شود و رفع خطا در زمان توسعه‌ی کد در هر مکان و زمانی امکان پذیر باشد. در ادامه گام‌های لازم برای تبدیل رایانه به سرور آمده است. 

\section{تنظیمات پایگاه داده}
پایگاه‌داده‌ی  مورد استفاده در این پژوهش \lr{MySQL 5.7.22} می‌باشد. در ابتدا لازم است امکان برقراری ارتباط از راه دور توسط \نام{آی‌پی}{IP}‌های خارج از رایانه فرآهم شود. در فایل   mysqld.cnf پیکربندهای پایگاه داده وجود دارد و این فایل در آدرس 
\lr{/etc/mysql/mysql.conf.d}
قرار دارد. در این فایل لازم است که پارامتر bind-address با استفاده از \# به \نام{کامنت} {Comment}تبدیل شود. \\ 
سپس لازم است که یک کاربر مشخص شود که با هر آی‌پی بتواند به پایگاه‌داده وارد شود. این عمل می‌تواند با استفاده از نرم‌افزار  Workbench به سادگی انجام شود. این نرم‌افزار ابزار طراحی، توسعه و مدیریت پایگاه‌داده است. در قسمت server و سپس
\lr{ users and previlages}
  می‌توان به سادگی کاربر مورد نیاز را تعریف کرد. پس از انجام این تنظیمات لازم است که پایگاه داده راه‌اندازی مجدد شود. 
  
\section{ارتباط با اینترنت}
پیشنیاز اولیه هر سرور ارتباط با اینترنت می‌باشد. در برخی از شبکه‌ها برای برقراری این ارتباط لازم است از VPN مخصوص به آن شبکه استفاده شود. مشکلی که اغلب  یک vpn  دارد قطع شدن ارتباط آن است و لازم است این ارتباط پس از قطع دوباره ایجاد شود. قطعه کد  \ref{code:vpn} هر ۳۰ ثانیه ارتباط را چک می‌کند و در صورت قطع vpn را مجددا راه اندازی می‌کند. 




\begin{latin}
	\begin{lstlisting}[language=bash]
#!/bin/bash +x
while [ "true" ]
	do
	CON="Sharif-ID2"
	STATUS=`nmcli con show --active | grep $CON | cut -f1 -d " "`
	if [ -z "$STATUS" ]; then
		echo "Disconnected, trying to reconnect..."
		(sleep 1s && nmcli con up $CON)
	else
		echo "Already connected !"
	fi
	sleep 30
done
\end{lstlisting}
\end{latin}
\captionof{lstlisting}{راه اندازی مجدد vpn}
\label{code:vpn}

\section{
	رفع مشکل آی‌پی \نام{پویا}{Dynamic}
}

برای ارتباط با هر رایانه از راه دور لازم است که آدرس آن رایانه را داشته باشیم که این آدرس همان آی‌پی می‌باشد. در بسیاری از شبکه‌ها این آدرس  به دلایل مختلف ثابت نیست. به منظور حل این مشکل  می‌توان از سرویس‌هایی استفاده کرد که امکان \واژه{انقیاد} آی‌پی به آدرس URL را فراهم می‌کنند و همواره با تغییر آی‌پی، آدرس URL را به آی‌پی جدید متصل می‌کنند. سرویسی که در این پژوهش استفاده شد متعلق به سایت
\نام{noip }{\url{www.noip.com}} بود. این سایت یک برنامه هم فراهم می‌کند که این برنامه بر روی رایانه‌ی مورد نظر نصب می‌شود و در بازه‌های زمانی مشخص آدرس آی‌پی را برای سرویس ارسال می‌کند و انقیاد آدرس انجام می‌شود. 
\section{ارتباط با ترمینال}
ترمینال این امکان را فراهم می‌سازد تا تمام عملیات‌های ممکن در یک سیستم عامل از طریق آن انجام شود. یکی از راههای متدوال و مطمئن  ارتباط از راه دور استفاده از پروتکل ssh می‌باشد. برای استفاده از این پروتکل لازم است یک سرویس SSH بر روی سرور راه اندازی شود یکی از نرم‌افزارهایی که این کار را انجام می‌دهد OpenSSH می‌باشد. نسخه‌ی Client از این ابزار نیز بر روی رایانه‌ی مشتری نصب می‌گردد. کاربران مجاز می‌توانند از طریق گذرواژه  مشخص شده با سرویس   SSH بر روی سرور ارتباط برقرار کنند. اما راه ایمن‌تر شناسایی از طریق کلید است. کاربر یک کلید عمومی و خصوصی تولید می‌کند. کلید عمومی در نزد  سرور نگهداری می‌شود و کلید خصوصی در نزد کاربر و برای برقراری ارتباط از این کلید‌ها استفاده می‌شود.\\

در ارتباط SSH یک ترمینال برای کاربر ایجاد می‌شود که برای اجرای پروژه در پس‌زمینه کافی نیست. زیرا با قطع ارتباط اجرا نیز متوقف می‌شود. برای دسترسی به چند ترمینال از طریق یک ترمینال می‌توان از ابزار Screen استفاده نمود. پس از برقراری ارتباط SSH در ترمینال باز شده ترمینال‌های مختلفی از طریق ابزار Screen می‌توان ایجاد و مدیریت کرد. هر یک از این ترمینال‌ها در پس زمینه می‌توانند کار خود را بدون توجه به وجود یا عدم وجود ارتباط SSH ادامه دهند. 
\section{ساخت و اجرای پروژه‌ی جاوا}
برای \واژه{ساخت} و اجرای یک پروژه‌ی جاوا به صورت خودکار ابزارهای مختلفی وجود دارد. یکی از ابزارهای مناسب که به آن اشاره شد Maven است. لازم است که پیکربندی‌های مناسب جهت استفاده از وابستگی‌ها، کامپایل و ساخت فایل اجرایی jar  در فایل pom.xml انجام شود. این تنظیمات در قطعه کد \ref{code:pom} آمده است. سپس با استفاده از قطعه کد xx پروژه ساخته و اجرا می‌شود. از آنجا که پروژه به منظور کامپایل به نسخه‌ی ۸ جاوا کامپایلر نیاز دارد و در هنگام اجرا به نسخه‌ی ۷، این تنظمات به صورت خودکار انجام می‌شود. همچنین لازم نیست که با تغییر کد بر روی رایانه‌ی \ترجمه{مشتری}{Client}  کدها مستقیما بر به سرور منتقل شوند. کافیست از سیستم کنترل نسخه استفاده شود. کدها  با استفاده از ابزار گیت در سیستم کنترل نسخه‌ی بر روی \ترجمه{ابر}{Cloud} آپلود شود و سپس توسط همین ابزار در سرور دانلود شود. اصطلاحا به این عمل Pull و Push می‌گویند.
\definecolor{dkgreen}{rgb}{0,0.6,0}
\definecolor{gray}{rgb}{0.5,0.5,0.5}
\definecolor{mauve}{rgb}{0.58,0,0.82}
\definecolor{gray}{rgb}{0.4,0.4,0.4}
\definecolor{darkblue}{rgb}{0.0,0.0,0.6}
\definecolor{lightblue}{rgb}{0.0,0.0,0.9}
\definecolor{cyan}{rgb}{0.0,0.6,0.6}
\definecolor{darkred}{rgb}{0.6,0.0,0.0}

\lstdefinelanguage{x}
{
	morestring=[s][\color{mauve}]{"}{"},
	morestring=[s][\color{black}]{>}{<},
	morecomment=[s]{<?}{?>},
	morecomment=[s][\color{dkgreen}]{<!--}{-->},
	stringstyle=\color{black},
	identifierstyle=\color{lightblue},
	keywordstyle=\color{red},
	morekeywords={xmlns,xsi,noNamespaceSchemaLocation,type,id,x,y,source,target,version,tool,transRef,roleRef,objective,eventually}% list your attributes here
}
\lstset{
	basicstyle=\ttfamily\footnotesize,
	columns=fullflexible,
	showstringspaces=false,
	numbers=left,                   % where to put the line-numbers
	numberstyle=\tiny\color{gray},  % the style that is used for the line-numbers
	stepnumber=1,
	numbersep=5pt,                  % how far the line-numbers are from the code
	backgroundcolor=\color{white},      % choose the background color. You must add \usepackage{color}
	showspaces=false,               % show spaces adding particular underscores
	showstringspaces=false,         % underline spaces within strings
	showtabs=false,                 % show tabs within strings adding particular underscores
	frame=none,                   % adds a frame around the code
	rulecolor=\color{black},        % if not set, the frame-color may be changed on line-breaks within not-black text (e.g. commens (green here))
	tabsize=2,                      % sets default tabsize to 2 spaces
	captionpos=b,                   % sets the caption-position to bottom
	breaklines=true,                % sets automatic line breaking
	breakatwhitespace=false,        % sets if automatic breaks should only happen at whitespace
	title=\lstname,                   % show the filename of files included with \lstinputlisting;
	% also try caption instead of title  
	commentstyle=\color{gray}\upshape
}

\begin{latin}
	\begin{lstlisting}[language=x]
   <?xml version="1.0" encoding="UTF-8"?>
   <build>
   <plugins>
   <plugin>
   <groupId>org.apache.maven.plugins</groupId>
   <artifactId>maven-compiler-plugin</artifactId>
   <configuration>
   <source>1.8</source>
   <target>1.8</target>
   </configuration>
   </plugin>
   <plugin>
   <!-- Build an executable JAR -->
   <groupId>org.apache.maven.plugins</groupId>
   <artifactId>maven-jar-plugin</artifactId>
   <version>3.1.0</version>
   <configuration>
   <archive>
   <manifest>
   <addClasspath>true</addClasspath>
   <classpathPrefix>lib/</classpathPrefix>
   <mainClass>Main</mainClass>
   </manifest>
   </archive>
   </configuration>
   </plugin>
   <plugin>
   <artifactId>maven-assembly-plugin</artifactId>
   <configuration>
   <archive>
   <manifest>
   <mainClass>Main</mainClass>
   </manifest>
   </archive>
   <descriptorRefs>
   <descriptorRef>jar-with-dependencies</descriptorRef>
   </descriptorRefs>
   </configuration>
   </plugin>
   </plugins>
   </build>
\end{lstlisting}
\end{latin}
\captionof{lstlisting}{پیکربندی pom.xml}
\label{code:pom}

