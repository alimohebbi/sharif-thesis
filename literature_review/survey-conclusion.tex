\section{جمع بندی مطالعات پیشین}
\label{sec:conclustion}
هدف از پیش‌بینی خطا کمک به توسعه‌دهندگان نرم‌افزار و کاهش هزینه‌های نرم‌افزاری می‌باشد. روند پیش‌بینی خطا به این صورت است که با استفاده از مخازن نرم‌افزاری همانند سیستم کنترل نسخه و سیستم ردگیری خطا، اطلاعات کد منبع، خطا و اطلاعات تاریخی پروژه جمع آوری می‌شود. با توجه به معیارهای مختلف داده‌هایی استخراج می‌شود که هر داده دارای برچسب "سالم" یا "حاوی خطا" می‌باشد. قسمتی از این داده‌ها با استفاده از روش‌های یادگیری ماشین، مدل‌های پیش‌بینی خطا را تولید می‌کنند و قسمت دیگر جهت آزمایش مدل به کار گرفته می‌شود.\\

معیارهای متداول در ارزیابی پیش‌بینی دقت و فراخوانی می‌باشند. این معیارها دارای نواقصی هستند. به عنوان مثال مدلی که همه‌ی داده‌ها را خطا دار معرفی می‌کند دارای فراخوانی  برابر یک است و مسلما این مدل کارایی مناسبی ندارد. معیار اف  میانگین هارمونیک دو معیار قبلی است و نواقص آنها را بر طرف می‌کند. یکی از معیار‌های رایج برای مقایسه‌ی مدل‌های یادگیری ماشین \lr{AUC} می‌باشد. هرچه این مساحت بیشتر باشد و منحنی مربوطه سریعتر  در راستای محور عمودی  به یک برسد مدل کارایی بهتری دارد. با استفاده از معیار \lr{AUCEC} می‌توان موثر بودن مدل از نظر هزینه را سنجید. معمولا چند درصد اول از منحنی مربوطه در نظر گرفته می‌شود و مساحت آن محاسبه می‌شود. \\

معیارهای مورد استفاده را می‌توان به سه دسته‌ی معیار سنتی کد، معیار شئ گرایی و معیار فرآیند تقسیم کرد. در برخی از منابع نیز  به دو دسته‌ی کلی معیار کد و معیار فرآیند تقسیم شده‌اند. معیارهای اندازه جزء معیارهای ابتدایی و موثر هستند و معیارهای پیچیدگی و شئ گرایی همبستگی فراوانی با معیارهای اندازه دارند. معیارهای شئ گرایی دارای وابستگی فراوانی با معیار‌های اندازه هستند. با این حال معیارهای شئ گرایی دارای توانایی بیشتری هستند. معیارهای فرآیند از جنبه‌های مختلفی  مانند عدم رکود در تکرار‌های چرخه‌ی تولید نرم‌افزار و موثر بودن از نظر هزینه از سایر معیارها برتری دارد. علی‌رغم توانمندی بالقوه‌ی معیارهای فرآیند در پیش‌بینی خطا، این معیارها در پژوهش‌های کمتری مورد تحقیق قرار گرفته‌اند. \\

در پژوهش‌های مختلف از روش‌های یادگیری ماشین متفاوتی استفاده شده است. در صورتی که هدف پیش‌بینی تعداد خطاها باشد از رگرسیون و در صورتی که هدف پیش‌بینی حاوی خطا بودن باشد از دسته‌بندی استفاده می‌شود. پژوهش \cite{arisholm2010systematic}  نشان داده است که روش دسته‌بندی تاثیر متوسطی بر کارایی پیش‌بینی خطا دارد و انتخاب معیار مهم‌تر است. \\

 در ابتدا از امتیاز جهش برای میزان موثر بودن مجموعه آزمون استفاده می‌شد و سپس کاربردهای دیگری همچون انتخاب، رتبه‌بندی و کمینه کردن مجموعه آزمون پیدا کرده است. همچنین در پژوهش‌های اخیر جهت مکان‌یابی خطا و پیش‌بینی خطا مورد استفاده قرار گرفته است. در پژوهش \cite{just2014mutants} نشان داده شده است که جهش‌یافته‌هایی  که با عملگرهای جهش ساده تولید شده‌اند می‌توانند تا 73 \lr{\%} خطاهای واقعی را شبیه سازی کنند و ازین جهت جایگزین مناسبی برای خطاهای واقعی باشند. 


\begin{table}[H] 
	\centering \caption{جدول مشخصات پژوهش‌ها‌‌ی مرور شده در حوزه‌ی پيش بيني خطا}
	\label{tab:survey}
	\newcolumntype{P}[1]{>{\centering\arraybackslash}m{#1}}
	\renewcommand*{\arraystretch}{1.5}
	\begin{tabular}{|P{1cm}|P{2cm}|P{3cm}|P{1.5cm}|P{2cm}|P{1.5cm}|P{1cm}|}
		\hline
		\hline
		مقاله & معیار   & تکنیک یادگیری &  ریزدانگی &روش ارزیابی & نوع پروژه‌ها&‌ زبان پروژه‌ها   \\
		\hline
		\hline
		
		\cite{ostrand2010programmer}
		 & فرآیند - سنتی & \lr{NBR} & فایل & مشابه \lr{AUCEC} & خصوصی & جاوا \\
		 \hline
		
		\cite{rahman2013and}
	&	فرآیند - سنتی - شئ‌گرایی & 
		\lr{Naive Bayes - Logestic Regression - SMV - J48}
		& فایل &  \lr{AUC - AUCEC - F-Measure} & متن باز& جاوا \\

		\hline
		\cite{bowes2016mutation}
		& سنتی - شئ‌گرایی & 
			\lr{Naive Bayes - Logestic Regression - Random Forest - J48} 
			&
			کلاس & غیره & متن باز & جاوا \\
		\hline
		\cite{malhotra2014comparative} &
		سنتی &  
		\lr{LR - ANN - DT - SVM - CCN - GMDH - GEP} & \lr {NA} & \lr {AUC - Precision} & 
		متن باز & سی \\
		\hline
		
		\cite{xia2016predicting} &
		سنتی - فرآیند &
		\lr{Naive Bayes - DT - kNN - RF} & سیستم & \lr {AUC - Precision - Recall - F-Measure} & 
		متن باز & اندروید \\
		\hline
		\cite{kumar2017emprical} &
		سنتی - شئ‌گرایی &
		\lr{LR - ANN - RBFN } & کلاس & \lr {Accuracy - F-Measure} & 
		متن باز & جاوا \\
		\hline
		
		
		
	
\end{tabular}
\end{table}









