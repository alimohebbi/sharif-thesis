\subsection{مدل‌های پیش‌بینی خطا}
\label{subsec:models}
اکثریت مدل‌های پیش‌بینی خطا بر اساس یادگیری ماشین می‌باشند. بر اساس اینکه چه چیزی پیش‌بینی شود (خطاخیز بودن یا تعداد خطا)، مدل‌ها به دو دسته‌ی کلی تقسیم می‌شوند، که عبارتند از دسته‌بندی و رگرسیون. با توسعه‌ی روش‌های جدیدتر یادگیری ماشین تکنیک‌های فعال و \واژه{نیمه-نظارتی} برای ساخت مدل‌های پیش‌بینی خطای کاراتر به کار گرفته شده است\cite{li2012sample}. علاوه بر مدل‌های یادگیری ماشین، مدل‌های غیر آماری مانند  \نام{باگ‌کَش}{BugCache} پیشنهاد داده شده است \cite{kim2007predicting}. در میان روش‌های دسته‌بندی، 
\واژه{رگرسیون منطقی} ،
\واژه{بیز ساده} و
\واژه{درخت تصمیم}
بیش از سایرین در پژوهش‌ها مورد استفاده قرار گرفته‌اند. همچنین در میان روش‌های رگرسیون،
\واژه{رگرسیون خطی} و 
\واژه{رگرسیون دوبخشی منفی}  
به طور گسترده به کار گرفته شده‌اند \cite{nam2014survey}. \\
\نام{کیم}{Kim} و همکاران  \موکد{باگ‌کَش} را ارائه داده‌اند که  اولویت موجودیت‌های خطاخیز در \واژه{حافظه‌ی موقت}  را نگهداری  می‌کند. این روش از اطلاعات محلی خطاها مانند اطلاعات زمانی و مکانی بهره می‌گیرد. به عنوان مثال اگر خطا در یک موجودیت به تازگی به وجود آمده یا همراه با سایر موجودیت‌ها تغییر کرده است، آن موجودیت با احتمال بیشتری حاوی خطا خواهد بود.\\
اگرچه مدل‌های یادگیری مختلف می‌تواند  با توجه به داده‌های ورودی یکسان، متفاوت عمل کنند و کارایی یک روش نسبت به دیگری متفاوت باشد، با این حال پژوهشی که توسط آریشلم  و همکاران  \cite{arisholm2010systematic} انجام شده است نشان می‌دهد که تاثیر  تکنیک یادگیری در حد متوسطی است و کمتر از انتخاب معیار بر روی کارایی تاثیر گذار است.  \\

\نام{مالهوترا}{Malhotra} با بکارگیری معیارهای سنتی کد، عملکرد تکنیک‌های یادگیری ماشین و رگرسیون را مقایسه کرده است\cite{malhotra2014comparative}. وی به منظور پیش پردازش نیز از \واژه{آماره‌های توصیفی}  استفاده کرده است و داده‌های نامناسب را شناسایی نموده است. آماره‌های توصیفی می‌توانند شامل میانگین، کمینه، بیشینه و واریانس باشد. متغیرهای مستقلی که  واریانس کمی دارند ماژول‌ها را به خوبی متمایز نمی‌کنند و بعید است که مفید باشند و می‌توانند حذف شوند. در این مقاله یک روش رگرسیون و شش روش دسته‌بندی مورد آزمایش قرار گرفته‌اند که در میان آنها سه روش رایج و سه روش که کمتر مورد استفاده قرار می‌گیرند انتخاب شده‌اند. \lr{Logestic Regression} به عنوان روش رگرسیون انتخاب شده و نتایج نشان می‌دهد که روش‌های دسته‌بندی بهتر از روش رگرسیون عمل می‌کند. در میان روش‌های دسته‌بندی \واژه{درخت تصمیم} بهتر از سایرین عمل کرده است. 


\subsection{درشت‌دانگی پیش‌بینی }
در پژوهش‌های انجام شده مدل‌های پیش‌بینی در سطوح مختلفی از ریزدانگی ساخته شده‌اند از جمله: زیر سیستم، قطعه یا بسته، فایل یا کلاس، تابع و تغییر. \نام{هتا}{Hata} و همکاران  پیش‌بینی در سطح تابع را ارائه داده‌اند و به این نتیجه رسیده‌اند که پیش‌بینی خطا در سطح تابع نسبت به سطوح درشت‌دانه‌تر از نظر هزینه موثرتر است \cite{hata2012bug}. کیم و همکاران نیز مدل جدیدی ارائه داده‌اند که \واژه{دسته‌بندی}  تغییر نام دارد. بر خلاف سایر مدل‌های پیش‌بینی، "دسته‌بندی تغییر می‌تواند به طور مستقیم به توسعه دهنده کمک کند. این مدل می‌تواند زمانی که توسعه دهنده تغییری در کد منبع ایجاد می‌کند و آنرا در سیستم کنترل نسخه ثبت می‌کند، نتایج آنی را فراهم کند.  از آنجا که این مدل بر اساس بیش از ده هزار ویژگی ساخته می‌شود، سنگین‌تر از آن است که در عمل مورد استفاده قرار گیرد\cite{kim2008classifying}. \\


