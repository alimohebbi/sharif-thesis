% abstract ...
% write it at the end...
%persian abstract

 توسعه‌دهندگان نرم‌افزار از طریق گزارش خطا در سیستم­‌های ردگیری خطا و یا شکست در آزمون نرم‌‌افزار متوجه حضور خطا می‌شوند و پس از آن به جستجوی محل خطا و درک مشکل  نرم‌‌افزار می‌‌پردازند. کشف زود هنگام خطاها موجب صرفه‌‌جویی در زمان و هزینه می­‌شود و فرآیند اشکال زدایی را تسهیل می‌­بخشد. ابزارهای آماری نوین امکان ساخت و بهره‌‌برداری از مدل‌های پیش‌بینی را فراهم می‌سازند. اصلی‌ترین جزء مدل‌های پیش‌بینی، معیارهای نرم‌افزار می‌باشد و با به کارگیری معیارهای نوین و موثر می‌توان به مدل‌های کاراتر دست پیدا کرد. در این پایان‌نامه از معیارهای فرآیند و معیارهایی که بر اساس تحلیل جهش ساخته شده‌اند استفاده شده  و عملکرد مدل‌های حاصل ارزیابی شده‌اند. علاوه بر بکارگیری معیارهای جهش در کنار معیارهای فرآیند دو دسته معیار جدید به نام‌های معیارهای \موکد{فرآیند مبتنی بر جهش } و معیارهای \موکد{ترکیبی جهش-فرآیند} نیز جهت به کارگیری در ساخت مدل‌های پیش‌بینی معرفی شده‌اند. نتایج ارزیابی نشان می‌دهد معیارهای جهش می‌تواند به قدرت پیش‌بینی معیارهای فرآیند بیافزاید. معیارهای فرآیند مبتنی بر جهش علی‌رغم داشتن قدرت پیش‌بینی بهتر از معیارهای جهش عمل نمی‌کنند. همچنین معیارهای ترکیبی جهش-فرآیند بهبود قابل توجهی را در عملکرد مدل‌های پیش بینی ایجاد می‌کنند. 
