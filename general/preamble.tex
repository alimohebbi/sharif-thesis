 \usepackage{color}

\usepackage{amssymb}
\usepackage{mathrsfs}
\usepackage{algorithmicx}
\usepackage{graphicx}
\usepackage{multirow}
\usepackage{comment}
\usepackage[style=ieee,backend=biber]{biblatex}
\newcommand{\bibliographytitle}{\rl{کتاب‌نامه}}
\usepackage{paralist}
\usepackage{textcomp}
\usepackage{ctable}% for tables (it provides table-notes and imports booktabs package too)
\usepackage[xindy]{glossaries}
\newcounter{tablerow}
\renewcommand{\arraystretch}{1.5}
\usepackage{cleveref}
\usepackage{csvsimple}
\usepackage{longtable}
\usepackage{color}
\usepackage{setspace}
\usepackage[table]{xcolor}  
\usepackage{tablefootnote}
\usepackage{supertabular}
\usepackage{relsize}
\usepackage{stmaryrd}


 

\newcommand{\crefrangeconjunction}{--}
\crefformat{tablerow}{#2#1#3}
\crefmultiformat{tablerow}{#2#1#3}%
{ و~#2#1#3}%
{, #2#1#3}%
{ و~#2#1#3}
\crefformat{section}{#2#1#3}
\crefmultiformat{section}{#2#1#3}%
{ و~#2#1#3}%
{, #2#1#3}%
{ و~#2#1#3}



\input{general/translitaration}

\نوواژه[کلید]{نمونه}{Example}
\نوواژه{خودگرد}{Automata}
\نوواژه{قالب}{Template}
\نوواژه{پوشه}{Folder}
\نوواژه{پرونده}{File}
\نوواژه{پروندهی}{File}
\نوواژه{بسته}{Package}
\نوواژه{سامانه‌ی کنترل نسخه}{Version Control System}
\نوواژه{قطعه}{Component}
\نوواژه{خطا‌خیزی}{Bug-proneness}
\نوواژه{ماتریس درهم‌ریختگی}{Confusion Matrix}
\نوواژه{معیارهای محصول}{Product Metrics}
\نوواژه{فرض زمینه‌ای}{Ground Assumption} 
\نوواژه{جریان کنترلی}{Control Flow} 
\نوواژه{پیچیدگی حلقوی}{Cyclomatic Complexity}
\نوواژه{همبستگی}{Cohesion} 
\نوواژه{زوجیت}{Coupling}
\نوواژه{بازآرایی کد}{Refactoring} 
\نوواژه{ثبت}{Commit}
\نوواژه{شهرت}{Popularity}
\نوواژه{بررسی قاعده‌مند}{Systematic Review}
\نوواژه{تکرار}{Iteration} 
\نوواژه{انتشار}{Release} 
\نوواژه{جهت‌گیری}{Bias}
\نوواژه{نیمه-نظارتی}{Semi-Supervised}
\نوواژه{آماره‌های توصیفی}{Descriptive Statistics} 
\نوواژه{درخت تصمیم}{Decision Tree}
\نوواژه{دسته‌بندی}{Classification} 
\نوواژه{اشکال زدایی}{Debugging} 
\نوواژه{اندازه‌گیری  وکالتی}{Proxy Measurement}
\نوواژه{امتیاز جهش }{Mutation Score}
\نوواژه{جهش‌یافته}{Mutant}
\نوواژه{شرط شاخه}{Branch Condition}
\نوواژه{مشکوک بودن}{Suspiciousness}
\نوواژه{منسوخ}{Obsolete}
\نوواژه{ارزیابی میان دسته‌ای}{Cross-validation}
\نوواژه{توضیح}{Comment}
\نوواژه{مستند‌جاوا}{Javadoc}
\نوواژه{پیکربندی}{Configuration}
\نوواژه{متن-باز}{Open-source}
\نوواژه{چهارچوب}{Framework}
\نوواژه{وابستگی}{Dependency}
\نوواژه{خط دستور}{Command line}
\نوواژه{افزونه}{Plugin}
\نوواژه{کارایی}{Performance}
\نوواژه{عمومی}{Generic}
\نوواژه{سبک‌وزن}{Lightweight}
\نوواژه{حاشیه‌نویسی شده}{Annotated}
\نوواژه{ورودی/خروجی}{Input/Output (IO)}
\نوواژه{پوشه‌ی کاری}{Working Directory}
\نوواژه{ماشین مجازی جاوا}{Java Virtual Machine}
\نوواژه{فراداده}{Metadata}
\نوواژه{زمان خروج}{Timeout}
\نوواژه{زباله‌روب}{Grabage Collector}
\نوواژه{متعادل}{Balanced}
\نوواژه{فضای ویژگی}{Feature Space}
\نوواژه{ابرصفحه}{Hyperplane}
\نوواژه{مورد آزمون}{Test Case}
\نوواژه{انقیاد}{Bind} 
\نوواژه{ساخت}{Build}


\newcommand{\myrotate}[3][]{\rotatebox{90}{\parbox[c][#1]{#2}{\centering\arraybackslash\rl{#3}}}}
\newcommand{\multilinescell}[2][c]{\begin{tabular}[#1]{@{}c@{}}#2\end{tabular}}
\newcommand{\twolinescell}[3][c]{\multilinescell[#1]{#2\\#3}}
\newcommand{\itemrl}[1][]{\item[\rl{#1}]}
\eqcommand{چرخش}{myrotate}
\eqcommand{سلولچندخطی}{multilinescell}
\eqcommand{سلول‌دوخطی}{twolinescell}
\eqcommand{فقره‌راست}{itemrl}



\newcommand{\Eqn}[1]%
{فرمول~(#1) }
\newcommand{\Eqns}[1]%
{فرمول‌های~(#1) }

\eqcommand{فرمول}{Eqn}
\eqcommand{فرمولهای}{Eqns}

\فرمان‌نو{\نگا}{ن.بـ.}
\فرمان‌نو{\نگاص}[1]{\نگا{} صفحه‌ی~\رجوع‌صفحه{#1}}
\فرمان‌نو{\آیم}[1][$i$]%
{#1اُم}

\newcolumntype{C}[1]{>{\centering\arraybackslash}m{#1}}

\فرمان‌نو{\نامک}[1]{%
\fcolorbox{red}{yellow}{\begin{minipage}{\textwidth}#1\end{minipage}}}

\newcommand{\tobewritten}[1]{\نامک{%
این جعبه، باید با متن متناظر با \چر{#1} جای‌گزین گردد.

برای دریافتن معنی \چر{#1} به پرونده‌ی \چر{TODO} نگاه کنید.
}}

\graphicspath{{img/}}% tell tex engine address of folder containing your pictures


\hypersetup{
	pdftitle = {\enTitle},
	pdfauthor = {\enAuthor},
	pdfsubject = {\enSubject},
	pdfkeywords = {\enKeywords}
}



\renewcommand{\lstlistingname}{\rl{قطعه کد}}

\raggedbottom

\definecolor{codegreen}{rgb}{0,0.6,0}
\definecolor{codegray}{rgb}{0.5,0.5,0.5}
\definecolor{codepurple}{rgb}{0.58,0,0.82}
\definecolor{backcolour}{rgb}{0.95,0.95,0.92}
\definecolor{codeblue}{rgb}{0.11,0.191,0.80}

\lstdefinestyle{mystyle}{
	backgroundcolor=\color{backcolour},   
	commentstyle=\color{codegreen},
	keywordstyle=\color{magenta},
	numberstyle=\tiny\color{codegray},
	stringstyle=\color{codepurple},
	basicstyle=\footnotesize,
	breakatwhitespace=false,         
	breaklines=true,                 
	captionpos=b,                    
	keepspaces=true,                 
	numbers=left,                    
	numbersep=5pt,                  
	showspaces=false,                
	showstringspaces=false,
	showtabs=false,                  
	tabsize=2
}

\lstset{style=mystyle}

\fontdimen2\font=3pt

\renewcommand{\autodot}{} %remove last dots from title




