
\lstdefinestyle{appstyle}{
	backgroundcolor=\color{white},   
	commentstyle=\color{codegreen},
	keywordstyle=\color{magenta},
	numberstyle=\tiny\color{codegray},
	stringstyle=\color{codeblue},
	basicstyle=\footnotesize,
	breakatwhitespace=false,         
	breaklines=true,                 
	captionpos=b,                    
	keepspaces=true,                 
	numbers=left,                    
	numbersep=5pt,                  
	showspaces=false,                
	showstringspaces=false,
	showtabs=false,                  
	tabsize=2,
	basicstyle=\normalsize
}

\lstset{style=appstyle}

\chapter{ساخت مدل‌های پیش‌بینی و ارزیابی}

در این قسمت قطعه کدهای ساخت مدل‌های پیش‌بینی و ارزیابی آنها آورده شده است. قطعه کد \ref{code:data-set}  مجموعه‌داده‌ها را آماده می کند و تنظیمات مربوط به آموزش مدل‌ها را انجام می‌دهد. 

\begin{latin}
	\begin{lstlisting}[language=R]
library(RMySQL);
library(caret);
library(pROC);
library(e1071)

mydb = dbConnect(MySQL(), user='root', password='1', dbname='bug_predict', host='127.0.0.1');

rs_mutation_metric = dbSendQuery(mydb, "select * from MutationMetric");
mutation_metircs = fetch(rs_mutation_metric, n=-1);
rs_process_metric = dbSendQuery(mydb, "select * from ProcessMetric ");
process_metircs = fetch(rs_process_metric, n=-1);

##### clean up data #####
source("/home/ali/project/R-scripts/kill-live-to-score.R");

merged_metrics <- merge(x=clean_mutation_metircs,y=process_metircs, by.x="MetricId", by.y="ID")
lables<- as.factor(merged_metrics[,names(merged_metrics) %in% c("FILE_TYPE")]);


############test and train#############
b_number<-nrow(merged_metrics[merged_metrics$FILE_TYPE == "B",])
c_number<- nrow(merged_metrics[merged_metrics$FILE_TYPE == "C",])
smp_size_b <- floor(0.9 * b_number);
smp_size_c <- floor(0.9 * c_number);

## set the seed to make your partition reproducible
set.seed(1423)
train_ind_b <- sample(seq_len(b_number), size = smp_size_b)
train_ind_c <- sample(seq_len(c_number), size = smp_size_c)
train_ind_c <- train_ind_c + b_number
train_ind <- c(train_ind_b, train_ind_c)

#########train control##########
MyFolds <- createMultiFolds(merged_metrics[train_ind,4], k = 10, times=10)
train_control <- trainControl(method = "cv", index = MyFolds,
savePredictions = TRUE,
classProbs = TRUE
#    ,summaryFunction = twoClassSummary
)

	\end{lstlisting}
\end{latin}
\captionof{lstlisting}{آماده‌سازی مجموعه داده}
\label{code:data-set}
در قطعه کد \ref{code:clean} پاک‌سازی داده‌ها و تبدیل داده‌های جهش به امتیاز جهش انجام می‌شود.
\begin{latin}
	\begin{lstlisting}[language=R]
	clean_mutation_metircs<- mutation_metircs[!is.na(mutation_metircs$Covered),];
	
	for(i in 1:dim(clean_mutation_metircs)[1])
	{
	if(clean_mutation_metircs[i,'Lived']==-1)
	{
	clean_mutation_metircs[i,'Lived']<- 0;
	clean_mutation_metircs[i,'Killed']<-clean_mutation_metircs[i,'Killed']-1;
	}
	}
	
	temp<-clean_mutation_metircs;
	temp[,6]<-clean_mutation_metircs[,6]/clean_mutation_metircs[,5]
	temp[,7]<-clean_mutation_metircs[,6]/clean_mutation_metircs[,2]
	clean_mutation_metircs <- temp
	
	for(i in 1:dim(clean_mutation_metircs)[1])
	{
	if(is.nan(clean_mutation_metircs[i,'Lived']))
	clean_mutation_metircs[i,'Lived']<- 0;
	if(is.nan(clean_mutation_metircs[i,'Killed']))
	clean_mutation_metircs[i,'Killed']<- 0;
	}
	\end{lstlisting}
\end{latin}
\captionof{lstlisting}{تبدیل داده‌های جهش به امتیاز جهش }
\label{code:clean}
در قطعه کد \ref{code:modeling} مدل‌های پیش‌بینی ساخته می‌شوند و با استفاده از داده‌های آزمون پیش‌بینی انجام می‌گیرد. تمامی معیارها در متغیر merged\_metrics وجود دارند و با انتخاب ستون‌های مورد نظر زیر مجموعه‌ی مناسب انتخاب می‌شود. همچنین در تابع train روش دسته‌بندی انتخاب می‌گردد.

\begin{latin}
	\begin{lstlisting}[language=R]

############Process Metrics#############
p_features<-merged_metrics[,-c(seq(1,13),17,20)];
model1 <- train(p_features[train_ind,], lables[train_ind], trControl = train_control, method="nnet");
predict1_raw<-predict.train(model1, p_features[-train_ind,], type="raw")
predict1_prob<-predict.train(model1, p_features[-train_ind,], type="prob")


############Process Metrics with mutation#############

m_features1<-merged_metrics[,!names(merged_metrics) %in% c("FILE_TYPE","MetricId","FileType","FileInfoId","FILE_INFO_ID")];
m_features1<-m_features1[,-c(seq(5,10))];
model2 <- train(m_features1[train_ind,], lables[train_ind], trControl=train_control, method="nnet");
predict1_raw<-predict.train(model2, m_features1[-train_ind,], type="raw")
predict1_prob<-predict.train(model2, m_features1[-train_ind,], type="prob")


m_features2<-merged_metrics[,!names(merged_metrics) %in% c("FILE_TYPE","MetricId","FileType","FileInfoId","FILE_INFO_ID")];
m_features2<-m_features2[,-c(7,8)];
model3 <- train(m_features2[train_ind,], lables[train_ind], trControl=train_control, method="nnet");
predict2_raw<-predict.train(model3, m_features2[-train_ind,], type="raw")
predict2_prob<-predict.train(model3, m_features2[-train_ind,], type="prob")

	\end{lstlisting}
\end{latin}
\captionof{lstlisting}{ساخت مدل‌های پیش‌بینی}
\label{code:modeling}
در قطعه کد \ref{code:evluation} پیش‌بینی‌های انجام شده ارزیابی می‌شوند.

\begin{latin}
	\begin{lstlisting}[language=R]

auc(lables[-train_ind], predict1_prob$B)
auc(lables[-train_ind], predict2_prob$B)
auc(lables[-train_ind], predict3_prob$B)

plot(roc(lables[-train_ind], predict1_prob$B), col="blue", cex.lab=2, cex.axis=1.5);
plot(roc(lables[-train_ind], predict2_prob$B), add=TRUE, col="red", lty=2);
plot(roc(lables[-train_ind], predict3_prob$B), add=TRUE, col="black", lty=4);
 legend(x="bottomright",col=c("blue", "red", "black"),lwd=3,legend=c("Process","Process & Mutation", "Process & Mutation Base"),bty="6", lty=c(1,2,4), cex = 1.5)


confusionMatrix(predict1_raw, lables[-train_ind])
confusionMatrix(predict2_raw, lables[-train_ind])
confusionMatrix(predict3_raw, lables[-train_ind])

	\end{lstlisting}
\end{latin}
\captionof{lstlisting}{ساخت مدل‌های پیش‌بینی}
\label{code:evluation}
