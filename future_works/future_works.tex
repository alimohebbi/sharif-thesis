\فصل{نتیجه‌گیری و کارهای آتی}
\برچسب{chap:future}
در این پایانامه سعی شد که تاثیر معیار‌های جهش بر پیش‌بینی خطا در هنگام قرار گیری در کنار معیار‌های فرآیند ارزیابی  شود و معیارهای جدیدی با استفاده از مفاهیم تحلیل جهش و تاریخچه‌ی توسعه‌ی نرم‌افزار ارائه گردد. در فصل \ref{chap:intro} به بیان مسئله و مفاهیم مقدماتی پرداخته شد. در فصل \ref{chap:survey}  پژوهش‌های پیشین در حوزه‌ی پیش‌بینی خطا مورد بررسی قرار گرفت. پژوهشگران به طرق مختلف سعی در دستیابی به نتایج بهتری در پیش‌بینی خطا هستند. در این بررسی مشخص شد که در پژوهش‌های پیشین دو دسته‌ی کلی از معیارها مورد استفاده قرار گرفته است. این دسته‌ها عبارتند از معیارهای کد و معیارهای فرآیند. معیارهای فرآیند دارای مزیت‌ها بیشتری نسبت به معیارهای کد هستند و پژوهش‌های کمتری نیز به بررسی آنها پرداخته است. در یکی از پژوهش‌های اخیر از معیارهای جهش  در کنار معیارهای کد به منظور پیش‌بینی‌خطا استفاده گردیده و موجب بهبود پیش‌بینی شده است. \\

پس از مشخص شدن بخش‌هایی از این حوزه که نیازمند تحقیق بیشتر هستند و شناسایی پتانسیل‌های موجود در معیارهای فرآیند و جهش در فصل \ref{chap:method} راهکارهایی ارائه شدند تا با استفاده از معیارهای فرآیند و مفاهیم تحلیل جهش پیش‌بینی خطا بهبود یابد. در رویکرد اول معیارهای فرآیند در کنار معیارهای جهش قرار می‌گیرند و پیش‌بینی خطا با استفاده از آنها انجام می‌پذیرد. در رویکرد دوم، چهار معیار فرآیند مبتنی بر مفاهیم تحلیل جهش ارائه شده‌اند و در رویکرد سوم دو معیار فرآیند با استفاده از مفاهیم جهش اصلاح شدند و معیارهای ترکیبی جهش-فرآیند به وجود آمدند. \\

در فصل \ref{chap:case-study}  نحوه‌ی پیاده‌سازی هر یک از سه رویکرد ارائه شده و ابزارهای مورد استفاده شرح داده شد. به منظور انجام مطالعه‌ی موردی، پنج پروژه‌ی صنعتی جاوا مورد استفاده قرار گرفتند و معیارهای مورد بررسی در آنها استخراج شد. این معیارها برای دو گروه از پرونده‌ها که یکی حاوی خطا و دیگری سالم هستند محاسبه شده است. در این دو گروه تعداد یکسانی پرونده وجود دارد. پرونده‌های حاوی خطا در مجموعه‌داده‌ی \lr{defects4j} مشخص شده‌اند و پرونده‌های سالم به طور تصادفی انتخاب شدند. \\

معیارهای استخراج شده در فصل \ref{chap:evaluation} ارزیابی شدند.  مدل‌های پیش‌بینی با استفاده از  چهار روش دسته‌بندی ساخته شدند و عملکرد مدل‌ها با یکدیگر مقایسه گردید. نتایج ارزیابی نشان داد که معیارهای جهش زمانی که در کنار معیارهای فرآیند قرار گیرند می‌توانند تاثیر قابل توجهی در بهبود پیش‌بینی داشته باشند.\\
 معیارهایی که تحت عنوان فرآیند مبتنی بر جهش ارائه شدند، زمانی که در کنار معیارهای فرآیند قرار می‌گیرند موجب بهبود پیش‌بینی خطا می‌شوند اما توانایی آنها بیشتر از معیارهای جهش نیست. از آنجا که این دسته از معیارها هزینه‌ی محاسباتی بیشتری دارند جایگزینی آنها با معیارهای جهش نمی‌تواند مزیتی داشته باشد. همچنین قرارگیر همه‌ی این معیارها در کنار هم نیز تاثیر مثبت چندانی نخواهد داشت.\\
 معیارهای ترکیبی جهش-فرآیند به طور میانگین ۶ درصد در صحت، $6.6$ درصد در دقت و $5.1$ در مساحت زیر نمودار ROC تغییر مثبت ایجاد کرده است و از نظر معیار بازخوانی تغییر قابل توجهی ایجاد نمی‌کند. این تغییرات نشان می‌دهد که اصلاح  معیارهای فرآیند موفق آمیز بوده است و عرصه‌ی جدیدی را می‌توان به منظور ساخت معیارهای پیش‌بینی در نظر گرفت و این عرصه ارائه‌ی معیارهای ترکیبی است. همچنین با توجه به این نکته  که تولید جهش‌یافته نیازمند وجود موارد آزمون نیست می‌توان برای این معیارها دامنه‌ی کاربرد وسیع‌تری در نظر گرفت. 

در ادامه به گام‌هایی اشاره می‌شود که می‌توانند  به نتایج این پایانامه   جامعیت بخشند شود و ابعاد دیگری از بکارگیری این معیارها  مورد بررسی قرار گیرد.

\begin{itemize}
	\item
	\textbf{بررسی تاثیر استفاده از عملگرهای متفاوت:}\\
	در این پایانامه مجموعه‌ی محدودی از عملگرها جهت ساخت جهش استفاده شده است. در پژوهش‌های آتی می‌توان به این موضوع پرداخت که افزایش و یا  کاهش مجموعه‌ی عملگرهای جهش‌یافته چه تاثیری بر پیش‌بینی خطا داشته باشد. همچنین اینکه کدام نوع از عملگرهای مورد استفاده در استخراج معیارهای ارائه شده تاثیر بیشتری بر پیش‌بینی خطا دارد. 
	\item
	\textbf{ارزیابی معیارهای کد در کنار معیارهای ارائه شده:}\\
همانطور که بیان شد معیارهای جهش می‌توانند به معیارهای فرآیند کمک کنند تا پیش‌بینی دقیقتری انجام شود. از طرف دیگر استفاده از معیارهای کد نیز می‌تواند به معیارهای جهش کمک کند و این معیارها هزینه‌ی محاسباتی کمتری دارند. با توجه به پر هزینه بودن معیارهای جهش لازم است میزان بهبود پیش‌بینی  خطا توسط آنها با معیارهای کد مقایسه شود و مشخص شود  در هنگام قرار گیری در کنار معیارهای فرآیند مزیتی در مقابل معیارهای کد دارند یا خیر. 
\item
\textbf{ساخت چهارچوب پیش‌بینی خطا با استفاده از پژوهش موجود:}\\
استخراج معیارها و ساخت مدل‌های پیش‌بینی در این پایانامه به صورت خودکار انجام می‌گیرد. با ایجاد تغییرات لازم می‌توان چهارچوبی ارائه داده که برای سایر پروژه‌های نرم‌افزاری نیز این معیارها را استخراج کند. با ایجاد یک چهارچوب هم انجام پژوهش‌های آتی توسط سایرین سهولت می‌یابد و هم زمینه‌ی به کارگیری پیش‌بینی خطا در صنعت توسعه می‌یابد. 
	
	
\end{itemize}




